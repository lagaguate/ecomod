
% to reduce the size of the PDF:
% use ghostscript:
% gs -q -dSAFER -dNOPAUSE -sDEVICE=pdfwrite -sPDFSETTINGS=printer -sOutputFile="resdoc2006sc.pdf" resdoc2006sc.pdf 

% the key being sPDFSETTINGS: with options: default, screen, ebook, printer, preprint
% try also:  -sCompressPages=true
% -sDownsampleColorImages=true
% -sColorImageResolution=300
% -sGrayImageResolution=300
% -sMonoImageResolution=300
% final choice to get it under 8MB :

% gs -q -dSAFER -dNOPAUSE -sDEVICE=pdfwrite -dPDFSETTINGS=/printer -dColorImageResolution=150 -dMonoImageResolution=150 -dGrayImageResolution=150  -dCompatibilityLevel=1.4 -sOutputFile="resdoc2006sc.printer.pdf" resdoc2006sc.pdf  < /dev/null



\documentclass[11pt]{article}
\usepackage{SIunits}
\usepackage{fullpage}
\usepackage[top=2.4cm, bottom=2.4cm, left=3cm, right=3cm]{geometry}

\usepackage[english]{babel}
\usepackage{times}
% \usepackage{rotating}
\usepackage{lscape}

% \usepackage{array}
% \usepackage{bigstrut} for adding verticle space inside of tables

\usepackage[font=small,labelfont=bf,labelsep=period]{caption}
  \captionsetup[table]{position=top}

\newif\ifpdf

\ifx\pdfoutput\undefined
    \pdffalse           % we are not running PDFLaTeX
\else
    \pdfoutput=1        % we are running PDFLaTeX
    \pdftrue
\fi

% specify pdftex as a global option. This will tell the
% graphics, color and hyperref packages to generate pdftex
% compatible control commands.
\ifpdf
  \usepackage[pdftex,
        colorlinks=true,
        urlcolor=blue,
        filecolor=blue,
        linkcolor=blue,
        citecolor=blue,
        pdftitle={Snow crab RES DOC},
        pdfauthor={Snow crab},
        pdfsubject={Assessment},
        pdfkeywords={snow crab research document scotian shelf},
 %       pagebackref,  % to ident pages where a reference exists .. good for editorial checks
        pdfpagemode=None,
        bookmarksopen=true]{hyperref}
  \pdfcompresslevel=9
  \usepackage[pdftex]{graphicx}
\else
  \usepackage[
        colorlinks=true,
        urlcolor=blue,               % \href{...}{...}
        anchorcolor=blue,
        filecolor=green,             % \href*{...}
        linkcolor=red,               % \ref{...} and \pageref{...}
        citecolor=black,
        menucolor=blue]{hyperref}
  \usepackage{graphicx}
\fi


\usepackage[pdftex]{graphicx}
\pdfcompresslevel=9


\usepackage{color}
  \definecolor{red}{rgb}{0.75,0,0}
  \definecolor{green}{rgb}{0,0.5,0}
  \definecolor{blue}{rgb}{0,0,0.75}
  \definecolor{black}{rgb}{0,0,0}
  \definecolor{gray}{rgb}{0.75,0.75,0.75}
  \definecolor{rltbrightred}{rgb}{1,0,0}
  \definecolor{rltred}{rgb}{0.75,0,0}
  \definecolor{rltdarkred}{rgb}{0.5,0,0}
  \definecolor{rltbrightgreen}{rgb}{0,0.75,0}
  \definecolor{rltgreen}{rgb}{0,0.5,0}
  \definecolor{rltdarkgreen}{rgb}{0,0,0.25}
  \definecolor{rltbrightblue}{rgb}{0,0,1}
  \definecolor{rltblue}{rgb}{0,0,0.75}
  \definecolor{rltdarkblue}{rgb}{0,0,0.5}
  \definecolor{webred}{rgb}{0.5,.25,0}
  \definecolor{webblue}{rgb}{0,0,0.75}
  \definecolor{webgreen}{rgb}{0,0.5,0}


\usepackage{natbib}
\bibpunct[, ]{(}{)}{;}{a}{}{,}
% \citeindextrue

% \citeindextrue
% see  /usr/share/texmf/tex/latex/natbib/natbib.sty for more codes
% natbib examples
%\citet{goossens93}     Goossens et al. (1993)
%\citep{goossens93}     (Goossens et al., 1993)
%\citet*{goossens93}    Goossens, Mittlebach, and Samarin (1993)
%\citep*{goossens93}    (Goossens, Mittlebach, and Samarin, 1993)
%\citeyear, \citeauthor
%\citealt


\usepackage{amsmath}
\usepackage{wrapfig}
%\usepackage{floatfig}

\setlength{\parskip}{12 pt} % spacing between paragraphs
\setlength{\parindent}{0cm}
\setlength{\floatsep}{2cm}


\newcommand*{\D}{.}  % required in graphics names as periods seem to confuse the parsing mechanism
\newcommand{\lt}{$<$}
\newcommand{\gt}{$>$}
\newcommand{\minus}{$-$}
\newcommand{\plus}{$+$}
\newcommand{\sq}{$^2$}
\newcommand{\cube}{$^3$}
\newcommand{\thou}{10$^3$}
\newcommand{\mill}{10$^6$}
\newcommand{\mult}{$\times$}
\newcommand{\kmsq}{km\sq}
\newcommand{\Xthou}{$\times$~10$^3$}
\newcommand{\Xmill}{$\times$~10$^6$}
%\newcommand{\degreesC}{${}^\circ \mbox{C}$ {}}
%\newcommand{\degreesK}{$\mbox{K}$}
%\newcommand{\carbondioxide}{{CO${}_{2}$}}

\begin{document}

\DeclareGraphicsExtensions{.png,.pdf,.jpg,{}}
%
%
% \ifx\pdfoutput\undefined % We're not running pdftex
%   \DeclareGraphicsExtensions{.eps,.mps,{}}  % the last {} is in case there is no extension
% \else
% \DeclareGraphicsExtensions{.png,.pdf,.jpg,{}}  % the last {} is in case there is no extension
% %  \DeclareGraphicsExtensions{.png,.jpg,.pdf,.mps,{}}  % the last {} is in case there is no extension
% \fi

% ---- CSAS coverpage begin ----

\thispagestyle{empty} % removes page numbers
  \noindent
  {\footnotesize
    \begin{tabular*}{15cm}{@{\extracolsep{\fill}}p{7.5cm}p{7.5cm}}
      \includegraphics{../common/DFOScience_bw.png} & {} \\
      {} & {} \\[9pt]
      {\large\textbf{\textsl{C~S~A~S}}} & {\large\textbf{\textsl{S~C~C~S}}} \\[3pt]
      \textbf{Canadian Science Advisory Secretariat} &
      \textbf{Secr\'{e}tariat canadien de consultation scientifique}
      \\[3pt]
      \hline
      \\[3pt]
      \textbf{Research Document 2007/017} &
      \textbf{Document de recherche 2007/017}
      \\[18pt]
      Not to be cited without\newline \underline{permission of the author}$^*$  &
      Ne pas citer sans\newline \underline{l'autorisation des auteurs}$^*$ \\
    \end{tabular*}
  }

  \begin{tabular*}{15cm}{@{\extracolsep{\fill}}p{7.5cm}p{7.5cm}}
      \large\textbf\raggedright{An assessment of the snow crab resident on the Scotian Shelf in 2006} &
      \large\textbf\raggedright{\'{E}valuation de stock du crabe des neiges du plateau continental de la Nouvelle-\'{E}cosse en 2006}
  \end{tabular*}

  \noindent\begin{normalsize}\begin{minipage}[t][7.5cm][c]{15cm}
  \begin{center}
       Choi, J.S. and Zisserson, B.M. \\[20pt]

    \begin{small}
      Population Ecology Division \\
      Department of Fisheries and Oceans \\
      Bedford Institute of Oceanography \\
      P.O. Box 1006, Dartmouth, Nova Scotia B2Y 4A2 \\
    \end{small}
  \end{center}
  \end{minipage}
\end{normalsize}

%\vspace{0.1cm}


\sloppy\noindent{\footnotesize\begin{tabular*}{15cm}{@{\extracolsep{\fill}}p{7.5cm}p{7.5cm}}
$^*$This series documents the scientific basis for the evaluation
of fisheries resources in Canada.  As such, it addresses the
issues of the day in the time frames required and the documents it
contains are not intended as definitive statements on the subjects
addressed but rather as progress reports on ongoing
investigations.
&
$^*$La pr\'{e}sente s\'{e}rie documente les bases scientifiques des
\'{e}valuations des ressources halieutiques du Canada.  Elle
traite des probl\`{e}mes courants selon les \'{e}ch\'{e}anciers
dict\'{e}s. Les documents qu'elle contient ne doivent pas \^{e}tre
consid\'{e}r\'{e}s comme des \'{e}nonc\'{e}s d\'{e}finitifs sur
les sujets trait\'{e}s, mais plut\^{o}t comme des rapports
d'\'{e}tape sur les \'{e}tudes en cours.
\\[3pt]
Research documents are produced in the official language in which
they are provided to the Secretariat.
&
Les documents de recherche
sont publi\'{e}s dans la langue officielle utilis\'{e}e dans le
manuscrit envoy\'{e} au secr\'{e}tariat.
\\[3pt]
This document is available on the Internet at:
&
Ce document est disponsible sur l'Internet \`{a}: \\[3pt]
\multicolumn{2}{c}{http://www.dfo-mpo.gc.ca/csas/} \\[3pt]

\hline
\small{} & \small{} \\
\multicolumn{2}{c}{ISSN 1499-3848 (Printed / Imprim\'{e})} \\
\multicolumn{2}{c}{\copyright Her Majesty the Queen in Right of Canada, 2007} \\
\multicolumn{2}{c}{\copyright Sa Majest\'{e} la Reine, Chef du Canada, 2007} \\[4pt]
\multicolumn{2}{c}{\scalebox{1}{\includegraphics{../common/canada.pdf}}}
\\
\end{tabular*}}

\newpage
% ---- CSAS coverpage end----


\thispagestyle{empty} \ \newpage % removes page numbers (do not remove the space between \ \    
% blank page required after coverpage



\newpage
\pagenumbering{roman}
\setcounter{page}{3}

\section*{Abstract}

For the 2006 fishing season, landings of snow crab were 486 and 4486~t for N-ENS and S-ENS, respectively. Relative to 2005 landings, this represents a decline of 14\% and 30\%, respectively. These declines were due to reductions in TACs, both of which were attained. Most landings were obtained from offshore areas. Average nonstandardised catch rates were 35.6~kg/trap and 90.6~kg/trap for N-ENS and S-ENS, respectively. N-ENS catch rates increased 16\% relative to 2005 levels while those of S-ENS decreased by 17\%. The increase in catch rates for N-ENS was largely due to the re-distribution of fishing effort away from areas of low catch rates to those with higher catch rates in the 2006 season. The offshore slope areas in S-ENS were minimally exploited.

Mature female numbers have increased as the leading edge of immature female pulses have begun to mature. Mature male numbers have declined, resulting in increasingly more balanced sex ratios. The fishable biomass in N-ENS declined 50\% from 1,460~t in 2005 to 720~t in 2006. In S-ENS, the fishable biomass increased (10\%) for the first time since 2001: from 23.4~\mult \thou ~t in 2005 to 25.4 \mult \thou t in 2006. Recruitment to the mature fishable biomass is expected for S-ENS in 2007 and for N-ENS in 2008. The 2007 season should see the beginning of a recovery of the fishery in S-ENS. The short-term future of the fishable biomass in N-ENS is uncertain. The medium term prospect for ENS in general is positive, if moderate exploitation rates are adopted. Soft-shell incidence will become increasingly an important issue as recruitment continues.

Due to positive signs of recruitment, fishable biomass and female reproduction in S-ENS, a status-quo to marginal increase in TAC is advised. Due to negative signs in fishable biomass and delayed recruitment of immature snow crab in N-ENS, a decrease in TAC is advised. 


\newpage
\section*{R\'{e}sum\'{e}}

En 2006, les d\'{e}barquements total du crabe des neiges du plateau continental de la Nouvelle-\'{E}cosse \'{e}taient de 486~t dans la r\'{e}gion nord (N-ENS) et de 4,486~t dans la r\'{e}gion sud (S-ENS). Ceci repr\'{e}sente un d\'{e}clin associ\'{e} aux r\'{e}ductions des allocations totale des captures (ATC) de 14\% et 30\% par rapport 2005, pour N-ENS et S-ENS, respectivement. La plupart des d\'{e}barquements provenaient loins de la c\^{o}te. Les taux de capture de la p\^{e}che commerciale \'{e}tait de 35.6 et 90.6~kg/casier lev\'{e} pour N-ENS et S-ENS, respectivement. Les taux de capture de N-ENS ont augment\'{e} de 16\% par rapport 2005, alors que ceux de S-ENS ont diminu\'{e} de 17\%. Cependant, en raison de la restructuration des zones de p\^{e}che commerciale, des taux de capture plus \'{e}lev\'{e}s avaient \'{e}t\'{e} pr\'{e}vus pour N-ENS, a priori. L'exploitation sur la pente continentale \'{e}tait minimale en 2006.

Le nombre de femelles adultes a augment\'{e} avec la maturation des femelles juv\'{e}niles observ\'{e}es depuis les quatre derni\`{e}res ann\'{e}es, alors que celui des m\^{a}les adultes a diminu\'{e}. Pour N-ENS, l'indice d'abondance commerciale relev\'{e} au chalut a diminu\'{e} en 2006 et \'{e}tait de 720~t, continuant la tendance \`{a} la baisse observ\'{e}e depuis la fin des ann\'{e}es 1990. Ceci repr\'{e}sente une diminution de 50\% par rapport \`{a} 2005. Pour la premi\`{e}re fois depuis la fin des ann\'{e}es 1990, une augmentation de l'indice d'abondance commerciale relev\'{e} au chalut a \'{e}t\'{e} observ\'{e}e dans la r\'{e}gion S-ENS (25.4 \mult \thou t, un augmentation de 10\% par rapport \`{a} 23.4~ \mult \thou ~t en 2005). Le recrutement de la biomasse commerciale pr\'{e}vue pour 2007 est bas pour N-ENS, alors que celui pour S-ENS est mod\'{e}r\'{e}. La saison 2007 devrait donc voir le d\'{e}but d'un r\'{e}tablissement de la p\^{e}che pour S-ENS. Pour N-ENS, des incertitudes continue en 2007. L'avenir g\'{e}n\'{e}ral \`{a} moyen terme pour ENS est relativement positif avec l'adoption d'un taux d'exploitation mod\'{e}r\'{e}. Une incidence plus \'{e}lev\'{e}e de crabe \`{a} carapace molle est pr\'{e}vu pour 2007. 

Pour N-ENS, en raison des d\'{e}clins continus de la biomasse commerciale et des incertitudes concernant le recrutement, une approche conservatrice de p\^{e}che est recommand\'{e}e pour la saison 2007. Pour S-ENS, en raison des ameliorations du recrutement, de la biomasse commerciale, et des females reproductives, une approche ``status-quo'' de p\^{e}che est recommand\'{e}e pour la saison 2006.



%\tableofcontents

%\listoftables

%\listoffigures


\newpage
\pagenumbering{arabic}
\setcounter{page}{1}

\section{Introduction}

In keeping with the demands of the Fishery Act to manage fisheries in an ecosystem context, we attempt to delineate and assess the basic structural and functional roles of the snow crab and the snow crab fishery on the Scotian Shelf ecosystem. Due to the intrinsic complexities of natural ecosystems, this initial attempt at developing an ecosystem-based assessment is necessarily an incomplete work. It will always remain a work-in-progress, that with the help of all parties will improve iteratively into the future. 


\section{History of the fishery and management}

The snow crab fishery in eastern Canada began in 1960 with incidental by-catches by groundfish draggers near Gasp\'{e}, Quebec \citep{Elner1986}. Its development was slow until the 1980s when it began expanding rapidly to become one of the largest fisheries in Canada (93,000~t in 2001; \citealt{Dufour2003}). 

On the Scotian Shelf, the snow crab fishery is managed as three main areas: N-ENS, S-ENS and CFA 4X (Figure~\ref{fig:map.cfa.locations}; Table~\ref{tab:management.measures}). There is no biological basis for these divisions: they represent ad hoc divisions based upon historical (i.e., political-social-economic) convenience. From 1982 to 1993, the management of this fishery was based on effort controls (size, sex, shell-hardness, season, license, trap limits). Additional management measures were introduced from 1994 to 1999: individual boat quotas (IBQs), total allowable catches (TACs), 100\% dockside monitoring, mandatory logbooks and at-sea monitoring by certified observers. Voluntary management measures requested by fishermen were also introduced in some areas, such as a shortened fishing season and reduced numbers of traps. For 2005, many areas and subareas were merged (except for Crab Fishing Areas or CFAs 23A,  CFA23 and CFA24) as was the designation of a temporary licence holder.

The snow crab fishery on the Scotian Shelf has been in existence since the late 1970's (Figures~\ref{ts.effort}, \ref{ts.landings}). The earliest records of landings were at levels of \lt~1,000~t, mostly in the near-shore areas. By 1979, this rose to 1,500~t subsequent to which the fishery declined substantially in the mid-1980s and was considered a collapsed fishery. Strong recruitment to the fishery returned in 1986 and since that time, landings have increased considerably (Figure~\ref{ts.landings}).

A trawl survey was developed as a fisheries-independent of assessment of the snow crab on the Scotian Shelf by the Gulf Fisheries Centre, Moncton, New Brunswick (GFC) in 1996 and accepted in 1999. These surveys demonstrated the presence of unexploited stocks in the south-east areas of the shelf were identified. This lead to large increases in TACs, especially in S-ENS (Tables~\ref{tab:fisheries.summary.north}, \ref{tab:fisheries.summary.south}). New temporary allocations were also granted to areas outside of the traditional fishing grounds. Catch rates continued to increase during this period before 2000 (Figure~\ref{ts.cpue}) as did fishing effort (Figure~\ref{ts.effort}). Following voluntary trap surveys in 2000 on the slope edge, additional allocations (200~t) were made in 2001 and 2002 with further additions in 2003 to 300~t. Annual TACs increased to historical maxima in 2002/2003 at 9,113 t in S-ENS and 1,500 t in N-ENS; landings were stable at approximately 10,000~t each year between 2000 to 2004. The \textit{post-1998 period was one of rapid expansion} of both the economic importance of the crab fishery and also the spatial extent of exploitation. However in 2004, with the persistently low levels of recruitment and a steady decline in fishable biomass since the early-2000s, a more precautionary tactic was adopted from 2004 to 2006. For the 2006 season, TACs were reduced to 4997~t (Tables~\ref{tab:fisheries.summary.north}, \ref{tab:fisheries.summary.south}). 

For the 2006 season, an updated soft shell protocol was adopted due to the possibility of an increased incidence of soft-shelled snow crab and the potential harm associated with handling mortality. The relative abundance of soft-shelled crab observed by at-sea-observers were relayed within 24 hours of landing to DFO which was subsequently  plotted on a 2 minute grid and broadcast to all members of industry. Fishers were requested to voluntarily avoid fishing within 1.5 nautical miles of the points with greater than 20\% soft crab. This voluntary response allowed fishers to move fishing gear away from or altogether avoid potentially problematic areas (helping also to save their time and fuel). Further, the return to the sea of immature legal sized crab (“pencil-clawed’ crab) was implemented in 2006 to allow these crab to moult to maturity and maximise the total meat yield per individual crab captured. 

The fishing seasons for 2006 are summarised in Table~\ref{tab:management.measures}. A two week extension was granted in CFA 23. Such extensions can be detrimental to the fishery due to the increased trapabilty of ``white'' crab that represent the new recruitment into the next year's fishery (crab newly moulted in the spring with low meat-yield); and increase the uncertainties associated with the assessment advice as they force alterations in the timing of the sampling of snow crab by the research surveys. 

\section{General biology and life history}

The snow crab (\textit{Chionoecetes opilio}, Brachyura, Majidae, O. Fabricius) is a subarctic species resident along the east coast of North America from northern Labrador to the Gulf of Maine. On the Scotian Shelf, commercially fished snow crab are generally observed between depths of 60 to 280~m (Figures~\ref{habitattemplate.depth}, \ref{habitattemplate.cpue.depth}) and between temperatures of \minus 1 to 6~\celsius {} (Figure~\ref{habitattemplate.temperatures}). Near 7~\celsius, metabolic costs are thought to match metabolic gains \citep{Foyle1989}. Snow crab are generally observed on soft mud bottoms although small-bodied and moulting crabs are also found on more complex (boulder, cobble) substrates \citep{Sainte-MarieHazel1992, Comeau1998}.

Snow crab eggs are brooded by their mothers for 2 years or more, depending upon ambient temperatures, food sources and the maturity status of the mother (27 months in primiparous females -- first breeding event; and 24 months in multiparous females -- second or possibly third breeding events; \citealt{Sainte-Marie1993}). More rapid development of eggs (from 12 to 18 months) have been observed in other systems \citep{Elner1995}. A primiparous female of $\sim$~57.4~mm carapace width (CW) would produce between 35,000 to 46,000 eggs which are extruded between February and April (in the Baie Sainte-Marguerite; \citealt{Sainte-Marie1993}). The actual range of fecundity is however quite large, especially as multiparous females are thought to be more fecund with more than 100,000 eggs being produced by each female. Eggs are hatched from April to June when the larvae become pelagic, feeding upon the plankton for 3 to 5 months (zoea stages 1 and 2 and then the megalopea stage). The larvae begin to settle to the bottom in autumn to winter (September to October in the Gulf area). On the Scotian Shelf, pelagic stages seem to have highest abundance in October and so may begin settling as late as January. Very little is known of survival rates at these early life stages.

Once settled to the bottom (benthic phase), snow crab grow rapidly, moulting $\sim$~twice a year \citep{Sainte-Marie1995, Comeau1998}. The first inter-moult stage (instar 1) is $\sim$~3~mm~CW. After the 5th instar (15~mm~CW) the frequency of moults decine, moulting occurring once a year in the spring until they reach a terminal maturity moult. Growth is allometric with weight increasing $\sim$~250\% with each moult (Figures~\ref{male.growth.stanzas}, \ref{growth.curves}; Tables~\ref{table.cw.stages}, \ref{table.mass.stages}). On the Scotian Shelf, the terminal moult has been determined to occur between the 9th to the 13th instar in males and the 9th to 10th instar in females (see Results). Just prior to the terminal moult, male crab may skip a moult in one year to moult in the next \citep{ConanComeauRobichaud1992}. Male snow crab generally reach legal size ($\geq$~95~mm~CW) by the 12th instar; however, a variable fraction of instar 11 snow crab are also within legal size. Male instar 12 snow crab represent an age of $\sim$~9~years since settlement to the bottom and 11 years since egg extrusion. Thereafter, the life expectancy of a male is $\sim$~5 to 6 years. Up to 10 months are required for the shell to harden (carapace conditions 1 and early 2; see Table~\ref{table.carapace.condition.description}) and up to 1 year for meat yields to be commercially viable. After hardening of the carapace (carapace conditions 3 to 4) the male is able to mate. Near the end of the lifespan of a snow crab (carapace condition 5), the shell decalcifies and softens, often with heavy epibiont growth. In warmer environments (e.g., continental slope areas), epibiont growth occurs at an accelerated rates creating some uncertainty in the classification of carapace condition 5 crab.

Females reproducing for the first time (primiparous females) generally begin their moult to maturity at an average size of 60~mm~CW and mate while their carapace is still soft (early spring: prior to the fishing season in ENS, and during the fishing season in CFA 4X). A second mating period later in the year (May to June) has also been observed for multiparous females \citep{Hooper1986}. Complex behavioural patterns have also been observed: the male helps the primiparous female moult, protects her from other males and predators and even feeds her (indirectly; \citealt{Hooper1986}). Pair formation (a mating embrace where the male holds the female) may occur up to 3 weeks prior to the mating event \citep{Hooper1986}. Upon larval release, males have been seen to wave the females about to help disperse the larvae (i.e., prior to a multiparous mating). Females are selective in their mate choice, as is often the case in sexually dimorphic species, and have been seen to die in the process of resisting mating attempts from unsolicited males \citep{Watson1972, Hooper1986}. Males compete heavily for females and often injure themselves (losing appendages) while contesting over a female. Larger males with larger chela are generally more successful in mating.


\section{Ecosystem considerations}

\subsection{Factors controlling the abundance of snow crab}

Anecdotal information from fishers and fishery-based catch rates (Figure~\ref{ts.cpue}) suggests that the abundance of snow crab was historically (prior to 1980) quite low in the near-shore areas of the Scotian Shelf. Increases in catch rates were observed throughout the shelf in the mid-1980s and 1990s in N- and S-ENS, respectively. As commercially exploitable snow crabs require 9 years or more from the time of settlement to reach the legal size of 95~mm~CW, their increasing dominance on the shelf must have had their origins as early as the late-1970s and 1980s (N- and S-ENS, respectively). For S-ENS, these time-lines are confounded by the expansion of the fishing grounds towards increasingly offshore areas and the exploiting of previously unexploited crab populations. However most of this expansion was observed in the post-2000 period when total allowable catches (TACs) and landings increased up to 6 fold relative to the TACs and landings of the 1990s (Figure~\ref{ts.landings}) and a doubling of fishing effort (Figure~\ref{ts.effort}). The catch rate increases observed in the 1980s and 1990s therefore were likely reflecting real increases in their abundance.

The possible causes of this increase in abundance can be broken down into four main categories of influence, or \emph{controlling factors}:

\begin{enumerate}
  \item \textbf{Top-down} influences refer to the \emph{role of predators} in controlling a population \citep{PaineRT1966, Tremblay1997, Worm2003}. The capacity of predatory groundfish to opportunistically feed upon snow crab \citep{Robichaud1991}, in combination with their numerical dominance prior to the 1990s \citep{Choi2004, Choi2005, FrankKT2005}, suggests that they may have been an important regulating factor controlling the recruitment of snow crabs. For example, snow crab in the size range of 5 to 30~mm~CW (with a 7~mm~CW mode; that is instars 2 to 7, with instar 7 being strongly selected) were targeted by thorny skate and cod. Soft-shelled males in the size range of 77 to 110~mm~CW during the spring moult were also a preferred food item. The demise of these predatory groundfish in the post-1990 period and the resultant release from predation upon the immature and soft-shelled crabs may have been an important determinant of the current rise to dominance of snow crab on the Scotian Shelf.

    Historically, the known predators of snow crab have been, in order of importance: Atlantic halibut (\textit{Hippoglossus hippoglossus}), skates (especially thorny skate, \textit{Raja radiata}), Atlantic cod (\textit{Gadus morhua}), seals, American plaice (\textit{Hippoglossoides platessoides}), squids, and other crabs \citep{Bundy2004}. In particular, Atlantic cod and thorny skate have been noted for their high selectivity for snow crab and therefore their potential to weaken recruitment to commercial sizes \citep{Bailey1982, Lilly1984, Robichaud1989, Robichaud1991}. Certainly, in the inshore areas of the Scotian Shelf, the anecdotal information that extremely high densities of these early stage snow crab are found in lobster traps indicates some degree of habitat overlap with adult lobsters. This suggests that one hypothesis for the current increase in lobster abundance in ENS may in part be related to the food base that the juvenile snow crab represent to lobsters. Predation levels upon small immature crabs are also likely to be on the rise in certain offshore areas. High local densities of these more traditional groundfish are found in areas where small immature crab are found in high densities (Figures~\ref{map.predators.cod}, \ref{map.predators.thorny.skate}). 

Seals are considered by fishermen to be a potential predator of snow crab. While they have on occasion been observed with snow crab in their stomachs, it should also be emphasised that the highest concentrations of snow crab are currently found in the immediate vicinity of Sable Island, an area where the abundance of grey seals are extremely high. The actual evidence indicating that seals have a negative influence upon the snow crab population therefore seems to be minimal. In fact, the seals may be having a positive influence by physically importing food and food waste (organic matter) from other more outlying areas to the immediate vicinity of Sable Island and so indirectly ``feeding'' the snow crab and even removing potential predators of crab.


% % Figure~30
% %
% \begin{figure}[ht]
% \begin{center}
% \rotatebox{0}{\scalebox{0.4}{\includegraphics*{figs/tsnonpelagicvw.png}}}
% \end{center}
% \caption{Variations of the biomass ($ log_{10} mass \:\: [kg \:\: m^{-2}] $) of groundfish and elasmobranchs as a function of time based upon summer groundfish research surveys for the eastern part of the Scotian Shelf (NAFO area 4VW). 1 standard error bars are presented. Note the decline of biomass in the 1990s and a subsequent increasing trend since 1999. Note also the peak biomass of groundfish and elasmobranchs during the period of the snow crab fishery ``crash'' in the mid-1980s.}
% \label{ts.nonpelagic.vw}
% \end{figure}

%    fish that preferentially feed upon crabs are:
%    species = c( seals, birds, cod, haddock, am.plaice, halibut,
%                 flatfish, skates, dogfish, pollock, demersals, squid, large.crab)
%
%    percent of each predator's diet :
%    small.crabs = c( 0.033, 0.015, 0.0355, 0.007, 0.028, 0.103,
%                     0.005, 0.119, 0.019, 0.001, 0.0255, 0.023, 0.025)
%    large.crabs = c( 0,     0    , 0.001, 0    , 0     , 0,
%                     0,     0.002, 0,     0    , 0.005, 0, 0)

\item
  \textbf{Bottom-up} influences refer to changes in a population due to \emph{resource (food) availability}. Diet studies and field observations \citep{Hooper1986, Bundy2004} indicate that the primary food items of larger (mature) crab are, in order of importance: echinoderms, polychaete worms (\textit{Maldane} sp., \textit{Nereis} sp.) and other worm-like invertebrates, detritus, large zooplankton, shrimps, smaller crabs (Rock crab, \textit{Cancer irroratus}; Toad and Lesser toad crabs, \textit{Hyas coarctatus}, \textit{Hyas araneus}), ocean quahog (\textit{Artica islandica}), bivalve molluscs (e.g., \textit{Mytilus edulis, Modiolus modiolus}), brittle stars (\textit{Ophiura sarsi}, \textit{Ophiopholis aculeata}) and sea anemones (\textit{Edwardsia} sp., \textit{Metridium senile}). Smaller crabs primarily feed upon, in order of importance: echinoderms, polychaete worms, large zooplankton, detritus and bivalves (e.g., \textit{Mytilus edulis, Modiolus modiolus, Hiatella arctica}). More recent studies have also demonstrated that cannibalism is also highly prevalent in intermediately sized (morphometrically) mature crabs, especially mature females \citep{Sainte-Marie2002, Squires2003}. 
  
  Most of these food items are part of the detrital food web, and so the proliferation of snow crab under the hypothesis of bottom-up control would be indicative of the proliferation of the detrital subsystem (potentially at the expense of the the other parts of the shelf ecosystem, including that of the demersals). This hypothesis is consistent with what is known of the current structure of the Scotian Shelf ecosystem \citep{Choi2005}: 
  \begin{itemize}
    \item Phytoplankton abundance in the most recent decade (1991-2001) was considerably higher and more variable than in the 1960s and early 1970s. This likely resulted in increased sedimentation of organic matter to the ocean bottom \citep{Choi2005}.
    \item The recent proliferation of northern shrimp (\textit{Pandalus borealis}), a detritivore and a potential food item of snow crab (Figure~\ref{map.prey.northern.shrimp}) was co-incident with the rise in abundance of snow crab.
    \item The demise of the groundfish that would competitively feed upon benthic invertebrates.
  \end{itemize} 

  Certainly the rapid rate of increase in abundance of snow crab would seem to indicate that resource competition was not a limiting factor, up to the late 1990s. However, the recent declines in their abundance since 2002 may be an indication that some form of a carrying capacity had been reached. This interpretation is of course confounded by the heavy exploitation of crabs at levels of almost the same order of magnitude as their biomass and the more recent absence of recruitment on the Scotian Shelf since 2002/3.


% \begin{figure}[ht]
% \begin{center}
% \rotatebox{0}{\scalebox{0.8}{\includegraphics*{R/cpr/CPRcolour.png}}}
% \end{center}
% \caption{Timeseries of an index of phytoplankton abundance (colour intensity) from the Continuous plankton recorder monitoring program over the Scotian Shelf. Note the very different productivity of the system inferred from this data series. }
% \label{ts.cpr.colour}
% \end{figure}



%    Examples:
%    food.species    = c( small.crab, shrimp, echinoderms, polychaetes, bivalve.molluscs,
%                       misc.benthic,inverts, large.zoopl, detritus)
%    diet.large.crab = c( 0.025, 0.085, 0.243, 0.108, 0.007, 0.284, 0.148, 0.1)
%    diet.small.crab = c( 0,     0,     0.153, 0.144, 0.012, 0.435, 0.157, 0.1)

\item
  \textbf{Lateral and internal influences}: refers to the \emph{competitive interactions} with groundfish, other crab species, cannibalism and reproduction-induced mortality (direct and indirect). The diet of snow crab overlap in many ways with that of groundfish, thus the demise of groundfish in the late 1980s and early 1990s would have been doubly beneficial to snow crab: reduction in predation pressure and also resource competition. The spatial distribution of snow crab overlaps with that of other crab species (Figures~\ref{fig:othercrabs.lessertoadcrab}, \ref{fig:othercrabs.jonahcrab}). The centers of high abundance may be areas where interactions may be strong causing resource competition, not only for food but habitat space as well. Of course these interactions are complex in that crabs may also serve as predators upon small snow crab as well as being food items for large snow crab. Nonetheless, where concentrations are elevated, the probability of strong negative (competitive) interactions are greater.


\item
  \textbf{Environmental influences}: refers to the \emph{abiotic influences} upon organisms, such as the physical habitat or temperature variations, oxygen concentrations, etc. Altered environmental conditions over extended periods of time have been observed on the Scotian Shelf (Figures~\ref{maps.bottom.temperature}, \ref{ts.habitat.surfacearea}, \ref{ts.habitat.surfacearea.temperatures}). Prior to 1986, the shelf was characterised by relatively warm bottom waters, low volume of the cold intermediate layer of water, and a Gulf Stream frontal position close to the shelf. The post-1986 period transitioned to an environment of cold bottom waters, a high volume of cold-intermediate layer waters, and a Gulf Stream frontal position distant from the shelf. The principal cause of the cold conditions is thought to have been along-shelf advection from both the Gulf of St. Lawrence and southern Newfoundland, and local atmospherically-induced, cooling. In the southwestern areas (Emerald Basin), the offshore Warm Slope water kept subsurface temperatures relatively warm throughout the 1980s and 1990s, the exception being in 1997-98, when cold Labrador Slope Water moved into the region along the shelf break and flooded the lower layers of the central and south western regions. While this event produced the coldest near-bottom conditions in these shelf regions since the 1960s, its duration was short, lasting about one year. 

Juvenile crab ($\sim$~instar 5, or 2 years since settlement) were already present in high numbers in the transitional year of 1986. These crab were therefore the benefactors of environmental amelioration and some other cause had allowed their larval and adolescent numbers to build up to very large level prior to these environmental changes. What this may be is not yet fully understood. Further, it is important to note that bottom temperatures have been increasing consistently since the early 1990s while snow crab continues to dominate the bottom environment, substantially weakening this hypothesis.
\end{enumerate}


\subsection{Geographic sources of snow crab larvae}

The potential for hydrodynamic transport of snow crab larvae from the southern Gulf of St. Lawrence to the Scotian Shelf has been studied by J. Chass\'{e} (Ocean Sciences Division, BIO, DFO; pers. comm.). By treating larvae as passive particles, he was able to demonstrate that a large number of larvae can be transported onto the Scotian Shelf and deposited in areas of strong eddies (near the Sable Bank area and in the shallow areas further west). While pelagic organisms can maintain their position in a single location in even very strong advective conditions via control of vertical migrations, the possibility of snow crab larvae entering the Scotian Shelf from the Gulf of St. Lawrence region cannot be ignored. While the biological importance of this downstream flow has not been quantified nor studied empirically, the following observations indicate that the Scotian Shelf population is acting as an autonomously reproducing system:

\begin{itemize}
  \item The temporal dynamics of the Scotian Shelf snow crab population is generally out-of-phase with the cycles seen so far in the Southern Gulf of St. Lawrence. If the Scotian Shelf were completely dependent upon the larval drift from the Gulf region, the temporal dynamics of the populations would be in-phase. 
  \item The spatial distribution of the Brachyuran larvae on the Scotian obtained from the Ichthyoplankton Sampling program in the 1980s (see summary in \citealt[page 14]{Choi2005resdoc}) have been observed to be quite pervasive throughout the Scotian Shelf with no spatial clines (i.e., no declines in abundance with distance from the Gulf of St. Lawrence area) as one might expect if the source of larvae were solely from the Gulf region.
  \item Planktonic samples from the Scotian Shelf obtained from the Continuous Plankton Monitoring Program (see summary in \citealt[page 14]{Choi2005resdoc}) indicate that the number of Brachyuran larvae have been consistently low, even with the assumed constant input from upstream sources, from 1976 to 1996 (with a small pulse being detected in 1980). 
  \item A clear pulse of larval abundance was observed from 1997 to 1999 with peak levels in 1998 (\citealt[page 14]{Choi2005resdoc}). The timing of this pulse is concordant with the growth schedules of the currently expected recruitment into the system. Approximately 9 years would be required to grow from the zoea stages to instar 11/12, the stages in which snow crab begin to moult to maturity in 2007, the same time difference between 1998 and 2007). 
  \item The period in the late 1990s when high larval production was observed was precisely the same period in which the abundance of mature males and females on the Scotian Shelf were at their peak (\citealt[page 14]{Choi2005resdoc}). 
\end{itemize}

The above circumstancial evidence suggests that the snow crab resident on the Scotian Shelf is more than capable of being a self-reproducing system, regardless of inputs from other systems. Further, the possibility of some external sources of larvae does not mean that we can ignore the reproductive potential of the snow crab resident on the Scotian Shelf proper. To this end, the snow crab industry adopted a precautionary approach to the conservation of large mature males (i.e., reduced exploitation rates) to allow them to mate with the more rapidly maturing females in 2006/7.




% \begin{figure}[ht]
% \begin{center}
% \rotatebox{0}{\scalebox{0.8}{\includegraphics*{R/larvae/Brachyura.png}}}
% \end{center}
% \caption{Temporal variations in the numerical density of Brachyura larvae in the eastern Scotian Shelf from various plankton monitoring systems $(number / m^3)$. Note the sharp pulse of larvae in the late 1990s, centered on 1998. This distinct pulse of larval release should become fishable in approximately 9 years. It is also indicative that the high abundance of reproductive crabs is indeed a recent phenomenon. Further, $\approx$ 9 years previous to this 1998 pulse (i.e, 1989) was when this dominant group of crab had their origins -- the same period as the crash of the groundfish on the eastern Scotian Shelf. 1 standard error bars are presented. Data courtesy of B. Petrie (Ocean Sciences Division, BIO, DFO).}
% \label{ts.larvae}
% \end{figure}



%\begin{figure}[ht]
%\begin{center}
%\rotatebox{0}{\scalebox{0.5}{\includegraphics*{R/larvae/annual/n.m3/n\D m3\D 2000.png}}}
%\end{center}
%\caption{Map of the numerical density of Brachyura larvae in the eastern Scotian Shelf from various plankton monitoring systems $log_{10}(number / m^3)$ spanning several decades of data. Data courtesy of B. Petrie (Ocean Sciences Division, BIO, DFO).}
%\label{map.larvae}
%\end{figure}

\subsection{Oil and gas exploration and development}

The interests of the oil and gas industry to explore and develop areas on the Scotian Shelf near to, or upstream of major crab fishing grounds and population centers (both N- and S-ENS) has been identified by numerous fishermen as a source of concern. The potential effects of seismic methods of exploration upon vulnerable components of the snow crab population and the uncertainties associated with the long-term biological effects of development need to be made explicit and include the following issues:

\begin{itemize}

\item A pulse of females is beginning to mature and become reproductive. This will continue for another 3 years minimum. They hold eggs for two years. That is, the snow crab are entering an important reproductive phase that will last at least 5 years into the future. The snow crab is behaviourally very complex, especially as they pertain to reproductive behaviour. Disruption of their mating rituals is particularly likely as the courting/mating period can last up to several weeks. This can modify the reproductive/regenerative capacity of the snow crab resident on the Scotian Shelf. Any damage to eggs and reproductive behaviour can critically impact the future of the fishery.

\item A pulse of males is recruiting into the fishery (2 years) after many years of a decline. Their soft-shell phase will become proportionately more important. Soft-shelled crab are particularly sensitive to physical trauma.

\item Many immature male and female are found in shallower waters. They are currently increasing rapidly in number. Damage to this component will have short- and long-term repercussions to the fishery.

\item No information is available for the effects of seismic pressure waves upon the planktonic forms of snow crab.

\item Being a very long-lived species, the snow crab is exposed to environmental hazards for up to 13 to 16 years (since egg extrusion). As such, simple short-term studies (e.g., of exposure to strong pressure
impulses and associated direct and indirect phenomena) do not describe the as yet unkown effect of long-term, compounded (cumulative) effects of oil and gas exploration and development upon snow crab. This is a very large uncertainty.

\item Snow crab are known to jettison legs or die when physically shocked (dropped onto the deck of a boat). This is an important unknown as pressure waves can be amplified when moving through media of differing densities.

\item Snow crab are important benthic predators. Bioaccumulation of heavy metals and toxic organic chemicals released from oil and gas development are possible, especially as they are so very long-lived. Any damage to the quality of snow crab can be highly detrimental to the reproductive capacity of the population and also taint the reputation of the quality of the Scotian Shelf snow crab.


\end{itemize}

Substantial sacrifices were made in the 2004 to 2006 seasons by snow crab fishermen to reduce any risks of damaging the reproductive potential of Scotian Shelf snow crab. The numerous uncertainties associated with oil and gas exploration/development increases the risk of destabilising the snow crab population on the Scotian Shelf.


\section{Methods}

This assessment uses a fisheries-independent dedicated snow crab trawl survey as the primary source of information about the current status of snow crab. The analytical approaches used to assess this snow crab population have developed to accomodate for the high interannual variability in the spatial distribution of this population of snow crab that exists on the southern-most extreme of their distributional range.

\subsection{General analytical tools}

All analytical tools were open-sourced software. Maps for visualisations were constructed with GMT (Generic Mapping Tools, \citealt[version 4.1]{wessel1998}). For results that were not kriged (due to biological categories with highly heterogeneous distributions), the \textit{splines-in-tension} data smoothing and interpolation technique was used with the tension parameter T set at 0.4 and a spatial extent of interpolation of 20~km radius from every datum, a range comparable with that observed in the empirical variograms (see below). Conversions between cartographic and cartesian co-ordinate systems for analytical purposes were computed with PROJ \citep[version~4.4.9]{evenden1995}, using a Lambert conformal conic projection and the WGS84 spheroid with the central meridian at 63$^\circ$W and standard parallels at 43$^\circ$N and 47$^\circ$N. All other data analyses were implemented in the statistical computing language and environment R (\citealt[version~2.4.1]{R2007}; \citealt{venebles2002}) to allow migration and documentation of methods into the future. Kriging was conducted with the R package, GSTAT \citep[version~0.9-35]{pebesma2007}. The complete analytical suite, coded in R, has now been completed and posted to \begin{small} \url{http://autocatalysis.googlepages.com/snowcrabanalysis} \end{small}. 


\subsection{Fisheries data}

Catch rates are biased indicators of abundance as both the spatial and temporal distribution of crabs and the fishing effort are not uniform, varying strongly with season, bottom temperatures (potentially causing crowding into core areas), food availability, timing of spring plankton blooms, reproductive behaviour and substrate/shelter availability, relative occurence of soft and immature crab and associated discards, fisher experience, bait type and soak time and ambient currents. These numerous but important factors have not been modelled, rendering the interpretation of catch rates as an index of abundance an uninformative tool. They are presented here only to maintain continuity with historical records. 

Mandatory logbooks provide information on location, effort (number of trap hauls) and landings (verified by dockside monitoring). The data are stored in the MARFIS database (Maritimes Region, Policy and Economics Branch, Commercial Data Division). Exhaustive data quality checks were completed. Catch rates (kg~trap$^{-1}$) were computed as the ratio of total landings to total trap hauls, using data from properly completed logbook records. Catch rates are reported without standardisation. Total trap hauls for a given group is computed by dividing the total landings by their respective catch rate.

At-sea-observer data provide information about the size structure and the carapace condition of the commercially exploited stock. The data are stored in a centrally organised relational database (Observer Database System). At-sea-observers are deployed randomly (Figure~\ref{observer.locations}) with the coverage being as evenly distributed between areas as possible. The target coverage (by quota) was 5\% and 10\% for N- and S-ENS, respectively. This information was also used to compute the potential by-catch of other non-snow crab species by the snow crab fishery. By-catch estimates of each species $i$, was extrapolated from the biomass of species $i$ observed in the catch and the relative observer coverage (by biomass) as follows:

\begin{equation}
  By \: catch_{i} \: [kg] = \dfrac{ Observed \: catch_{i} \: [kg] } { Observer \: coverage \: [proportion \: by \: biomass]} .
\end{equation}


\subsection{Research survey data}

Assessments are based upon a survey dedicated to snow crab assessment on the Scotian Shelf, introduced by the GFC in 1996 \citep{Biron1997} and funded by the snow crab industry through a Joint Project Agreement. Spatial coverage in the survey is (1) \textbf{extensive}, going well beyond all known commercial fishing grounds and (2) \textbf{intensive}, with a minimum of one survey station located pseudo-randomly in every 10~\mult~10 minute area (Figure~\ref{trawl.survey.locations}). This sampling design was developed to facilitate geostatistical estimation techniques (i.e., \textit{kriging}; \citealt{Cressie1993, legendre1998, Kern2000}). Since 2004, over 350 stations have been sampled annually on the same fishing vessel, The Gentle Lady (a 65 foot dragger) with the same captain. In the 2006 survey, 374 stations were sampled, 15 less than in 2005 due to expected funding uncertainties associted with the Larocque Decision, low TACs and low prices.

The extensiveness of the sampling design allows the objective determination of the spatial bounds of the snow crab population, information that must be known if reliable estimates of biomass and population structure (e.g., size, sex, maturity) are to be made. The spatial distribution of snow crab is quite dynamic and so can rapidly shift to areas where they are not ``traditionally'' found. For the purposes of monitoring such changes in spatial distribution, sampling is required even in areas where crab have not been previously observed. In addition, the distributional patterns of immature, soft-shelled, very old and female crabs do not correspond to those of legal size males. The former are considered to be less competitive and more susceptible to predation \citep{Hooper1986} and usually observed in more marginal environments or substrates with greater cover (gravel, rocks; \citealt{Comeau1998}). Focusing upon only those areas where large hard-shelled males occur in high frequency would preclude the reliable estimation of the relative abundance of these other important segments of the crab population.

With the evolution of the areal extent and alterations in the intensity and timing of surveys since the mid-1990s, direct inter-annual comparisons of the data are made difficult. The survey is conducted in the autumn (September to November; i.e. post-fishing season in ENS and just prior to the fishing season in CFA~4X). The timing of the surveys have stabilised to this latter period only since 2002. Prior to 2002, surveys were conducted during the spring (April to July; i.e., pre-fishing season in ENS). As a consequence, temporal trends are most reliable for the post-2002 period. In the southern-most area of snow crab distribution (CFA 4X) trawl survey coverage has been historically sporadic. A renewed effort has been made to stabilise the spatial coverage in this area since 2004.

A \textit{Bigouden Nephrops} trawl, a net originally designed to dig into soft sediments for the capture of lobsters in Europe was used to sample the substrate (headline of 20~m, 27.3~m foot rope mounted with a 3.2~m long 8~mm chain, with a mesh size of 80~mm in the wings and 60~mm in the belly and 40~mm in the cod-end). Tows were conducted for $\sim$~5~minutes in duration with actual duration of bottom contact being monitored by Netmind sensors. The width of the mouth of the net was also monitored with Netmind sensors. The ship speed was maintained at $\sim$~2~knots. The warp length was $\sim$~3~\mult the depth. Positional information as well as water temperature measurements were collected using a global positioning system and Minilog data recorders, respectively. The surface area swept by the net was calculated from swept distance and net width information.

All crab were enumerated, measured with calipers, shell condition described (see Table~\ref{table.carapace.condition.description}), claw hardness determined with a durometer and weighed with motion-compensated scales. The latter allowed direct biomass measurement rather than estimates relying upon allometric relationships between body parts (the approach in 2003 and earlier; see below). Data entry and quality control was provided by JaviTech and migrated onto the Observer Database System, held at DFO, BIO (Bedford Institute of Oceanography, Dartmouth, Nova Scotia).

Pre-2004 snow crab biomass estimates were approximated from carapace width (CW) measurements by applying the allometric relationships developed for the Scotian Shelf adult hard shelled snow crab (\citealt{Biron1999}; R$^{2}$ $=$ 0.98, n $=$ 750):

\begin{equation}
  mass [g] = 1.543 \times 10^{-4} \times {(CW [mm])}^{3.206}
\end{equation}

The maturity status of males was determined from morphometric maturity. While physiological maturity is not directly co-incident with the onset of morphometric maturity (morphometrically immature male crabs are more than capable of mating in the absence of competition from terminally moulted males;  \citealt{Sainte-Marie1993}), the latter is more readily quantified. In the terminal moult of male snow crab, a disproportionate increase of chela height (CH) relative to carapace width (CW) is generally observed (a factor which may be associated with increased mating and/or reproductive success). Such morphometrically mature males can be discriminated from those that have not undergone the rapid chela growth via the following equation (E. Wade, personal communication, GFC):

\begin{equation}
  M_{male} = -25.324 \times \ln(CW [mm]) + 19.776 \times \ln(CH [mm]) + 56.650
\end{equation}

where the individual is considered mature if $M_{male} > 0$.

The maturity status of females is assessed from direct visual inspection of eggs or gonad development. Where maturity status was ambiguous, maturity was determined morphometrically, as the width of abdomen (measured by the width of the fifth abdominal segment, AW) increases rapidly relative to carapace width (CW) at the onset of morphometric maturity, facilitating the brooding of eggs. This onset of morphometric maturity can be delineated via the following equation (E. Wade, personal communication, GFC):

\begin{equation}
  M_{female} = -16.423 \times \ln(CW [mm]) + 14.756 \times \ln(AW[mm]) + 14.900
\end{equation}

where the individual is considered mature if $M_{female} > 0 $.

Sex ratios were calculated from kriged numerical abundance estimates $N$ as:


\begin{eqnarray}
  & Sex\:ratio_{(immature)} & = \dfrac{ N_{(female,\: immature)} }{N_{(female,\: immature)} + N_{(male, \: immature)}} \\
  & Sex\:ratio_{(mature)} & = \dfrac{ N_{(female,\: mature)} }{N_{(female,\: mature)} + N_{ (male,\: mature) }} 
\end{eqnarray}


Size-frequency histograms were constructed (expressed as number per unit area swept by the trawl (km$^{-2}$); i.e., the arithmetic mean numerical density per unit area in each size interval). Modes and the bounds of the each modal group were identified from size frequency distributions. Each instar was determined after an analysis of size-frequency distributions to have a lower bound of carapace width (mm) approximated by (see also Figure~\ref{growth.curves}):

\begin{eqnarray}
  CW_{(lower, \: male)}\:[mm]   & = \exp(1.918 + 0.299 \times (Instar - 3) ) \\ 
  CW_{(lower, \: female)}\:[mm] & = \exp(2.199 + 0.315 \times (Instar - 4) )
\end{eqnarray}


%    mb = function(i) exp(1.917564 + 0.298914*i)
%    # mb = male bound in cw where i=1 when cw = ~ 9.2 mm,
%    # instar 5 is bounded by mb(2) and mb(3)
%    # prior to instar 5 moulting can be variable and so should not be
%    # relied upon to heavily
%    #  see "model.size.structure.r" for details of derivation

%   fb = function(i) exp(2.198848 + 0.315026*i)
%    # fb = female bound in cw where i=1 when cw = ~ 12.5 mm,
%    # instar 5 is bounded by fb(1) and fb(2)
%    # prior to instar 5 moulting can be variable and so should not be
%    # relied upon to heavily
%    #  see "model.size.structure.r" for details of derivation


The locations of ``\textit{most likely snow crab habitats}'' were determined on high resolution depth maps (15 arc-second resolution) obtained from the Canadian Hydrographic Service. The range of observed depths of Scotian Shelf snow crab (between 60~m to 280~m; Figures~\ref{habitattemplate.depth}, \ref{habitattemplate.cpue.depth}) were used to delineate these potential snow crab grounds. For each year, maps of preferred snow crab temperature locations were made using Optimally Estimated bottom temperature maps (obtained from the Ocean Science Division of BIO, DFO). Preferred snow crab temperatures on the Scotian Shelf ranged between $-1 \celsius$ {} to $+6 ~\celsius$ {} (Figure~\ref{habitattemplate.temperatures}). These two sources of information were combined to delineate the most probable locations of snow crab habitat (Figure~\ref{fig:method}). This dynamically changing habitat range was used to predict the biomass and numerical densities of crab. 

The ranges chosen were expansive to allow abundance estimates of both the females, immatures, and CC5 crab as well as the fishable biomass. Due to the choice of these more expansive ranges, it is likely that fishable biomass is slightly over-estimated while the numerical abundance of females, immatures and CC5 crab are likely underestimated due to the greater abundance of non-commericially exploited crabs in warmer and shallower environments.  These biases will be corrected in the near future. 

Further, deviating slightly from the 2005 assessment protocol, additional areas were included as potential snow crab habitat even if they were warmer than $+6 ~\celsius$ {} to a spatial extent of 5~km radius from a survey station. This modification was implemented as on occasion, crab (mostly immature and female crab) were observed in warm waters (e.g., CFA 4X and continental slope waters). This result may have been due to rapid changes in bottom temperatures trapping crab into pockets of high local density and/or adaptation (acclimation) to the warmer temperatures of some areas and the overall very warm bottom conditions in 2006. This issue will be studied further in the future. It should be emphasized that this was mainly an issue with immature and female snow crab in S-ENS and so will likely contribute to their under-estimation. Conversely, this addition of the extra surface area may also contribute to the over-estimation of fishable biomass in S-ENS.

The estimation of biomass and numerical densities involved the use of geostatistical methods: modelling of variograms (the behaviour of variance as a function of distance) for each of the individual variables, in each year. The variogram, $2 \hat{ \gamma}(\cdot) $ or alternately the quantity, $\hat{ \gamma} (\cdot)$ known as the semi-variance or semi-variogram, is classically determined by the method of moments \citep{Cressie1993} for some random process, $Z(\cdot)$, such as biomass or number of crab. In the simple case of an isotropic random process, $Z(\cdot)  \approx Z( \bigtriangleup x )$, where $\bigtriangleup x = x_i - x_j$, the distance between all pairwise sampling locations or positions $x$. For such an isotropic processes, the method of moments estimator of the variogram is:

\begin{eqnarray}
  & 2 \hat{ \gamma} (\bigtriangleup x) & = Var( Z (x_i) - Z(x_j) )  \\
  &   & = \dfrac{1} {| N_{\bigtriangleup x} |}  {\displaystyle \sum_{N_{\bigtriangleup x}}} [ Z(x_i+\bigtriangleup x) - Z(x_i) ]^2
\end{eqnarray}


where $Var$ is the variance, $N_{\bigtriangleup x}$ is the number of pairwise cases. 

An alternative and more robust formulation used in this assessment is Cressie's \emph{modulus} estimator \citep{Cressie1993}:

\begin{equation}
  2 \hat{ \gamma} (\bigtriangleup x) =  \dfrac { \left [ \dfrac{1} {N_{\bigtriangleup x}} {\displaystyle \sum_{N_{\bigtriangleup x}} } | Z(x_{i+{\bigtriangleup x}}) - Z(x_i)| ^{1/2} \right ] ^4 } { 0.457 + 0.494/ | N_{\bigtriangleup x} | }
\end{equation}

Variograms were constructed using an automated process that chose a functional form (bessel, spherical, exponential, circular, gaussian or generalised Matern) that best described the spatial variance using a nonlinear least-squares criterion (an example is provided in Figure \ref{fig:variogram}). When solutions did not converge, a moving time averaged empirical variograms (scaled to unit variance) of varying window sizes were constructed until a reliable solution was obtained. A three-year window was generally sufficient for stabilisation. The final solution was re-scaled to the local variance of the region and year of interest.

Spatially explicit means and variances were then obtained using Universal Kriging with External Drift which were used for visualisations. External Drift is a technique that linearly accounts for variations in external parameters. The most significant such variables were determined from bi-variate analyses to be: depth, bottom temperature, total fishery landings since 1998 and fishery catch rates, all discretised to 2~\mult~2~\kmsq. Where the relationship to abundance measures were nonlinear, the variables were transformed to a linear relationship. Due to the extreme irregularity of the potential snow crab habitat space, Ordinary Block Kriging, using blocks of 1~\kmsq {} resolution was used to estimate 95\% confidence bounds of the predicted intensive variables (biomass and numerical densities). When problems of solution convergence were observed (due to sparse spatial coverage as with some classes such as immature, female, carapace condition 1 and 5 crabs), categories were agglomerated when appropriate. The numerical abundance of each of the nominal growth stanzas (Figure~\ref{male.growth.stanzas}) were also determined via kriging.

Relative exploitation rates ($ER$) at time $t$ is calculated as:

\begin{equation}
  ER_{t} = \dfrac { Landings_{t} } {  Landings_{t} + Mature \: fishable \: biomass_{t}  }
\end{equation}

where $t$ is time; $Landings_{t}$ is the total landed snow crab in year $t$; and $Mature \: fishable \: biomass_{t}$ is the total mature and legally fishable biomass (mature male snow crab $\geq$~95~mm~CW) estimated from kriging for year $t$. This definition is used as there is agreement to focus exploition upon mature individuals and to return immature crab (pencil-clawed) to the water. 

Relative numerical exploitation rates of each growth stanza was also estimated from at-sea-observed catches for each major area with the assumption of 100\% catchability:

\begin{equation}
  ER_{t,i} = \dfrac{ Number \: landed_{t,i} } {  Number \: landed_{t,i} + Number \: surveyed_{t,i}  }
\end{equation}

where $t$ is time; $i$ is growth stanza; $Number \: landed_{t,i}$ are the total number of snow crab estimated to have been landed from at-sea-observed proportions of each growth stanza $i$ in the catches of year $t$; and $Number \: surveyed_{t,i}$ is the total number of snow crab estimated from kriged numerical abundance of each growth stanza $i$, in year $t$.

Markov-type transition matrices (Tables~\ref{tm.nens}, \ref{tm.sens}, \ref{tm.4x}) were determined for each nominal growth stanza of male snow crab based upon historical data from 2003 to the present. Data prior to 2003 could not be used to compute the transition matrix due to the very different timing of the surveys (spring) and differences in the spatial extent of the research surveys. Due to gear and sampling bias and the bi-annual moulting of snow crab instars 1 to 5, numerical abundance and transition matrix estimates were limited to instars 6 and greater. No information on reproduction and early pelagic and benthic survival were assumed and so these transition matrices are referred to as \emph{pseudo-transition matrices}. Further, the relative differences in catchability of the various size and maturity classes were not separated from survivorship resulting in transfer functions that can be greater than 1. The catchability of the commercially exploitable population was assumed to be 100\%. These pseudo-transition matrices were developed for each major region separately (N-ENS, S-ENS, CFA 4X) where-ever possible and used for forward projection under varying scenarios of exploitation rates.

Forward projection scenarios were derived from fishing patterns in the most recent year of at-sea-observer estimates of relative exploitation for each of the above growth stanzas and the most recent year of abundance estimates from trawl surveys. Errors ($\Delta x$) from all potential sources were propagated assuming all $n$ variables ($x_n$) were independent of each other:


\begin{eqnarray}
 &  z & = f(x_1, x_2, \ldots, x_n) \\
 & {(\Delta z)}^2 & = {(\dfrac{ \partial f }{\partial x_1 } \Delta x_1)}^2  + 
                   {(\dfrac{ \partial f }{\partial x_2 } \Delta x_2)}^2  +
                  \ldots +
                   {(\dfrac{ \partial f }{\partial x_n } \Delta x_n)}^2
\end{eqnarray}

The estimation of recruitment and mortality (flow variables or input-output variables) requires the use of a model to integrate all aspects of the population structure and dynamics. A Virtual Population Analysis (VPA) was recommended as a possible mechanism in a 2006 review. However, upon consideration of the approach, it was noted as being inappropriate as only one ``age-class / size -class'' is being exploited. There would be no ``age-structure'' in the commerical data, rendering a VPA unsolvable (singular). A spatially-explicit individual-based simulation model is being explored as a possible alternative but the development cycle of such an approach requires time and effort. For the purposes of the 2006 assessment, the above Markov-type model will be used.



\section{Environmental conditions}

The spatial extent of what may be considered potential snow crab habitat based upon bottom temperature and depth, has been very stable in N-ENS, at about 7.1~\Xthou~\kmsq{} (Figures~\ref{maps.bottom.temperature}, \ref{ts.habitat.surfacearea}). For S-ENS, the surface area of potential habitat has been much more variable, ranging from between 40 to 70~\Xthou~\kmsq{} over the past three decades. In the most recent period, the surface area has declined to near normal levels since 2005 (Figure~\ref{ts.habitat.surfacearea}, near 50~\Xthou~\kmsq). Such sharp decreases as seen in S-ENS between 1998 to 2000 and from 2004 to 2005 may increase the crowding of the snow crab and therefore also their catchability as they are concentrated into stronger aggregations in core areas. In CFA 4X, the southern-most limit of the distribution of snow crab, potential habitat has been most variable: ranging from near 0 to 25~\Xthou~\kmsq; since 2004, it has declined from 22 to less than 5~\Xthou~\kmsq in 2006.

Within each area that is potentially snow crab habitat, below average temperatures were observed throughout most of the 1990s (Figures~\ref{maps.bottom.temperature}, \ref{ts.habitat.surfacearea.temperatures}). However, an overall warming trend has been evident over the last four years and more generally since the early 1990s. This warming trend may disrupt moult cycles and  result in the observation of soft-shell crabs at earlier times in the fishery. Note also that in CFA 4X, the average bottom temperatures are extremely erratic and high relative to ENS. In such areas of environmental variability, a more precautionary approach to exploitation is warranted.


\section{Fishery performance}

\subsection{Effort}

The spatial distribution of fishing effort continued the trend of increasing effort in offshore areas and declines in inshore areas (Figure~\ref{maps.effort}). In N-ENS, total effort in the Glace Bay Hole area increased. In S-ENS, the majority of the fishing effort was observed north of Sable Island, southeast of French Bank, northeast of Middle Bank, and south of Misaine Bank (between Misaine and Banquereau Banks). Relative to 2005, the spatial distribution of fishing effort was more constrained to a smaller core offshore area in S-ENS. This shift may have in part been encouraged by increased trap limits in CFA 23 (60 traps in the offshore vs. 45 traps in the inshore). In CFA 4X, the majority of fishing effort continued to be centered south of Sambro.

In 2006, a calculated total of $\sim$~13,700 and 49,400 traps hauls were applied in N-ENS and S-ENS, respectively. Relative to 2005, this represents a decline of 26 and 15\%, respectively, due in part to reduced TACs in all areas.

\subsection{Landings}

The total landings were 486~t in N-ENS and 4,486~t in S-ENS (Tables~\ref{tab:fisheries.summary.north}, \ref{tab:fisheries.summary.south}), representing declines of 14 and 30\% relative to 2005, respectively in each area (Figure~\ref{ts.landings}). These declines in landings corresponded to TAC reductions of 14 and 29\%, respectively in N- and S-ENS. The geographical distribution of landings mirrored the geographical distribution of fishing effort  (Figure~\ref{maps.landings}).


\subsection{Catch rates}

In N-ENS, the 2006 catch rates\footnote{Please see the caveats about catch rates being inappropriate indicators of fishable biomass, in the Methods section.} were 35.6 kg/trap, a 16\% increase from 30.6 kg/trap in 2005. This is mainly due to the movement of fishing effort away from areas of lower catch rates (i.e., Inner Areas of N-ENS) to higher catch rates, namely, the Glace Bay Hole (Figure~\ref{maps.cpue}). Though management areas were combined in 2005, most fishers fished within their respective (traditional) boundaries in 2005. This resulted in areas of high depletion and low catch rates. In 2006, the fishing effort was less spatially contrained. However, N-ENS catch rates are still well below the 10 year mean (58.7 kg/trap; Figure~\ref{ts.cpue}; Table~\ref{tab:fisheries.summary.north}). 

In S-ENS, the 2006 catch rates were 90.6 kg/trap, a 17\% decline from 109.4 kg/trap in 2005. These are the lowest catch rates in the five year record (Figure~\ref{ts.cpue}; Table~\ref{tab:fisheries.summary.south}). The spatial distribution of catch rates were highest in the offshore areas and very low in most inshore areas (Figure~\ref{maps.cpue}). Peak levels were found towards the Misaine Bank and Sable Island areas of S-ENS. 


\subsection{At-sea-observer coverage}

In N-ENS, the at-sea-observer coverage level fell just short of the target level of 5\% of the TAC, at 4.2\% (Figure~\ref{observer.locations}). A total of 113 traps were sampled (0.82\% of estimated commercial trap hauls). Of a total of 4,039 male crab sampled by at-sea-observers, 3,325 were of legal commercial size. 

In S-ENS, 9.8\% of the TAC was observed (with a target level of 10\%). A total of 942 traps (1.9\% of estimated commercial trap hauls) were sampled. Of the 31,161 male crab sampled from these traps, 28,645 were of legal commercial size.

Discard rates of undersized snow crab were therefore 17.7\% and 8\% in N-ENS and S-ENS, respectively. These discard rates are an improvement relative to 2005 levels of 34\% and 21\% in  N-ENS and S-ENS, respectively as they result in reduced handling mortality. True discard rates were likely higher as immature individual were also discarded in 2006.


\subsection{By-catch}

By-catch levels in the snow crab fishery are low, being mostly dominated by other crabs. Over the past three years, by-catch levels were $\sim$~0.025\% of the total landed biomass (an average of 1.5 t/yr of by-catch relative to an average of 7,100 t/yr of landed snow crab; see Table~\ref{table.bycatch}). This low incidence of by-catch is attributable to trap design (top entry conical traps), the large mesh size (5.25 inches, knot to knot) and the passive nature of the gear. Further, in the past three years, observers have reported three leatherback turtles entangled in buoy lines. However, all turtles were released with minimal or no damage to the turtles. 

The by-catch of snow crab in other fisheries remains an area requiring attention. The spatial distribution of Northern shrimp overlaps with that of snow crab and so represents an industry that requires particular attention. The inshore lobster fishery may also represent an important source of juvenile snow crab mortality in some areas due to their capture in lobster traps and illegal usage as potential bait.

\subsection{Carapace condition} 

The entry of the new recruits that were expected for 2006 caused some clear changes in the relative composition of moult stages of at-sea-monitored snow crab (Figure~\ref{observer.carapacecondition}). For snow crab smaller than legal size (\lt~95~mm~CW), CC2 crab increased in their relative numbers in both N-and S-ENS (Tables~\ref{table.CC.small.north.obs}, \ref{table.CC.small.south.obs}). CC3 males dominated the sub-legal crabs. The relative proportion of CC5 snow crab increased marginally in N-ENS, while in S-ENS, CC5 males remained comparable to 2005 levels. 

For legal-sized crab ($\geq$~95~mm~CW), similar to sub-legal males, an increase in the relative abundance of CC2 crab and a decrease in CC4 crab was observed in 2006. CC3 crab were also the dominant crab in both N-ENS and S-ENS (71.4\% and 69.3\%, respectively; Tables \ref{table.CC.large.north.obs} and \ref{table.CC.large.south.obs}). CC5 crab in the legal-sized fraction increased slightly in N-ENS while it declined in S-ENS. 

Trawl survey abundance estimates of the legal sized fraction of male snow crab (Tables \ref{table.CC.north.trawl}, \ref{table.CC.south.trawl}) have also been historically dominated by the CC3 category. However, in 2006, a strong pulse of CC2 crab was evident, the newly entered recruits into the system (19.73\% in N-ENS and 18.64\% in S-ENS). The relative abundance of CC1 males was below the detection limit of the trawl surveys, due in the main to the lateness of the survey. Similarly, CC5 males were below the detection limit of the trawl survey. 


\subsection{Soft-shelled snow crab}

For N-ENS, the occurrence of soft-shell crab in 2006 was 4.1\% by number, lower than the 18\% observed in 2005 (Table \ref{table.CC.large.north.obs}; Figure~\ref{map.softshell}). Soft shell catches in N-ENS were primarily observed in the Inner areas where most of the effort occured. High soft shell incidence did not occur solely in low catch areas; indeed, high soft-shell catch rates were also found in areas with high catch rates (Figure \ref{softshell.cpue}). 

In S-ENS, the incidence of soft-shelled crab catches was higher in 2006 than 2005 (Table~\ref{table.CC.large.north.obs}). In S-ENS, 6.2\% were observed to be soft-shelled crab for the 2006 season, relative 5.2 \% in 2005 and were distributed throughout the fishing grounds. High soft shell catches ($>$20\% by number) were observed throughout the fishing grounds with the exception of the area east of French Bank. 

Assuming that released soft-shelled snow crab were not recaptured, the soft-shell captures may represent up to an additional 20~t being discarded as soft crab in N-ENS and 278~t in S-ENS for a total of up to 298~t for all of ENS. This is a problem as elevated mortality is associated with handling of soft-shelled crab. The actual mortality may be less, depending upon the condition of the crab upon return to the water, however this handling of soft crab can have a significant and direct deleterious effect upon the fishable biomass of snow crab. The willingness of fishers to cross the former area and subarea lines in N-ENS in 2006 likely contributed to this decrease in soft shell catches: fishers were able to shift effort away from locally depleted areas where a high incidence of soft crab becomes more likely.


\section{Resource Status}

\subsection{State variables}

\subsubsection{Size structure}

The size frequency distributions of males in N-ENS (Figure~\ref{hist.male.cw}) show a pulse of immature male crab detected in 2003 and 2004 that continue to grow and propagate through the system. The main pulse of immature crab is currently centered over the 55~mm modal group (instar 9/10). The leading edge of the potential recruits were near 80~mm~CW (instars 10/11). A fraction of these instar 10/11 crab will moult and enter fishable size in the 2007 season; however, they will mostly be still soft-shelled or white crab. There is therefore a continued likelihood that soft-shell incidence may be an issue in the 2007 season. Unfortunately, it is only in 2008 that any recruitment will become evident in N-ENS. The main pulse currently centered over the 55~mm~CW mode will begin entering the fishery in a significant fashion in the 2010/2011 fishing seasons. These expectations are contingent upon no significant increases in natural mortality of crab (e.g., predation, competition, temperature conditions, etc.). The number of mature undersized males has also increased in 2006 due to a fraction of the abundant immature crabs moulting to maturity. It is unlikely that this is due to genetic selection for earlier maturing individuals as the selection pressure upon the snow crab has only been expressed for 2 generations at most. However, the early maturation of organisms that are heavily exploited remains an issue that must not be ignored. An initial snapshot of spatial varaitions in the size at 50\% maturity is provided in Figure~\ref{map.maturity}.

In S-ENS, similar to N-ENS, the main pulse was also centered over the 55~mm~CW modal group (instar 9/10). However, unlike the N-ENS, immature crab were also observed spanning all size ranges from 40 to 110~mm~CW in 2005 and 2006. This is a positive sign for the S-ENS, in that recruitment to the fishery is going to be stabilised (and increasing) for at least the next four years. Based upon established growth patterns, the main peak over the 55~mm~CW mode should enter the fishable biomass by the 2010/2011 fishing seasons. Size frequency distributions in CFA 4X exist in a state intermediate between N-ENS and S-ENS and similar time trends may be expected.

The size frequency distributions of female snow crab in 2006 clearly indicate that the pulses of immature females detected in 2003 for N-ENS and in 2004 for S-ENS are continuing to grow and intensify (Figure~\ref{hist.female.cw}). A large fraction of the females are now sexually mature and reproductive. This trend should continue for another 2 to 3 years. A similarly important increase in the number of mature females have been observed in S-ENS and CFA 4X for the first time since the late 1990s. 


\subsubsection{Sex ratios}

When sex ratios of mature crab are too high (i.e., too many females relative to males), the possibility of sperm limitation becomes a conservation issue. This is particularly the case in the absence of large mature males to mate with and protect the more rapidly maturing and smaller females. Conversely with very low sex ratios (e.g., the extended period of very low sex ratios observed in the early-2000s throughout the Scotian Shelf) there is very poor egg and larval production (Figure~\ref{ts.sex.ratio.mature}, \ref{map.sex.ratio.mature}). As female snow crab of a given year-class will mature 2 to 3 years earlier than male snow crab from the same year-class, and because the females have a shorter life span, there is a high likelihood that sex ratios will fluctuate over time. This is particularly the case when strong year classes dominate a population.

The sex ratios of immature snow crab are currently near 40\% and 28\% females in N- and S-ENS, respectively. They represent a slight decreasing trend as the females of the current wave of immature snow crab (40 to 65~mm~CW) begin to mature approximately 2 years in advance of the males from the same year classes (Figure~\ref{ts.sex.ratio.immature}, \ref{map.sex.ratio.immature}). The spatial patterns of the sex ratios are distinct between offshore and inshore areas: immature males are found in greater proportion in offshore whereas immature females are found in greater proportion towards the inshore areas (Figure~\ref{map.sex.ratio.immature}). This spatial segregation likely exposes the crab to differential predation effects. Inshore females are likely more intensely predated upon by inshore fish, other macro-invertebrates (including other female snow crab, other crabs and lobster -- immature snow crab have been reportedly caught in large numbers in lobster traps; \citealt{Sainte-Marie2002, Squires2003}). This pattern would be exacerbated by the sexual dimorphism of snow crab, as males grow to be larger and so escape some of the size-dependent predation to which the smaller females would be exposed. 

The sex ratios of mature snow crab have been consistently very low since 1998 for all areas, being dominated by males. This is a very different situation relative to the very high ratios observed in the Gulf of St. Lawrence where male limitation is an issue. What caused this historically poor reproductive potential is not known, especially as this fishery is a male-only fishery. A partial explanation may be associated with the differential predation pressures upon immature snow crab due to spatial distribution and sexual dimorphism (above). There has however been an increase in sex ratios (\% female) of mature snow crab in 2005 and 2006 to levels comparable with pre-1998 levels (just under 40\% female in all areas; Figure~\ref{ts.sex.ratio.mature}). This more balanced sex ratio is indicative of the ENS crab entering an important reproductive mode after almost 5 years of low reproductive output. A strong coherence in the manner in which sex ratios have been changing over time throughout the Scotian Shelf was evident (Figure~\ref{map.sex.ratio.mature}). For the first time since the late 1990s, a more heterogeneous (mixed) distribution of sexes was observed: pockets of male dominated areas were mixed with pockets of female dominated areas (Figure~\ref{map.sex.ratio.mature}). During mating periods, mature crab would therefore be able find the other sex with minimal movement.

Primiparous females mate during their moulting stage when they are highly vulnerable without protection from a large male \citep{Watson1972, Hooper1986}. If their mate is small and unable to definitively defend against other potential mates, females have been observed to be torn apart during the agonistic behaviour (fighting). When potential mates are small, females have been observed to refuse mating and in the process of refusal are also killed. Thus, an abundance of large males would certainly increase the likelihood of successful reproduction for the new wave of maturing females. Further, in an evolutionary context, if heavy fishing of large males causes increased mating with early maturing dwarf sized males, a greater selection for such traits would be passed onto future generations, potentially leading to stunted populations (a trend observed in many highly exploited species). This however, is a genetic effect occurring over generational time scales. It is important to note that phenotypic plasticity can accelerate the rate of morphometric change in this adaptive species.


\subsubsection{Numerical abundance} 

The number of immature females caught\footnote{Please note that most categories of snow crab are likely under-estimated. Their primary use is to understand their general trends over time. In contrast, fishable biomass is likely slightly over-estimated due to the more expansive criteria used for determining snow crab habitat (see Methods).} in the trawl surveys has been increasing since historical lows in 2002 (N-ENS) and 2003 (S-ENS), reaching historical highs in 2006 (Figure~\ref{ts.female.no}). Most of these females are found in Glace Bay Hole, inner areas of N-ENS, Chedabucto Bay and the Misaine Bank of S-ENS (Figure~\ref{maps.fi.no}). These are the core areas where primiparous females will be mating in the spring of 2008. 

The numerical abundance of mature females are beginning to rapidly increase: newly matured females from the leading edge of the recruitment pulse and the continued longevity of the older mature female population (Figures~\ref{maps.fm.no}; \ref{maps.fm.berried.no}). Most of the mature females are currently located in the inshore areas of S-ENS as well the main fishing grounds in N-ENS; these were therefore the core areas where larval production occured in 2006/2007.

The numerical abundance of older immature males of instars 11 and 12 consistently declined throughout the Scotian Shelf (Figure~\ref{ts.mi.no}). However, a resurgence of instars 8 and 9 was seen in 2003 and 2004. Currently, rapid increases in instar 10 and 11 immature males have been occurring with the exception of CFA 4X. These crab represent the leading edge of the pulse of recruits that the fishery will be dependent upon for the next decade. Skip moulters have been similarly decreasing in number throughout the Scotian Shelf. Their numbers should begin to increase once the pre-recruit pulse enters mature size ranges (Figure~\ref{ts.mi.sm.no}).

Newly matured crab (CC1 and CC2; Figure~\ref{ts.maCC12.no}) are also currently in low numbers. However, instars 9 and 10  (destined to be dwarfed crab, inaccessible to the fishery due to their small size) have begun to increase in number due to their earlier maturation. These are the males that will mate with the maturing females if there are no large male competitors. The early reduction in number of larger (especially instar 12 crab) CC1 and CC2 crab since 2001 in N-ENS may be due to their generally higher exploitation rates relative to S-ENS and the more recent nature of the expansion of the fishing ground in S-ENS (Table \ref{fishing.mortality}). The gradual and consistent reduction in hard-shelled snow crab (CC3 and CC4) is evident for all instars on the Scotian Shelf (Figure~\ref{ts.maCC34.no}). The numerical abundance estimates of carapace condition 5 crab are close to being undetectable on the Scotian Shelf (Figure~\ref{ts.maCC5.no}). Their low representation in survey data and the fishery-observed data (\lt~1\%) may be indicative of high exploitation rates upon the hard-shelled phase. 


\subsubsection{Fishable biomass}

In N-ENS, the post-fishery fishable biomass of snow crab in 2006 was estimated to be 720~t (with a 95\% confidence range of: 580 to 880~t; see Figure~\ref{ts.R0.mass}) -- a decline of approximately 50\% relative to 2005 (1460~t). The declines were evident in all areas in N-ENS (Figure~\ref{maps.R0.mass}).

In S-ENS, the post-fishery fishable biomass\footnote{Fishable biomass is likely over-estimated in S-ENS, see Methods.} of snow crab in 2006 was estimated to be 25.4~\Xthou~t (with a 95\% confidence range of: 23.4 to 27.5~\Xthou~t; Figure~\ref{ts.R0.mass}) -- an increase of 10\% from 23.1~\Xthou~t in 2005, the first increase in fishable biomass since 2001. The majority of the increases were evident on the continental slope areas as well as a slight amelioration in the Chedabucto area (Figure~\ref{maps.R0.mass}).

In CFA 4X, the pre-fishery fishable biomass\footnote{The season in CFA 4X begins 1 October and ends 31 May.} was estimated to be 850~t (with a 95\% confidence range of: 700 to 1,000~t; see Figure~\ref{ts.R0.mass}. This represents a decline of approximately 13.5\% from approximately 990 t in 2005 (pre-fishery).


\subsection{Process variables}

All information about the following process (flow) variables are dependent upon a correct model representation of fisheries activity and reasonable survey results. As the model requires further validation and testing, and also various caveats related to fishery and survey data quality, the results of this section must be treated with appropriate caution.

\subsubsection{Recruitment}

The true recruitment into the mature fishable biomass is not known. It is however, possible to provide an indication of the number of recruits found during the survey (soft-shelled crab). It must be emphasised that as the survey is conducted in late autumn (since 2002), an unknown proportion of the annual recruitment would have already progressed into the mature fishable biomas; and the catchability of soft-shelled crab is likely reduced due to their behaviour of sheltering in rocky burrows. Thus the recruitment index (Figures~\ref{ts.R1a.no}, \ref{maps.R1a.no}) is only a partial and biased index that is sensitive to the annual variations in temperature, food availability and crowding, factors that control the onset of moutling and the speed of shell hardenning. 


\subsubsection{Natural mortality}

\citet{Wade2003} suggested that mortality rates for legal sized crab resident in the southern Gulf of St. Lawrence are within the range of 0.26 to 0.48. Based upon diet studies \citep{Bundy2004}, very few natural predators seem to exist for large snow crabs (i.e., legal sized) on the Scotian Shelf. This has been particularly the case since the demise of most large-bodied predatory groundfish from the eastern part of the Scotian Shelf \citep{Choi2004}. As such, these natural mortality estimates may be higher than those occuring on the Scotian Shelf. However, concentrations of thorny skate have been found in the offshore slope areas, suggesting that mortality may be high for small crabs (instars 7 and less) as well as soft-shelled crab in these areas. Other potential mortality factors include: seals (near Sable Island; although see arguments to the contrary in Ecosystem considerations, above), soft-shell mortality, unreported landings, by-catch in other fisheries (lobster and other crab traps, long-lining, gill-nets, trawling) and potentially activities associated with exploration and development of oil and gas reserves.


\subsubsection{Fishing mortality}


Exploitation rate\footnote{The relative abundance of CC5 crab (Figure~\ref{ts.maCC5.no}) has been used historically as an index of ``exploitation rate''. The rationale being that any increase in numbers of this stage of crab being indicative of ``insufficient'' harvesting as this stage of crab is considered by some to be ``wasted'' to the fishery. The temporal dynamics of CC5 abundance is ambiguous as their numbers are generally below the detection limit of the current survey design. Small increases in a few instars have been observed with the general senescense of the male snow crab population.} estimates from abundance estimates increased exponentially from 2001 to 2004 in N-ENS (Figure \ref{relativeF}). Large reductions in TAC were implemented for the 2005 season, resulting in sharp reductions of exploitation rates from 47\% to 30\% by biomass. In 2006, the exploitation rate increased to 43\%. In S-ENS, the relative exploitation rates has been generally stable between 15 to 20\%. In 2006, due to conservation concerns, the exploitation rate was reduced to 15\%, relative to 22\% in 2005. In CFA 4X, exploitation rates have been comparable with those observed in S-ENS; in 2005, the CFA 4X exploitation rate was 25\%.


\section{Recommendations}

The long-term, forward-thinking precautionary approach adopted by the S-ENS fishers over the past 3 years has allowed the S-ENS fishers to bridge the recruitment gap that has lasted for more than 5 years on the Scotian shelf. Now, with the pending recruitment pulses nearing full entry into the fishable biomass and with large numbers of females having had the opportunity to mate with larger and older males (in 2006 and in the spring of 2007), the health of the S-ENS stock can be said to improving. For the first time since 2001, an increase in fishable biomass has been observed and a strong and steady recruitment is expected for the next 4 to 5 years. Forecasts into the future (Figure~\ref{projections.south}) indicate that there is a strong potential for this fishery for at least the next five years but this strength will be dependent upon how aggressively it is exploited. Based upon the crude projection scenarios, maintaining an exploitation rate between 10 and 20\% would provide the greatest longevity to the fishery. Ensuring the longevity of the fishable biomass is important as on the Scotian shelf recruitment has so far occurred in pulses and not as a constant stream, such as is the case in the Gulf of St. Lawrence. A status-quo or a marginal increase in TAC is suggested.

In contrast, in N-ENS, extremely high exploitation rates have pushed the fishable component of the N-ENS snow crab population to an historic low. They have not been able to ``bridge'' the recruitment gap. What will occur to the fishable biomass in 2007 is highly uncertain as the leading edge of the recruitment pulse is still another year away from entry into the fishable biomass. The reproductive females have mated, but the males that were there to mate were predominantly small, immature males. Associated female mortality may have been elevated as a consequence. Indeed, with the more depeleted fishable biomass, the occurrence of soft and white crab may also be expected to increase in N-ENS for 2007. Projections of fishable biomass for N-ENS (Figures~\ref{projections.north}) suggest that exploitation at recent rates of 40 to 50\% are unlikely to help maintain the longevity of a strong fishable biomass. A reduction in TAC is suggested.

In CFA 4X, exploitation rates have been comparable to that of S-ENS near 20\%. If maintained at this level, the CFA 4X population may be follow a trajectory similar to that of S-ENS (Figure~\ref{projections.4x}). However, large inter-annual temperature variations in the area increase the uncertainty associated with these scenarios. The CFA 4X snow crab population exists in a state intermediate between N- and S-ENS in terms of the 5-year population trends. This is the case even though they are the southern-most population, existing in a more ``marginal'' environment relative to the ``prime'' areas of N-ENS. Until the CFA 4X fishery is completed, it is not yet possible to provide any clear advice for this area.

The longevity of the fishable biomass (i.e., the stabilisation of the fishery) can be improved by fishing solely upon morphometrically mature crab. The arguments for this approach is as follows:

\begin{itemize}
  \item Fishing mature crab would allow them to mate as the fishing season is post-mating season. This has the important result of reducing Darwinian selection for early maturation which is a long-term hazzard for any fishery that harvests immature individuals.
  \item The capture of immature crab (``pencil claws'') reduces the longevity of the fishable biomass directly relative to a mature-only fishery. The time difference is 2 to 3 years as immature crab go through a soft- and white shelled phases that exclude them from the fishery. Specifically targeting mature (male) crabs would be a more optimal exploitation strategy (CC3 and CC4 crab).
  \item There is a sigificantly large weight increase if immature crab are allowed to grow and mature (an increase of 250 to even 400\%; Figure~\ref{growth.curves}).
\end{itemize}

In the 2007 season, much of the fishable biomass may be composed of immature individuals. Excessive fishing of this component of the fishable biomass is unwise.

High catches of soft-shelled crab will likely continue to be a major issue for the next 3 to 4 years. Timely responses from industry to avoid fishing in areas showing potential or actual high incidence of soft crab must continue if unnecessary mortality of future recruits is to be averted.


\section{Acknowledgments}

We thank W. T. Grover Fisheries Ltd., the captain John Baker and the crew of the M/V Gentle Lady for their valued fishing experience and diligence in providing a safe and hospitable environment for the conduct of the survey.

Linda Worth-Bezanson has been the enabler working in the background, ensuring that all aspects of the scientific assessment run smoothly and optimally. Her efforts are greatfully acknowledged.

Finally, this assessment could not have been completed without the numerous contributions of experience and time of the stewards of the fishery, the snow crab fishers of the Scotian Shelf.



%%%% bibliography %%%%%

\clearpage

\bibliography{primary}
\bibliographystyle{jae}


\clearpage 

\section{Tables}

\vspace{2cm}


\begin{table}[h]
\caption{Snow crab fishing seasons on the Scotian Shelf in the year 2006.}
\label{tab:management.measures}
\begin{center}
\begin{tabular}{c r r r r r r r}
\hline \hline
Area & Season \\
\hline
N-ENS & 22 July - 15 Sept  \\
S-ENS (CFA 23) & 1 June - 30 Sept  (extended to 14 Oct)  \\
S-ENS (CFA 24) & 1 June - 30 Sept  \\
\hline
\end{tabular}
\end{center}
\end{table}


\clearpage \newpage
\begin{table}
\caption{Summary of snow crab fisheries activity of N-ENS.}
\label{tab:fisheries.summary.north}
\begin{center}
\begin{tabular}{c r r r r r}
\hline \hline
Year & Landings (t) & TAC (t) & Licenses & CPUE (kg/trap) & Effort ($10^3$ traps) \\
\hline
1997 &   534 &   540 & 74 & 23.3 & 22.9 \\
1998 &   657 &   660 & 74 & 41.6 & 15.8 \\
1999 &   899 &   900 & 78 & 54.8 & 16.4 \\
2000 & 1,017 & 1,015 & 79 & 68.3 & 14.9 \\
2001 & 1,066 & 1,065 & 80 & 94.3 & 11.3 \\
2002 & 1,495 & 1,493 & 80 & 101.0 & 14.8 \\
2003 & 1,492 & 1,493 & 80 & 76.8 & 19.4 \\
2004 & 1,418 & 1,416 & 79 & 60.6 & 23.4 \\
2005 &   562 &   566 & 78 & 30.6 & 18.4 \\
2006 &   486 &   487 & 78 & 35.6 & 13.7 \\
\hline
\end{tabular}
\end{center}
\end{table}


\begin{table}
\caption{Summary of snow crab fisheries activity of S-ENS. Catch rates and trap hauls for 2001 to 2004 are calculated excluding slope area landings and effort as they were design-constrained trap surveys, however these landings are included in total landings and TACs. These slope allocations were for 200~t in 2001-2002 and 300~t in 2003-2004.}
\label{tab:fisheries.summary.south}
\begin{center}
\begin{tabular}{c r r r r r}
\hline \hline
Year & Landings (t) & TAC (t) & Licenses & CPUE (kg/trap) & Effort ($10^3$ traps) \\
\hline
1997 & 1,157 & 1,163 &  59 &  50.9  &  22.7 \\
1998 & 1,558 & 1,671 &  67 &  68.9  &  22.6 \\
1999 & 2,700 & 2,700 &  -  &  71.1  &  38.0 \\
2000 & 8,701 & 8,799 & 158 &  85.0  & 102.4  \\
2001 & 9,048 & 9,023 & 163 &  87.8  & 103.1  \\
2002 & 8,891 & 9,022 & 149 & 111.7  &  79.6  \\
2003 & 8,836 & 9,113 & 145 &  98.6  &  89.6   \\
2004 & 8,022 & 8,241 & 130 & 105.6  &  76.0  \\
2005 & 6,407 & 6,353 & 118.5 & 109.4  &  58.6  \\
2006 & 4,486 & 4,510 & 118.5 &  90.6  &  49.4  \\
\hline
\end{tabular}
\end{center}
\end{table}




\begin{table}[!tbp]
 \caption{Mean carapace width of the various male snow crab instars and life stages. The stages are immature (imm), immature skip moulters (imm.sm), carapace condition 1 and 2 (CC1to2), carapace condition 3 and 4 (CC3to4) and carapace condition 5 (CC5). The numeric suffix to stage indicates the instar. Thus: CC1to2.9 is carapace condition 1 or 2 of instar 9. \label{table.cw.stages}} 
 \begin{center}
 \begin{tabular}{lrrrr}\hline\hline
\multicolumn{1}{l}{\bfseries Stage}&
\multicolumn{4}{c}{\bfseries Mean carapace width (cm)}
\\ \cline{2-5}
\multicolumn{1}{l}{}&
\multicolumn{1}{c}{N-ENS}&
\multicolumn{1}{c}{S-ENS}&
\multicolumn{1}{c}{CFA 4X}&
\multicolumn{1}{c}{Scotian Shelf}
\\ \hline
imm.5&$ 15.1$&$ 14.8$&$ 14.9$&$ 14.9$\\
imm.6&$ 20.1$&$ 20.0$&$ 19.3$&$ 20.0$\\
imm.7&$ 27.0$&$ 26.8$&$ 26.8$&$ 26.9$\\
imm.8&$ 35.1$&$ 35.6$&$ 36.5$&$ 35.6$\\
imm.9&$ 47.1$&$ 48.3$&$ 49.0$&$ 48.1$\\
imm.10&$ 64.3$&$ 65.2$&$ 64.4$&$ 65.1$\\
imm.11&$ 88.3$&$ 86.8$&$ 84.2$&$ 87.0$\\
imm.12&$107.6$&$107.7$&$108.4$&$107.7$\\
imm.sm.9&$ 50.6$&$ 50.4$&$ 52.9$&$ 50.4$\\
imm.sm.10&$ 67.5$&$ 68.2$&$ 67.1$&$ 68.1$\\
imm.sm.11&$ 89.2$&$ 88.1$&$ 87.9$&$ 88.3$\\
imm.sm.12&$109.0$&$108.4$&$108.5$&$108.4$\\
CC1to2.9&$ 46.0$&$ 47.6$&$ 49.7$&$ 47.3$\\
CC1to2.10&$ 66.1$&$ 65.9$&$ 64.8$&$ 66.0$\\
CC1to2.11&$ 88.1$&$ 87.2$&$ 87.8$&$ 87.4$\\
CC1to2.12&$113.9$&$113.9$&$110.0$&$114.4$\\
CC1to2.13&$137.2$&$139.1$&$138.5$&$138.5$\\
CC3to4.9&$ 50.7$&$ 51.2$&$ 50.6$&$ 50.6$\\
CC3to4.10&$ 68.3$&$ 68.1$&$ 65.7$&$ 68.2$\\
CC3to4.11&$ 89.4$&$ 89.9$&$ 90.0$&$ 89.7$\\
CC3to4.12&$112.8$&$114.0$&$110.4$&$113.9$\\
CC3to4.13&$138.2$&$138.1$&$138.1$&$138.1$\\
CC5.9&$ 51.9$&$ 53.9$&$ 52.9$&$ 52.9$\\
CC5.10&$ 67.9$&$ 69.5$&$ 68.8$&$ 68.8$\\
CC5.11&$ 87.6$&$ 88.5$&$ 87.9$&$ 87.9$\\
CC5.12&$109.1$&$110.5$&$112.9$&$109.9$\\
CC5.13&$141.1$&$141.1$&$141.1$&$141.1$\\
\hline
\end{tabular}
\end{center}
\end{table}


\begin{table}[!tbp]
 \caption{Mean body mass the various male snow crab instars and life stages. The stages are immature (imm), immature skip moulters (imm.sm), carapace condition 1 and 2 (CC1to2), carapace condition 3 and 4 (CC3to4) and carapace condition 5 (CC5). The numeric suffix to stage indicates the instar. Thus: CC1to2.9 is carapace condition 1 or 2 of instar 9.\label{table.mass.stages}} 
 \begin{center}
 \begin{tabular}{lrrrr}\hline\hline
\multicolumn{1}{l}{\bfseries Stage}&
\multicolumn{4}{c}{\bfseries Mean body mass (g).}
\\ \cline{2-5}
\multicolumn{1}{l}{}&
\multicolumn{1}{c}{N-ENS}&
\multicolumn{1}{c}{S-ENS}&
\multicolumn{1}{c}{CFA 4X}&
\multicolumn{1}{c}{Scotian Shelf}
\\ \hline
imm.5&$   0.7$&$   0.8$&$   1.9$&$   0.8$\\
imm.6&$   2.0$&$   2.0$&$   2.6$&$   1.9$\\
imm.7&$   5.2$&$   5.2$&$   6.4$&$   5.2$\\
imm.8&$  12.4$&$  14.2$&$  18.8$&$  13.6$\\
imm.9&$  33.2$&$  38.5$&$  43.0$&$  37.3$\\
imm.10&$  97.4$&$ 105.7$&$ 108.6$&$ 104.7$\\
imm.11&$ 277.1$&$ 265.4$&$ 250.0$&$ 266.4$\\
imm.12&$ 511.0$&$ 512.8$&$ 547.0$&$ 510.0$\\
imm.sm.9&$  51.6$&$  50.9$&$  58.1$&$  51.0$\\
imm.sm.10&$ 125.3$&$ 129.8$&$ 124.1$&$ 129.4$\\
imm.sm.11&$ 299.2$&$ 288.7$&$ 286.2$&$ 290.4$\\
imm.sm.12&$ 557.7$&$ 545.9$&$ 548.2$&$ 545.5$\\
CC1to2.9&$  32.4$&$  37.0$&$  48.9$&$  35.2$\\
CC1to2.10&$ 114.0$&$ 113.0$&$ 102.3$&$ 113.4$\\
CC1to2.11&$ 275.7$&$ 261.9$&$ 285.4$&$ 263.9$\\
CC1to2.12&$ 591.6$&$ 585.5$&$ 533.1$&$ 590.2$\\
CC1to2.13&$1036.9$&$1101.3$&$1082.2$&$1082.2$\\
CC3to4.9&$  51.3$&$  53.3$&$  51.5$&$  51.5$\\
CC3to4.10&$ 130.1$&$ 129.4$&$ 116.0$&$ 129.5$\\
CC3to4.11&$ 301.4$&$ 307.4$&$ 307.9$&$ 305.7$\\
CC3to4.12&$ 618.6$&$ 640.4$&$ 575.6$&$ 637.6$\\
CC3to4.13&$1141.1$&$1139.3$&$1139.0$&$1139.0$\\
CC5.9&$  54.9$&$  61.6$&$  58.2$&$  58.2$\\
CC5.10&$ 126.8$&$ 136.9$&$ 133.0$&$ 133.0$\\
CC5.11&$ 284.3$&$ 293.1$&$ 287.5$&$ 287.5$\\
CC5.12&$ 556.6$&$ 579.3$&$ 608.7$&$ 569.6$\\
CC5.13&$1217.6$&$1217.6$&$1217.6$&$1217.6$\\
\hline
\end{tabular}
\end{center}
\end{table}




\begin{table}[ht]
\caption{Snow crab carapace conditions and their description. Hardness is measured by a durometer.}
\label{table.carapace.condition.description}
\begin{center}
\begin{tabular*}{\textwidth}{@{\extracolsep{\fill}} p{1.5cm} c c p{6cm} p{3cm}}
\hline \hline
Carapace condition & Category & Hardness & Description & Age after terminal moult (approx) \\
\hline
1 & New soft      & $<$ 68    & claws easily bent, carapace soft, brightly coloured, iridescent, no epibionts     & 0 - 5 months \\
2 & Clean         & variable  & claws easily bent, carapace soft, brightly coloured, iridescent, some epibionts   &  5 months - 1 year \\
3 & Intermediate  & $>$ 68    & carapace hard, dull brown dorsally, yellow-brown ventrally, no iridescence, shell abrasion, epibionts & 8 months - 3 years \\
4 & Old           & $>$ 68    & carapace hard, very dirty, some decay at leg joints, some epibionts & 2 - 5 years  \\
5 & Very old      & variable  & carapace soft, very dirty, extensive decay, extensive epibionts & 4 - 6 years \\
\hline
\end{tabular*}
\end{center}
\end{table}



\begin{landscape}

\begin{table}[!tbp]
 \caption{Pseudo-Markov transition matrix used for projections in N-ENS.\label{tm.nens}} 
 \begin{center}
\begin{scriptsize}
 \begin{tabular}{llllllllllllllllllllllllllll}\hline\hline
\multicolumn{28}{c}{\bfseries Stage (t)}
\\ 
\multicolumn{1}{c}{ }&
\multicolumn{8}{c}{Immature}&
\multicolumn{4}{c}{Immature skip moulters}&
\multicolumn{5}{c}{CC1/2}&
\multicolumn{5}{c}{CC3/4}&
\multicolumn{5}{c}{CC5}
\\
\multicolumn{1}{c}{Stage (t+1)}&
\multicolumn{1}{c}{5}&
\multicolumn{1}{c}{6}&
\multicolumn{1}{c}{7}&
\multicolumn{1}{c}{8}&
\multicolumn{1}{c}{9}&
\multicolumn{1}{c}{10}&
\multicolumn{1}{c}{11}&
\multicolumn{1}{c}{12}&
\multicolumn{1}{c}{9}&
\multicolumn{1}{c}{10}&
\multicolumn{1}{c}{11}&
\multicolumn{1}{c}{12}&
\multicolumn{1}{c}{9}&
\multicolumn{1}{c}{10}&
\multicolumn{1}{c}{11}&
\multicolumn{1}{c}{12}&
\multicolumn{1}{c}{13}&
\multicolumn{1}{c}{9}&
\multicolumn{1}{c}{10}&
\multicolumn{1}{c}{11}&
\multicolumn{1}{c}{12}&
\multicolumn{1}{c}{13}
\\ \hline

imm.5&-&-&-&-&-&-&-&-&-&-&-&-&-&-&-&-&-&-&-&-&-&-\\
imm.6&1.14&-&-&-&-&-&-&-&-&-&-&-&-&-&-&-&-&-&-&-&-&-\\
imm.7&-&1.82&-&-&-&-&-&-&-&-&-&-&-&-&-&-&-&-&-&-&-&-\\
imm.8&-&-&1.88&-&-&-&-&-&-&-&-&-&-&-&-&-&-&-&-&-&-&-\\
imm.9&-&-&-&1.43&-&-&-&-&-&-&-&-&-&-&-&-&-&-&-&-&-&-\\
imm.10&-&-&-&-&1.42&-&-&-&-&-&-&-&-&-&-&-&-&-&-&-&-&-\\
imm.11&-&-&-&-&-&0.93&-&-&-&-&-&-&-&-&-&-&-&-&-&-&-&-\\
imm.12&-&-&-&-&-&-&0.35&-&-&-&-&-&-&-&-&-&-&-&-&-&-&-\\
imm.sm.9&-&-&-&-&0.76&-&-&-&-&-&-&-&-&-&-&-&-&-&-&-&-&-\\
imm.sm.10&-&-&-&-&-&0.63&-&-&-&-&-&-&-&-&-&-&-&-&-&-&-&-\\
imm.sm.11&-&-&-&-&-&-&0.53&-&-&-&-&-&-&-&-&-&-&-&-&-&-&-\\
imm.sm.12&-&-&-&-&-&-&-&1.13&-&-&-&-&-&-&-&-&-&-&-&-&-&-\\
CC1to2.9&-&-&-&0.21&-&-&-&-&-&-&-&-&-&-&-&-&-&-&-&-&-&-\\
CC1to2.10&-&-&-&-&0.18&-&-&-&0.26&-&-&-&-&-&-&-&-&-&-&-&-&-\\
CC1to2.11&-&-&-&-&-&0.33&-&-&-&0.28&-&-&-&-&-&-&-&-&-&-&-&-\\
CC1to2.12&-&-&-&-&-&-&0.36&-&-&-&0.36&-&-&-&-&-&-&-&-&-&-&-\\
CC1to2.13&-&-&-&-&-&-&-&-&-&-&-&0.28&-&-&-&-&-&-&-&-&-&-\\
CC3to4.9&-&-&-&-&-&-&-&-&-&-&-&-&1.24&-&-&-&-&0.67&-&-&-&-\\
CC3to4.10&-&-&-&-&-&-&-&-&-&-&-&-&-&1.39&-&-&-&-&0.67&-&-&-\\
CC3to4.11&-&-&-&-&-&-&-&-&-&-&-&-&-&-&1.09&-&-&-&-&0.67&-&-\\
CC3to4.12&-&-&-&-&-&-&-&-&-&-&-&-&-&-&-&1.24&-&-&-&-&0.67&-\\
CC3to4.13&-&-&-&-&-&-&-&-&-&-&-&-&-&-&-&-&1.24&-&-&-&-&0.67\\
CC5.9&-&-&-&-&-&-&-&-&-&-&-&-&-&-&-&-&-&0.33&-&-&-&-\\
CC5.10&-&-&-&-&-&-&-&-&-&-&-&-&-&-&-&-&-&-&0.33&-&-&-\\
CC5.11&-&-&-&-&-&-&-&-&-&-&-&-&-&-&-&-&-&-&-&0.33&-&-\\
CC5.12&-&-&-&-&-&-&-&-&-&-&-&-&-&-&-&-&-&-&-&-&0.33&-\\
CC5.13&-&-&-&-&-&-&-&-&-&-&-&-&-&-&-&-&-&-&-&-&-&0.33\\


\hline
\end{tabular}
\end{scriptsize}
\end{center}
\end{table}
\end{landscape}



\begin{landscape}

\begin{table}[!tbp]
 \caption{Pseudo-Markov transition matrix used for projections in S-ENS.\label{tm.sens}} 
 \begin{center}
\begin{scriptsize}
 \begin{tabular}{llllllllllllllllllllllllllll}\hline\hline
\multicolumn{28}{c}{\bfseries Stage (t)}
\\ 
\multicolumn{1}{c}{ }&
\multicolumn{8}{c}{Immature}&
\multicolumn{4}{c}{Immature skip moulters}&
\multicolumn{5}{c}{CC1/2}&
\multicolumn{5}{c}{CC3/4}&
\multicolumn{5}{c}{CC5}
\\
\multicolumn{1}{c}{Stage (t+1)}&
\multicolumn{1}{c}{5}&
\multicolumn{1}{c}{6}&
\multicolumn{1}{c}{7}&
\multicolumn{1}{c}{8}&
\multicolumn{1}{c}{9}&
\multicolumn{1}{c}{10}&
\multicolumn{1}{c}{11}&
\multicolumn{1}{c}{12}&
\multicolumn{1}{c}{9}&
\multicolumn{1}{c}{10}&
\multicolumn{1}{c}{11}&
\multicolumn{1}{c}{12}&
\multicolumn{1}{c}{9}&
\multicolumn{1}{c}{10}&
\multicolumn{1}{c}{11}&
\multicolumn{1}{c}{12}&
\multicolumn{1}{c}{13}&
\multicolumn{1}{c}{9}&
\multicolumn{1}{c}{10}&
\multicolumn{1}{c}{11}&
\multicolumn{1}{c}{12}&
\multicolumn{1}{c}{13}

\\ \hline

imm.5&-&-&-&-&-&-&-&-&-&-&-&-&-&-&-&-&-&-&-&-&-&-\\
imm.6&1.47&-&-&-&-&-&-&-&-&-&-&-&-&-&-&-&-&-&-&-&-&-\\
imm.7&-&1.83&-&-&-&-&-&-&-&-&-&-&-&-&-&-&-&-&-&-&-&-\\
imm.8&-&-&1.29&-&-&-&-&-&-&-&-&-&-&-&-&-&-&-&-&-&-&-\\
imm.9&-&-&-&1.25&-&-&-&-&-&-&-&-&-&-&-&-&-&-&-&-&-&-\\
imm.10&-&-&-&-&1.87&-&-&-&-&-&-&-&-&-&-&-&-&-&-&-&-&-\\
imm.11&-&-&-&-&-&1.7&-&-&-&-&-&-&-&-&-&-&-&-&-&-&-&-\\
imm.12&-&-&-&-&-&-&0.36&-&-&-&-&-&-&-&-&-&-&-&-&-&-&-\\
imm.sm.9&-&-&-&-&0.56&-&-&-&-&-&-&-&-&-&-&-&-&-&-&-&-&-\\
imm.sm.10&-&-&-&-&-&0.32&-&-&-&-&-&-&-&-&-&-&-&-&-&-&-&-\\
imm.sm.11&-&-&-&-&-&-&0.4&-&-&-&-&-&-&-&-&-&-&-&-&-&-&-\\
imm.sm.12&-&-&-&-&-&-&-&0.98&-&-&-&-&-&-&-&-&-&-&-&-&-&-\\
CC1to2.9&-&-&-&0.11&-&-&-&-&-&-&-&-&-&-&-&-&-&-&-&-&-&-\\
CC1to2.10&-&-&-&-&0.1&-&-&-&0.08&-&-&-&-&-&-&-&-&-&-&-&-&-\\
CC1to2.11&-&-&-&-&-&0.17&-&-&-&0.11&-&-&-&-&-&-&-&-&-&-&-&-\\
CC1to2.12&-&-&-&-&-&-&0.15&-&-&-&0.15&-&-&-&-&-&-&-&-&-&-&-\\
CC1to2.13&-&-&-&-&-&-&-&-&-&-&-&0.2&-&-&-&-&-&-&-&-&-&-\\
CC3to4.9&-&-&-&-&-&-&-&-&-&-&-&-&1.84&-&-&-&-&0.67&-&-&-&-\\
CC3to4.10&-&-&-&-&-&-&-&-&-&-&-&-&-&1.42&-&-&-&-&0.67&-&-&-\\
CC3to4.11&-&-&-&-&-&-&-&-&-&-&-&-&-&-&1.23&-&-&-&-&0.67&-&-\\
CC3to4.12&-&-&-&-&-&-&-&-&-&-&-&-&-&-&-&2.39&-&-&-&-&0.67&-\\
CC3to4.13&-&-&-&-&-&-&-&-&-&-&-&-&-&-&-&-&1.84&-&-&-&-&0.67\\
CC5.9&-&-&-&-&-&-&-&-&-&-&-&-&-&-&-&-&-&0.33&-&-&-&-\\
CC5.10&-&-&-&-&-&-&-&-&-&-&-&-&-&-&-&-&-&-&0.33&-&-&-\\
CC5.11&-&-&-&-&-&-&-&-&-&-&-&-&-&-&-&-&-&-&-&0.33&-&-\\
CC5.12&-&-&-&-&-&-&-&-&-&-&-&-&-&-&-&-&-&-&-&-&0.33&-\\
CC5.13&-&-&-&-&-&-&-&-&-&-&-&-&-&-&-&-&-&-&-&-&-&0.33\\


\hline
\end{tabular}
\end{scriptsize}
\end{center}
\end{table}
\end{landscape}


\begin{landscape}
\begin{table}[!tbp]
 \caption{Pseudo-Markov transition matrix used for projections in CFA 4X.\label{tm.4x}} 
 \begin{center}
\begin{scriptsize}
 \begin{tabular}{llllllllllllllllllllllllllll}\hline\hline
\multicolumn{28}{c}{\bfseries Stage (t)}
\\ 
\multicolumn{1}{c}{ }&
\multicolumn{8}{c}{Immature}&
\multicolumn{4}{c}{Immature skip moulters}&
\multicolumn{5}{c}{CC1/2}&
\multicolumn{5}{c}{CC3/4}&
\multicolumn{5}{c}{CC5}
\\
\multicolumn{1}{c}{Stage (t+1)}&
\multicolumn{1}{c}{5}&
\multicolumn{1}{c}{6}&
\multicolumn{1}{c}{7}&
\multicolumn{1}{c}{8}&
\multicolumn{1}{c}{9}&
\multicolumn{1}{c}{10}&
\multicolumn{1}{c}{11}&
\multicolumn{1}{c}{12}&
\multicolumn{1}{c}{9}&
\multicolumn{1}{c}{10}&
\multicolumn{1}{c}{11}&
\multicolumn{1}{c}{12}&
\multicolumn{1}{c}{9}&
\multicolumn{1}{c}{10}&
\multicolumn{1}{c}{11}&
\multicolumn{1}{c}{12}&
\multicolumn{1}{c}{13}&
\multicolumn{1}{c}{9}&
\multicolumn{1}{c}{10}&
\multicolumn{1}{c}{11}&
\multicolumn{1}{c}{12}&
\multicolumn{1}{c}{13}

 \\ \hline

imm.5&-&-&-&-&-&-&-&-&-&-&-&-&-&-&-&-&-&-&-&-&-&-\\
imm.6&1.01&-&-&-&-&-&-&-&-&-&-&-&-&-&-&-&-&-&-&-&-&-\\
imm.7&-&0.86&-&-&-&-&-&-&-&-&-&-&-&-&-&-&-&-&-&-&-&-\\
imm.8&-&-&1.38&-&-&-&-&-&-&-&-&-&-&-&-&-&-&-&-&-&-&-\\
imm.9&-&-&-&1.28&-&-&-&-&-&-&-&-&-&-&-&-&-&-&-&-&-&-\\
imm.10&-&-&-&-&1.21&-&-&-&-&-&-&-&-&-&-&-&-&-&-&-&-&-\\
imm.11&-&-&-&-&-&1.28&-&-&-&-&-&-&-&-&-&-&-&-&-&-&-&-\\
imm.12&-&-&-&-&-&-&0.14&-&-&-&-&-&-&-&-&-&-&-&-&-&-&-\\
imm.sm.9&-&-&-&-&0.27&-&-&-&-&-&-&-&-&-&-&-&-&-&-&-&-&-\\
imm.sm.10&-&-&-&-&-&0.13&-&-&-&-&-&-&-&-&-&-&-&-&-&-&-&-\\
imm.sm.11&-&-&-&-&-&-&0.46&-&-&-&-&-&-&-&-&-&-&-&-&-&-&-\\
imm.sm.12&-&-&-&-&-&-&-&0.33&-&-&-&-&-&-&-&-&-&-&-&-&-&-\\
CC1to2.9&-&-&-&0.03&-&-&-&-&-&-&-&-&-&-&-&-&-&-&-&-&-&-\\
CC1to2.10&-&-&-&-&0.13&-&-&-&0.03&-&-&-&-&-&-&-&-&-&-&-&-&-\\
CC1to2.11&-&-&-&-&-&0.06&-&-&-&0.06&-&-&-&-&-&-&-&-&-&-&-&-\\
CC1to2.12&-&-&-&-&-&-&0.25&-&-&-&0.25&-&-&-&-&-&-&-&-&-&-&-\\
CC1to2.13&-&-&-&-&-&-&-&-&-&-&-&0.03&-&-&-&-&-&-&-&-&-&-\\
CC3to4.9&-&-&-&-&-&-&-&-&-&-&-&-&1.29&-&-&-&-&0.67&-&-&-&-\\
CC3to4.10&-&-&-&-&-&-&-&-&-&-&-&-&-&1.29&-&-&-&-&0.67&-&-&-\\
CC3to4.11&-&-&-&-&-&-&-&-&-&-&-&-&-&-&1.29&-&-&-&-&0.67&-&-\\
CC3to4.12&-&-&-&-&-&-&-&-&-&-&-&-&-&-&-&1.29&-&-&-&-&0.67&-\\
CC3to4.13&-&-&-&-&-&-&-&-&-&-&-&-&-&-&-&-&1.29&-&-&-&-&0.67\\
CC5.9&-&-&-&-&-&-&-&-&-&-&-&-&-&-&-&-&-&0.33&-&-&-&-\\
CC5.10&-&-&-&-&-&-&-&-&-&-&-&-&-&-&-&-&-&-&0.33&-&-&-\\
CC5.11&-&-&-&-&-&-&-&-&-&-&-&-&-&-&-&-&-&-&-&0.33&-&-\\
CC5.12&-&-&-&-&-&-&-&-&-&-&-&-&-&-&-&-&-&-&-&-&0.33&-\\
CC5.13&-&-&-&-&-&-&-&-&-&-&-&-&-&-&-&-&-&-&-&-&-&0.33\\

\hline
\end{tabular}
\end{scriptsize}
\end{center}
\end{table}
\end{landscape}





\begin{table}
\caption{By-catch (kg) estimates from the snow crab fishery. The estimates are extrapolated from at-sea-observed by-catch and at-sea-observer coverage, by biomass. Note that the snow crab fishery is in general a highly species-specific fishery with extremely low by-catch of other species. By-catch levels have been $<$~0.025\%  of total landings in the past three years, with most by-catch species being other crabs. At-sea-observers have noted that three leatherback turtles had been entangled in buoy lines, however, they were all released with little to no visable harm. }
\label{table.bycatch}
\begin{center}
\begin{tabular}{lrrrr}
  
  \hline\hline
  
  \multicolumn{1}{l}{Species} &
  \multicolumn{4}{c}{Extrapolated by-catch (kg)}

  \\

  \multicolumn{1}{l}{} &
  \multicolumn{1}{r}{2004}&
  \multicolumn{1}{r}{2005}&
  \multicolumn{1}{r}{2006}&
  \multicolumn{1}{r}{Total (3-year)}

  \\ 
  \hline
  \\

American Lobster  &  21  & 0  & 75  & 96  \\
American Plaice  & 0  & 10  & 0  & 10  \\
Basket Star  & 0  & 68  & 0  & 68  \\
Cod  & 231  & 10  & 0  & 241  \\
Grenadier  & 0  & 10  & 0  & 10  \\
Halibut  & 0  & 890  & 0  & 890  \\
Hermit Crab  & 105  & 0  & 0  & 105  \\
Iceland Scallop  & 0  & 10  & 0  & 10  \\
Jellyfish  & 158  & 0  & 0  & 158  \\
Jonah Crab  & 21  & 587  & 11  & 619  \\
Northern Shrimp  & 0  & 10  & 0  & 10  \\
Northern Stone Crab  & 494  & 127  & 172  & 793  \\
Red Crab  & 0  & 20  & 0  & 20  \\
Redfish  & 0  & 10  & 32  & 42  \\
Rock Crab  & 0  & 0  & 32  & 32  \\
Sea Cucumber  & 0  & 59  & 22  & 81  \\
Sea Raven  & 0  & 29  & 0  & 29  \\
Sea Urchin  & 0  & 0  & 11  & 11  \\
Snailfish  & 0  & 10  & 0  & 10  \\
Spiny Crab  & 21  & 0  & 0  & 21  \\
Spotted Wolffish  & 389  & 0  & 54  & 443  \\
Striped Wolffish  & 126  & 0  & 54  & 180  \\
Thorny Skate  & 53  & 10  & 32  & 95  \\
Toad Crab  & 474  & 49  & 32  & 555  \\
Turbot  & 21  & 0  & 0  & 21  \\
Windowpane Flounder  & 42  & 0  & 0  & 42  \\
Witch Flounder  & 0  & 10  &  0  &10  \\
\\
Snow Crab TAC  & 9,336,000  & 6,996,000  & 4,997,000  & 21,329,000 \\

\hline
\end{tabular}
\end{center}
\end{table}



\clearpage \newpage
\begin{table}
  \caption{Carapace condition of crab \lt~95~mm~CW (percent by number) over time for N-ENS from at-sea-observed data.}
  \label{table.CC.small.north.obs}
 \begin{center}
\begin{tabular}{lrrrrr}
\hline\hline
\multicolumn{1}{l}{\bfseries Year}&
\multicolumn{5}{c}{\bfseries Carapace condition}
\\ \cline{2-6}
\multicolumn{1}{l}{}&
\multicolumn{1}{c}{1}&
\multicolumn{1}{c}{2}&
\multicolumn{1}{c}{3}&
\multicolumn{1}{c}{4}&
\multicolumn{1}{c}{5}
\\ \hline
2004 &$  4.0 $&$  0.3 $&$ 56.2 $&$ 38.5 $&$ 1.0 $\\
2005 &$ 12.4 $&$  1.2 $&$ 41.1 $&$ 43.2 $&$ 2.0 $\\
2006 &$  8.1 $&$ 25.9 $&$ 43.8 $&$ 17.6 $&$ 4.5 $\\
\hline
\end{tabular}
\end{center}
\end{table}


\begin{table}
\caption{Carapace condition of crab \lt~95~mm~CW (percent by number) over time for S-ENS from at-sea-observed data.}
\label{table.CC.small.south.obs}
\begin{center}
\begin{tabular}{lrrrrr}
\hline \hline
\multicolumn{1}{l}{\bfseries Year}&
\multicolumn{5}{c}{\bfseries Carapace condition}
\\ \cline{2-6}
\multicolumn{1}{l}{}&
\multicolumn{1}{c}{1}&
\multicolumn{1}{c}{2}&
\multicolumn{1}{c}{3}&
\multicolumn{1}{c}{4}&
\multicolumn{1}{c}{5}
\\ \hline
2004 &$  7.1 $&$  2.8 $&$ 64.3 $&$ 24.3 $&$ 1.5 $\\
2005 &$ 11.3 $&$ 17.1 $&$ 49.6 $&$ 19.7 $&$ 2.2 $\\
2006 &$ 11.9 $&$ 24.5 $&$ 54.3 $&$  8.0 $&$ 1.3 $\\
\hline
\end{tabular}
\end{center}
\end{table}


\begin{table}
\caption{Carapace condition of crab $\geq$~95~mm~CW (percent by number) over time for N-ENS from at-sea-observed data.}
\label{table.CC.large.north.obs}
\begin{center}
\begin{tabular}{lrrrrr}
\hline \hline
\multicolumn{1}{l}{\bfseries Year}&
\multicolumn{5}{c}{\bfseries Carapace condition}
\\ \cline{2-6}
\multicolumn{1}{l}{}&
\multicolumn{1}{c}{1}&
\multicolumn{1}{c}{2}&
\multicolumn{1}{c}{3}&
\multicolumn{1}{c}{4}&
\multicolumn{1}{c}{5}
\\ \hline
2004 &$  2.5 $&$4.9 $&$ 72.5 $&$ 19.8 $&$ 0.4 $\\
2005 &$ 18.1 $&$2.1 $&$ 61.0 $&$ 18.0 $&$ 0.8 $\\
2006 &$  3.8 $&$9.7 $&$ 71.4 $&$ 13.5 $&$ 1.6 $\\
\hline
\end{tabular}
\end{center}
\end{table}


\begin{table}
\caption{Carapace condition of crab $\geq$~95~mm~CW (percent by number) over time for S-ENS from at-sea-observed data.}
\label{table.CC.large.south.obs}
\begin{center}
\begin{tabular}{lrrrrr}
\hline \hline
\multicolumn{1}{l}{\bfseries Year}&
\multicolumn{5}{c}{\bfseries Carapace condition}
\\ \cline{2-6}
\multicolumn{1}{l}{}&
\multicolumn{1}{c}{1}&
\multicolumn{1}{c}{2}&
\multicolumn{1}{c}{3}&
\multicolumn{1}{c}{4}&
\multicolumn{1}{c}{5}
\\ \hline
2004 &$ 3.2 $&$  3.6 $&$ 74.5 $&$ 18.0 $&$ 0.7 $\\
2005 &$ 5.9 $&$ 11.0 $&$ 68.2 $&$ 14.3 $&$ 0.7 $\\
2006 &$ 5.9 $&$ 17.3 $&$ 69.3 $&$  7.2 $&$ 0.3 $\\
\hline
\end{tabular}
\end{center}
\end{table}


\clearpage \newpage
\begin{table}
\caption{Carapace condition of crab $\geq$~95~mm~CW (percent by number) over time for N-ENS from trawl surveys. The transition from a spring to a fall survey occurred in 2002/2003.}
\label{table.CC.north.trawl}
\begin{center}
\begin{tabular}{lrrrrr}
\hline \hline
\multicolumn{1}{l}{\bfseries Year}&
\multicolumn{5}{c}{\bfseries Carapace condition}
\\ \cline{2-6}
\multicolumn{1}{l}{}&
\multicolumn{1}{c}{1}&
\multicolumn{1}{c}{2}&
\multicolumn{1}{c}{3}&
\multicolumn{1}{c}{4}&
\multicolumn{1}{c}{5}
\\ \hline
1999&$43.47$&$ 5.87$&$31.42$&$19.23$&$0.00$\\
2000&$ 5.16$&$ 0.83$&$76.43$&$17.59$&$0.00$\\
2001&$ 5.92$&$ 0.00$&$68.53$&$25.55$&$0.00$\\
2002&$ 0.00$&$ 2.19$&$67.07$&$30.35$&$0.39$\\
2003&$ 0.79$&$17.75$&$61.93$&$19.52$&$0.00$\\
2004&$ 0.00$&$ 0.00$&$62.95$&$37.05$&$0.00$\\
2005&$ 0.00$&$ 0.00$&$58.90$&$39.07$&$2.03$\\
2006&$ 0.00$&$19.73$&$21.26$&$59.01$&$0.00$\\
\hline
\end{tabular}
\end{center}
\end{table}


\begin{table}
\caption{Carapace condition of crab $\geq$~95~mm~CW (percent by number) over time for S-ENS from trawl surveys. The transition from a spring to a fall survey occurred in 2002/2003.}
\label{table.CC.south.trawl}
\begin{center}
\begin{tabular}{lrrrrr}
\hline\hline
\multicolumn{1}{l}{\bfseries Year}&
\multicolumn{5}{c}{\bfseries Carapace condition}
\\ \cline{2-6}
\multicolumn{1}{l}{}&
\multicolumn{1}{c}{1}&
\multicolumn{1}{c}{2}&
\multicolumn{1}{c}{3}&
\multicolumn{1}{c}{4}&
\multicolumn{1}{c}{5}
\\ \hline
1999&$24.87$&$ 4.18$&$62.37$&$ 8.59$&$0.00$\\
2000&$15.70$&$10.11$&$72.88$&$ 1.31$&$0.00$\\
2001&$17.24$&$ 0.00$&$75.97$&$ 6.79$&$0.00$\\
2002&$ 5.25$&$15.42$&$74.73$&$ 4.60$&$0.00$\\
2003&$ 3.61$&$ 8.13$&$67.63$&$19.39$&$1.24$\\
2004&$ 0.00$&$ 6.93$&$61.25$&$31.82$&$0.00$\\
2005&$ 0.00$&$ 9.33$&$74.25$&$16.41$&$0.00$\\
2006&$ 0.00$&$18.64$&$62.56$&$18.80$&$0.00$\\
\hline
\end{tabular}
\end{center}
\end{table}






\clearpage \newpage
\begin{table}[!tbp]
\caption{The three-year mean annual fishing mortality broken down by region for various life stages of male snow crab. The stages are immature (imm), immature skip moulters (imm.sm), carapace condition 1 and 2 (CC1to2), carapace condition 3 and 4 (CC3to4) and carapace condition 5 (CC5). The numeric suffix to stage indicates the instar. Thus: CC1to2.9 is carapace condition 1 or 2 of instar 9.}
\label{fishing.mortality}
\begin{center}
\begin{tabular}{ccrrrrrrr}
\hline\hline
\multicolumn{1}{c}{Stage}&
\multicolumn{1}{c}{N-ENS}&
\multicolumn{1}{c}{S-ENS}&
\multicolumn{1}{c}{CFA 4X}
\\ \hline

imm.11&0.004&0.003&0.006\\
imm.12&0.014&0.034&0.008\\
imm.sm.11&0.139&0.041&0.024\\
imm.sm.12&0.308&0.309&0.062\\
CC1to2.11&0.013&0.025&0.016\\
CC1to2.12&0.228&0.238&0.016\\
CC1to2.13&0.12&0.132&0.016\\
CC3to4.11&0.1&0.044&0.147\\
CC3to4.12&0.372&0.158&0.175\\
CC3to4.13&0.236&0.132&0.161\\

\hline
\end{tabular}
\end{center}
\end{table}



\clearpage

\section{Figures}

\vspace{2cm}
%1
%\begin{landscape}
\begin{figure}[!h]
\begin{center}
\scalebox{0.55}{\includegraphics*{figs/map\D scotianshelf.png}}
\end{center}
\caption{Location of geographic areas and management areas on the Scotian Shelf.}
\label{fig:map.cfa.locations}
\end{figure}
%\end{landscape}

%2


\clearpage

\begin{figure}[]
\begin{center}
\scalebox{0.13}{\includegraphics*{ts.fisheries/effort\D ns.png}}
\end{center}
\caption{Temporal variations in the fishing effort, expressed as the number of trap hauls. Note the doubling of effort in the year 2000.}
\label{ts.effort}
\end{figure}

%3
\begin{figure}[]
\begin{center}
\scalebox{0.125}{\includegraphics*{ts.fisheries/landings\D ns.png}}
\end{center}
\caption{Temporal variations in the landings of snow crab on the Scotian Shelf (t). Note the sharp increase in landings associated with dramatic increases to TACs and a doubling of fishing effort in the year 2000. The landings follow the TACs with little deviation (and so are not shown).}
\label{ts.landings}
\end{figure}

\clearpage

%4
\begin{figure}[]
\begin{center}
\scalebox{0.12}{\includegraphics*{ts.fisheries/cpue\D ns.png}}
\end{center}
\caption{Temporal variations in catch rates of snow crab on the Scotian Shelf, expressed as kg per trap haul. Trap design and size have changed over time. No correction for these varying trap-types nor soak time and bait-type have been attempted (see Methods).}
\label{ts.cpue}
\end{figure}

%5
\begin{figure}[]
\begin{center}
\scalebox{0.12}{\includegraphics*{habitat.templates/trawl\D depth\D totno\D male\D mat.png}}
\end{center}
\caption{ Habitat preferences of fishable snow crab on the Scotian Shelf. Numerical density of snow crab (\Xthou~kg/\kmsq) as a function of sampling depth (m). 1 standard error bars are presented with numbers indicating the number of stations. A loess filter was used for the heuristic trend line. }
\label{habitattemplate.depth}
\end{figure}

\clearpage

%6
\begin{figure}[]
\begin{center}
\scalebox{0.12}{\includegraphics*{habitat.templates/cpue\D depth\D cpue.png}}
\end{center}
\caption{ Habitat preferences of snow crab on the Scotian Shelf. Fishery catch rates (kg/trap) as a function of trap depth (m). 1 standard error bars are presented with numbers indicating the number of trap hauls. A loess filter was used for the heuristic trend line. }
\label{habitattemplate.cpue.depth}
\end{figure}

%7
\begin{figure}[]
\begin{center}
\scalebox{0.12}{\includegraphics*{habitat.templates/trawl\D temperatures\D totno\D male\D mat.png}}
\end{center}
\caption{ Habitat preferences of fishable snow crab on the Scotian Shelf. Numerical density of snow crab (\Xthou~kg/\kmsq) as a function of bottom temperature (\celsius). 1 standard error bars are presented with numbers indicating the number of stations. A loess filter was used for the heuristic trend line. }
\label{habitattemplate.temperatures}
\end{figure}

\clearpage

%8
\newpage
\begin{figure}[!ht]
\begin{center}
\scalebox{0.6}{\includegraphics*{biology/flow\D model.png}}
\end{center}
\caption{The growth stanzas of male snow crab. Each instar is determined from carapace width bounds obtained from modal analysis and categorised to carapace condition (CC) and maturity from visual inspection and/or maturity equations. Snow crab are resident in each growth stanza for 1 year, with the exception of CC2 to CC4 which are known from mark-recapture studies to last from 2 to 5 years.}
\label{male.growth.stanzas}
\end{figure}

%9
\begin{figure}[t]
\begin{center}
\begin{minipage}{5in}
\scalebox{0.15}{\includegraphics*{growth/male\D growth\D cw.png}}
\end{minipage}
\begin{minipage}{5in}
\scalebox{0.15}{\includegraphics*{growth/male\D growth\D mass.png}}
\end{minipage}
\end{center}
\caption{Growth curves determined from Scotian Shelf male snow crab.}
\label{growth.curves}
\end{figure}


%10

\begin{figure}[ht]
\begin{center}
\rotatebox{0}{\scalebox{1}{\includegraphics*{R/ndensity/cod.2min.crab/annual/totno/totno\D 2004.png}}}
\rotatebox{0}{\scalebox{1}{\includegraphics*{R/ndensity/cod.2min.crab/annual/totno/totno\D 2005.png}}}
\rotatebox{0}{\scalebox{1}{\includegraphics*{R/ndensity/cod.2min.crab/annual/totno/totno\D 2006.png}}}
\end{center}
\caption{Locations of potential predators of snow crab: cod. Scale is $\log_{10}$ (numerical density~[number/\kmsq]).}
\label{map.predators.cod}
\end{figure}



%11
\begin{figure}[ht]
\begin{center}
\rotatebox{0}{\scalebox{1}{\includegraphics*{R/ndensity/thornyskate.2min.crab/annual/totno/totno\D 2004.png}}}
\rotatebox{0}{\scalebox{1}{\includegraphics*{R/ndensity/thornyskate.2min.crab/annual/totno/totno\D 2005.png}}}
\rotatebox{0}{\scalebox{1}{\includegraphics*{R/ndensity/thornyskate.2min.crab/annual/totno/totno\D 2006.png}}}

\end{center}
\caption{Locations of potential predators of snow crab: thorny skate. Scale is $\log_{10}$ (numerical density~[number/\kmsq]).}
\label{map.predators.thorny.skate}
\end{figure}




%12
\begin{figure}[ht]
\begin{center}
\rotatebox{0}{\scalebox{1}{\includegraphics*{R/ndensity/northernshrimp.2min.crab/annual/totno/totno\D 2004.png}}}
\rotatebox{0}{\scalebox{1}{\includegraphics*{R/ndensity/northernshrimp.2min.crab/annual/totno/totno\D 2005.png}}}
\rotatebox{0}{\scalebox{1}{\includegraphics*{R/ndensity/northernshrimp.2min.crab/annual/totno/totno\D 2006.png}}}
\end{center}
\caption{Locations of potential food items of snow crab: northern shrimp. Abundance of these potential food sources roughly match the spatial distributions of snow crab. Scale is $\log_{10}$ (numerical density~[number/\kmsq]).}
\label{map.prey.northern.shrimp}
\end{figure}


%13
\begin{figure}[ht]
\begin{center}
\rotatebox{0}{\scalebox{1}{\includegraphics*{R/ndensity/lessertoadcrab.2min.crab/annual/totno/totno\D 2004.png}}}
\rotatebox{0}{\scalebox{1}{\includegraphics*{R/ndensity/lessertoadcrab.2min.crab/annual/totno/totno\D 2005.png}}}
\rotatebox{0}{\scalebox{1}{\includegraphics*{R/ndensity/lessertoadcrab.2min.crab/annual/totno/totno\D 2006.png}}}
\end{center}
\caption{Locations of potential competitors of snow crab: lesser toad crab. High competitive interactions are probable in inshore areas. Scale is $\log_{10}$ (numerical density~[number/\kmsq]).}
\label{fig:othercrabs.lessertoadcrab}
\end{figure}

%14
\begin{figure}[ht]
\begin{center}
\rotatebox{0}{\scalebox{1}{\includegraphics*{R/ndensity/jonahcrab.2min.crab/annual/totno/totno\D 2004.png}}}
\rotatebox{0}{\scalebox{1}{\includegraphics*{R/ndensity/jonahcrab.2min.crab/annual/totno/totno\D 2005.png}}}
\rotatebox{0}{\scalebox{1}{\includegraphics*{R/ndensity/jonahcrab.2min.crab/annual/totno/totno\D 2006.png}}}
\end{center}
\caption{Locations of potential competitors of snow crab: Jonah crab. High competitive interactions are probable in inshore areas. Scale is $\log_{10}$ (numerical density~[number/\kmsq]).}
\label{fig:othercrabs.jonahcrab}
\end{figure}


%15
\clearpage \newpage

\begin{figure}[ht]
\begin{center}
\begin{tabular}{ccc}
\begin{minipage}{2in}
\scalebox{0.75}{\includegraphics*{R/temperature.oe.2min.crab/annual/t/t\D 1996.png}}
\end{minipage}
\begin{minipage}{2in}
\scalebox{0.75}{\includegraphics*{R/temperature.oe.2min.crab/annual/t/t\D 1997.png}}
\end{minipage}
\begin{minipage}{2in}
\scalebox{0.75}{\includegraphics*{R/temperature.oe.2min.crab/annual/t/t\D 1998.png}}
\end{minipage}
\end{tabular}

\begin{tabular}{ccc}
\begin{minipage}{2in}
\scalebox{0.75}{\includegraphics*{R/temperature.oe.2min.crab/annual/t/t\D 1999.png}}
\end{minipage}
\begin{minipage}{2in}
\scalebox{0.75}{\includegraphics*{R/temperature.oe.2min.crab/annual/t/t\D 2000.png}}
\end{minipage}
\begin{minipage}{2in}
\scalebox{0.75}{\includegraphics*{R/temperature.oe.2min.crab/annual/t/t\D 2001.png}}
\end{minipage}
\end{tabular}

\begin{tabular}{ccc}
\begin{minipage}{2in}
\scalebox{0.75}{\includegraphics*{R/temperature.oe.2min.crab/annual/t/t\D 2002.png}}
\end{minipage}
\begin{minipage}{2in}
\scalebox{0.75}{\includegraphics*{R/temperature.oe.2min.crab/annual/t/t\D 2003.png}}
\end{minipage}
\begin{minipage}{2in}
\scalebox{0.75}{\includegraphics*{R/temperature.oe.2min.crab/annual/t/t\D 2004.png}}
\end{minipage}
\end{tabular}

\begin{tabular}{ccc}
\begin{minipage}{2in}
\scalebox{0.75}{\includegraphics*{R/temperature.oe.2min.crab/annual/t/t\D 2005.png}}
\end{minipage}
\begin{minipage}{2in}
\scalebox{0.75}{\includegraphics*{R/temperature.oe.2min.crab/annual/t/t\D 2006.png}}
\end{minipage}
\end{tabular}

\end{center}
\caption{Map of bottom temperatures on the Scotian Shelf during late summer/autumn.}
\label{maps.bottom.temperature}
\end{figure}



%16
\begin{figure}[ht]
\begin{center}
\scalebox{0.18}{\includegraphics*{ts/surfacearea.png}}
\end{center}
\caption{Annual variations in the surface area of potential snow crab habitat. Note that in N-ENS, very little variations occur between years (i.e., total habitat space does not change, although the quality of the habitat does vary (see Figure~\ref{ts.habitat.surfacearea.temperatures}). In S-ENS, the potential surface has been more variable with the 2005 season being at near the long-term mean for the area. Stronger variations have been occurring in the area since the late 1990s, relative to the  historical record. In CFA 4X, the southern-most limit of the distribution of snow crab, the fluctuations are also quite pronounced. The stipled horizontal line indicates the long-term arithmetic mean surface area within each subarea.}
\label{ts.habitat.surfacearea}
\end{figure}


%17
\begin{figure}[ht]
\begin{center}
\scalebox{0.18}{\includegraphics*{ts/meantemp.png}}
\end{center}
\caption{Annual variations in the summer/autumn mean bottom temperature in the areas of potential snow crab habitat. The stipled horizontal line indicates the long-term arithmetic mean temperature within each subarea.}
\label{ts.habitat.surfacearea.temperatures}
\end{figure}




%18
\newpage
\begin{figure}[h]
\begin{center}
  \begin{minipage}{4in}
    \scalebox{1.25}{\includegraphics*{R/maps/observer\D locations\D 2004.png}}
  \end{minipage}
  \begin{minipage}{4in}
    \scalebox{1.25}{\includegraphics*{R/maps/observer\D locations\D 2005.png}}
  \end{minipage}
  \begin{minipage}{4in}
    \scalebox{1.25}{\includegraphics*{R/maps/observer\D locations\D 2006.png}}
  \end{minipage}
\end{center}
\caption{At-sea-observer monitored locations on the Scotian Shelf for 2004 to 2006.}
\label{observer.locations}
\end{figure}


%19
\newpage
\begin{figure}[ht]
\begin{center}
\begin{minipage}{4in}
\scalebox{1.25}{\includegraphics*{R/maps/survey\D locations\D 2004.png}}
\end{minipage}
\begin{minipage}{4in}
\scalebox{1.25}{\includegraphics*{R/maps/survey\D locations\D 2005.png}}
\end{minipage}
\begin{minipage}{4in}
\scalebox{1.25}{\includegraphics*{R/maps/survey\D locations\D 2006.png}}
\end{minipage}
\end{center}
\caption{Trawl survey locations on the Scotian Shelf for 2004 to 2006. }
\label{trawl.survey.locations}
\end{figure}

%20
\begin{landscape}
\begin{figure}[!ht]
\begin{center}
  \scalebox{0.7}{\includegraphics*{figs/illustration\D methods.png}}
\end{center}
\caption{Schematic of geostatistical methods used to estimate potential snow crab habitat area.}
\label{fig:method}
\end{figure}
\end{landscape}

\clearpage \newpage

%21
\begin{figure}[h]
\begin{center}
  \scalebox{0.6}{\includegraphics*{R/variograms/R0\D mass\D 2006.png}}
  \scalebox{0.6}{\includegraphics*{R/variograms/totno\D female\D berried\D 2006.png}}
\end{center}
\caption{Example variograms of fishable biomass in 2006 (top) and berried females (bottom) used to constrain the spatial interpolation of snow crab abundance estimates via universal kriging with external drift. Variogram form is more erratic with less abundant categories of crab.}
\label{fig:variogram}
\end{figure}


%22
\newpage
\begin{figure}[ht]
\begin{center}
\begin{tabular}{ccc}
  \begin{minipage}{2in}
    \scalebox{0.75}{\includegraphics*{R/logbook.1min.crab/annual/effort/effort\D 1996.png}}
  \end{minipage}
  \begin{minipage}{2in}
    \scalebox{0.75}{\includegraphics*{R/logbook.1min.crab/annual/effort/effort\D 1997.png}}
  \end{minipage}
  \begin{minipage}{2in}
    \scalebox{0.75}{\includegraphics*{R/logbook.1min.crab/annual/effort/effort\D 1998.png}}
  \end{minipage}
\end{tabular}

\begin{tabular}{ccc}
  \begin{minipage}{2in}
    \scalebox{0.75}{\includegraphics*{R/logbook.1min.crab/annual/effort/effort\D 1999.png}}
  \end{minipage}
  \begin{minipage}{2in}
    \scalebox{0.75}{\includegraphics*{R/logbook.1min.crab/annual/effort/effort\D 2000.png}}
  \end{minipage}
  \begin{minipage}{2in}
    \scalebox{0.75}{\includegraphics*{R/logbook.1min.crab/annual/effort/effort\D 2001.png}}
  \end{minipage}
\end{tabular}

\begin{tabular}{ccc}
  \begin{minipage}{2in}
    \scalebox{0.75}{\includegraphics*{R/logbook.1min.crab/annual/effort/effort\D 2002.png}}
  \end{minipage}
  \begin{minipage}{2in}
    \scalebox{0.75}{\includegraphics*{R/logbook.1min.crab/annual/effort/effort\D 2003.png}}
  \end{minipage}
  \begin{minipage}{2in}
    \scalebox{0.75}{\includegraphics*{R/logbook.1min.crab/annual/effort/effort\D 2004.png}}
  \end{minipage}
\end{tabular}

\begin{tabular}{ccc}
  \begin{minipage}{2in}
    \scalebox{0.75}{\includegraphics*{R/logbook.1min.crab/annual/effort/effort\D 2005.png}}
  \end{minipage}
  \begin{minipage}{2in}
    \scalebox{0.75}{\includegraphics*{R/logbook.1min.crab/annual/effort/effort\D 2006.png}}
  \end{minipage}
\end{tabular}

\end{center}
\caption{Fishing effort (number of trap hauls) from fisheries logbook data for the years 1996 to present. Note the increase in effort offshore and reduction inshore in S-ENS. No visible changes are evident in N-ENS.}
\label{maps.effort}
\end{figure}



%23
\newpage
\begin{figure}[ht]
\begin{center}
\begin{tabular}{ccc}
  \begin{minipage}{2in}
    \scalebox{0.75}{\includegraphics*{R/logbook.1min.crab/annual/landings/landings\D 1996.png}}
  \end{minipage}
  \begin{minipage}{2in}
    \scalebox{0.75}{\includegraphics*{R/logbook.1min.crab/annual/landings/landings\D 1997.png}}
  \end{minipage}
  \begin{minipage}{2in}
    \scalebox{0.75}{\includegraphics*{R/logbook.1min.crab/annual/landings/landings\D 1998.png}}
  \end{minipage}
\end{tabular}

\begin{tabular}{ccc}
  \begin{minipage}{2in}
    \scalebox{0.75}{\includegraphics*{R/logbook.1min.crab/annual/landings/landings\D 1999.png}}
  \end{minipage}
  \begin{minipage}{2in}
    \scalebox{0.75}{\includegraphics*{R/logbook.1min.crab/annual/landings/landings\D 2000.png}}
  \end{minipage}
  \begin{minipage}{2in}
    \scalebox{0.75}{\includegraphics*{R/logbook.1min.crab/annual/landings/landings\D 2001.png}}
  \end{minipage}
\end{tabular}

\begin{tabular}{ccc}
  \begin{minipage}{2in}
    \scalebox{0.75}{\includegraphics*{R/logbook.1min.crab/annual/landings/landings\D 2002.png}}
  \end{minipage}
  \begin{minipage}{2in}
    \scalebox{0.75}{\includegraphics*{R/logbook.1min.crab/annual/landings/landings\D 2003.png}}
  \end{minipage}
  \begin{minipage}{2in}
    \scalebox{0.75}{\includegraphics*{R/logbook.1min.crab/annual/landings/landings\D 2004.png}}
  \end{minipage}
\end{tabular}

\begin{tabular}{ccc}
  \begin{minipage}{2in}
    \scalebox{0.75}{\includegraphics*{R/logbook.1min.crab/annual/landings/landings\D 2005.png}}
  \end{minipage}
  \begin{minipage}{2in}
    \scalebox{0.75}{\includegraphics*{R/logbook.1min.crab/annual/landings/landings\D 2006.png}}
  \end{minipage}

\end{tabular}


\end{center}
\caption{Crab landings (kg) from fisheries logbook data for the years 1996 to the present. Note the increase in landings offshore and reduction inshore for S-ENS. No visible changes are evident in N-ENS.}
\label{maps.landings}
\end{figure}



\newpage
%24
\begin{figure}[ht]
\begin{center}
\begin{tabular}{ccc}
  \begin{minipage}{2in}
    \scalebox{0.75}{\includegraphics*{R/logbook.1min.crab/annual/cpue/cpue\D 1996.png}}
  \end{minipage}
  \begin{minipage}{2in}
    \scalebox{0.75}{\includegraphics*{R/logbook.1min.crab/annual/cpue/cpue\D 1997.png}}
  \end{minipage}
  \begin{minipage}{2in}
    \scalebox{0.75}{\includegraphics*{R/logbook.1min.crab/annual/cpue/cpue\D 1998.png}}
  \end{minipage}
\end{tabular}

\begin{tabular}{ccc}
  \begin{minipage}{2in}
    \scalebox{0.75}{\includegraphics*{R/logbook.1min.crab/annual/cpue/cpue\D 1999.png}}
  \end{minipage}
  \begin{minipage}{2in}
    \scalebox{0.75}{\includegraphics*{R/logbook.1min.crab/annual/cpue/cpue\D 2000.png}}
  \end{minipage}
  \begin{minipage}{2in}
    \scalebox{0.75}{\includegraphics*{R/logbook.1min.crab/annual/cpue/cpue\D 2001.png}}
  \end{minipage}
\end{tabular}

\begin{tabular}{ccc}
  \begin{minipage}{2in}
    \scalebox{0.75}{\includegraphics*{R/logbook.1min.crab/annual/cpue/cpue\D 2002.png}}
  \end{minipage}
  \begin{minipage}{2in}
    \scalebox{0.75}{\includegraphics*{R/logbook.1min.crab/annual/cpue/cpue\D 2003.png}}
  \end{minipage}
  \begin{minipage}{2in}
    \scalebox{0.75}{\includegraphics*{R/logbook.1min.crab/annual/cpue/cpue\D 2004.png}}
\end{minipage}
\end{tabular}

\begin{tabular}{ccc}
  \begin{minipage}{2in}
    \scalebox{0.75}{\includegraphics*{R/logbook.1min.crab/annual/cpue/cpue\D 2005.png}}
  \end{minipage}
  \begin{minipage}{2in}
    \scalebox{0.75}{\includegraphics*{R/logbook.1min.crab/annual/cpue/cpue\D 2006.png}}
  \end{minipage}
\end{tabular}

\end{center}
\caption{Catch rates (kg/trap) from logbook data for the years 1996 to the present. Note the expansion of fisheries activity to more offshore locations with time. TACs were raised dramatically in 2000. Since that time, large decreases of catch rates in the inshore areas have become evident, indicating strong depletion. The movement to more offshore areas (S-ENS) has offset this lowering of catch rates, where previously unexploited areas became more fully exploited. The temporal increases in crude catch rates of S-ENS are therefore due to the spatial expansion of the targeted areas and the fishers learning to find newer fishing grounds.}
\label{maps.cpue}
\end{figure}





%25
\begin{figure}[ht]
\begin{center}

\begin{tabular}{cc}
  \begin{minipage}{2in}
  \rotatebox{0}{\scalebox{0.065}{\includegraphics*{R/observer/size\D freqcfanorth2004.png}}}
  \end{minipage}
  \begin{minipage}{2in}
  \rotatebox{0}{\scalebox{0.065}{\includegraphics*{R/observer/size\D freqcfanorth2005.png}}}
  \end{minipage}
  \begin{minipage}{2in}
  \rotatebox{0}{\scalebox{0.065}{\includegraphics*{R/observer/size\D freqcfanorth2006.png}}}
  \end{minipage}
\end{tabular}

\begin{tabular}{cc}
\begin{minipage}{2in}
\rotatebox{0}{\scalebox{0.065}{\includegraphics*{R/observer/size\D freqcfasouth2004.png}}}
\end{minipage}
\begin{minipage}{2in}
\rotatebox{0}{\scalebox{0.065}{\includegraphics*{R/observer/size\D freqcfasouth2005.png}}}
\end{minipage}
\begin{minipage}{2in}
\rotatebox{0}{\scalebox{0.065}{\includegraphics*{R/observer/size\D freqcfasouth2006.png}}}
\end{minipage}
\end{tabular}

\begin{tabular}{cc}
\begin{minipage}{2in}
\rotatebox{0}{\scalebox{0.065}{\includegraphics*{R/observer/size\D freqcfa4x2004.png}}}
\end{minipage}
\begin{minipage}{2in}
\rotatebox{0}{\scalebox{0.065}{\includegraphics*{R/observer/size\D freqcfa4x2005.png}}}
\end{minipage}
\begin{minipage}{2in}
\rotatebox{0}{\scalebox{0.065}{\includegraphics*{R/observer/size\D freqcfa4x2006.png}}}
\end{minipage}
\end{tabular}

\end{center}
\caption{Size frequency distribution of at-sea-observer monitored snow crab broken down by carapace condition.}
\label{observer.carapacecondition}
\end{figure}



%26
\newpage
\begin{figure}[h]
\begin{center}
\begin{minipage}{5in}
  \scalebox{0.65}{\includegraphics*{figs/softshell\D 2005.png}}
\end{minipage}
\begin{minipage}{5in}
  \scalebox{0.65}{\includegraphics*{figs/softshell\D 2006.png}}  
\end{minipage}
\end{center}
\caption{Location of soft-shell snow crab occurence in the commerical fishery in 2005 (top) and 2006 (bottom).}
\label{map.softshell}
\end{figure}


%27
\newpage
\begin{figure}[h]
\begin{center}
\begin{minipage}{6.5in}
  \scalebox{0.7}{\includegraphics*{figs/softshell\D catchrates\D 2006.jpg}}
\end{minipage}
\end{center}
\caption{The relationship between the percentage soft-shell in the observed commerical landings as a function of the catch rates in at-sea-observed catches. Higher relative numbers of soft-shelled crab are expected in depleted areas as soft-shelled crab generally avoid hard-shelled males. However, high proportions of soft-shelled crab are found even in areas with high catch rates indicating that there is a large potential for damaging the fishable biomass if soft-shell catches are not carefully managed.}
\label{softshell.cpue}
\end{figure}




%28
\clearpage \newpage
\begin{figure}[ht]
\begin{center}
  \scalebox{0.18}{\includegraphics*{size.freq/male.png}}
\end{center}
\caption{Size-frequency histograms of carapace width of male snow crabs. Note the increasing numbers of juvenile crab, 1 to 3 years from entering morphometrically mature size classes. Due to the expansion of the survey from core areas to the full extent of the snow crab grounds, the areal densities of crab in S-ENS and CFA 4X are not directly comparable between all years. For N-ENS, however, the relative heights are comparable between all years. }
\label{hist.male.cw}
\end{figure}


%27a
\begin{figure}[ht]
\begin{center}
  \scalebox{1.5}{\includegraphics*{R/maturity/sex1/2min\D crab/globalaverage/cw50/cw50\D 1000}}
  \scalebox{1.5}{\includegraphics*{R/maturity/sex2/2min\D crab/globalaverage/cw50/cw50\D 1000}}
\end{center}
\caption{The spatial variations in size (CW; mm) at 50\% maturity of male (top) and female (bottom) snow crab on the Scotian Shelf. Large size at maturity for males is generally observed in ``core'' snow crab areas. Inshore regions generally show smaller size at 50\% maturity for males.  For females, size at 50\% maturity is heterogeneous.}
\label{map.maturity}
\end{figure}



%29
\clearpage \newpage
\begin{figure}[ht]
\begin{center}
  \scalebox{0.18}{\includegraphics*{size.freq/female.png}}
\end{center}
\caption{Size-frequency histograms of carapace width of female snow crabs. Note the increasing numbers of juvenile crab in recent years. The leading edge of the recruitment pulse has begun to enter morphometrically mature size classes. Due to the expansion of the survey from core areas to the full extent of the snow crab grounds, the areal densities of crab in S-ENS and CFA 4X are not directly comparable between all years. For N-ENS, however, the relative heights are comparable between all years.}
\label{hist.female.cw}
\end{figure}


%30
\begin{figure}[ht]
\begin{center}
  \scalebox{0.17}{\includegraphics*{ts/sexratio\D mat.png}}
\end{center}
\caption{Annual variations in the mean sex ratio (proportion female) for morphometrically mature crabs. One standard error bars are presented. The sex-ratio of mature crabs has been declining steadily since 1996 and increasing since historic lows in 2003.}
\label{ts.sex.ratio.mature}
\end{figure}


%31
\begin{figure}[ht]
\begin{center}

\begin{tabular}{cc}
\begin{minipage}{3in}
  \scalebox{1}{\includegraphics*{R/trawl.spline.2min.crab/annual/sexratio.mat/sexratio\D mat\D 1999}}
\end{minipage}
\begin{minipage}{3in}
  \scalebox{1}{\includegraphics*{R/trawl.spline.2min.crab/annual/sexratio.mat/sexratio\D mat\D 2000}}
\end{minipage}
\end{tabular}

\begin{tabular}{cc}
\begin{minipage}{3in}
  \scalebox{1}{\includegraphics*{R/trawl.spline.2min.crab/annual/sexratio.mat/sexratio\D mat\D 2001}}
\end{minipage}
\begin{minipage}{3in}
  \scalebox{1}{\includegraphics*{R/trawl.spline.2min.crab/annual/sexratio.mat/sexratio\D mat\D 2002}}
\end{minipage}
\end{tabular}

\begin{tabular}{cc}
\begin{minipage}{3in}
  \scalebox{1}{\includegraphics*{R/trawl.spline.2min.crab/annual/sexratio.mat/sexratio\D mat\D 2003}}
\end{minipage}
\begin{minipage}{3in}
  \scalebox{1}{\includegraphics*{R/trawl.spline.2min.crab/annual/sexratio.mat/sexratio\D mat\D 2004}}
\end{minipage}
\end{tabular}

\begin{tabular}{cc}
\begin{minipage}{3in}
  \scalebox{1}{\includegraphics*{R/trawl.spline.2min.crab/annual/sexratio.mat/sexratio\D mat\D 2005}}
\end{minipage}
\begin{minipage}{3in}
  \scalebox{1}{\includegraphics*{R/trawl.spline.2min.crab/annual/sexratio.mat/sexratio\D mat\D 2006}}
\end{minipage}
\end{tabular}

\end{center}
\caption{Morphometrically mature sex ratios. Since the early 2000s, most of the Scotian Shelf was uniformly male dominated. More balanced and spatially heterogeneous sex ratios have been observed since 2005.}
\label{map.sex.ratio.mature}
\end{figure}




%32

\begin{figure}[ht]
\begin{center}
  \scalebox{0.18}{\includegraphics*{ts/sexratio\D imm.png}}
\end{center}
\caption{Annual variations in the mean sex ratio (proportion female) for morphometrically immature crabs. One standard error bars are presented. The sex-ratio of immature crabs have been steadily returning to more balanced numbers. The initial decline of immature sex ratios may have been due to size-selective snow crab cannibalism and/or predation from groundfish and macro-invertebrates (other crabs, lobsters). }
\label{ts.sex.ratio.immature}
\end{figure}




%33
\begin{figure}[ht]
\begin{center}

\begin{tabular}{cc}
\begin{minipage}{3in}
  \scalebox{1}{\includegraphics*{R/trawl.spline.2min.crab/annual/sexratio.imm/sexratio\D imm\D 1999}}
\end{minipage}
\begin{minipage}{3in}
  \scalebox{1}{\includegraphics*{R/trawl.spline.2min.crab/annual/sexratio.imm/sexratio\D imm\D 2000}}
\end{minipage}
\end{tabular}

\begin{tabular}{cc}
\begin{minipage}{3in}
  \scalebox{1}{\includegraphics*{R/trawl.spline.2min.crab/annual/sexratio.imm/sexratio\D imm\D 2001}}
\end{minipage}
\begin{minipage}{3in}
  \scalebox{1}{\includegraphics*{R/trawl.spline.2min.crab/annual/sexratio.imm/sexratio\D imm\D 2002}}
\end{minipage}
\end{tabular}

\begin{tabular}{cc}
\begin{minipage}{3in}
  \scalebox{1}{\includegraphics*{R/trawl.spline.2min.crab/annual/sexratio.imm/sexratio\D imm\D 2003}}
\end{minipage}
\begin{minipage}{3in}
  \scalebox{1}{\includegraphics*{R/trawl.spline.2min.crab/annual/sexratio.imm/sexratio\D imm\D 2004}}
\end{minipage}
\end{tabular}

\begin{tabular}{cc}
\begin{minipage}{3in}
  \scalebox{1}{\includegraphics*{R/trawl.spline.2min.crab/annual/sexratio.imm/sexratio\D imm\D 2005}}
\end{minipage}
\begin{minipage}{3in}
  \scalebox{1}{\includegraphics*{R/trawl.spline.2min.crab/annual/sexratio.imm/sexratio\D imm\D 2006}}
\end{minipage}
\end{tabular}

\end{center}
\caption{Morphometrically immature sex ratios. Inshore areas are generally more balanced in sex ratios or more female dominated. Offshore areas were more male dominated. In the past, this was not always the case and sex ratios of immature crabs were more heterogeneous. Currently, a return to this more heterogeneous state has been observed.}
\label{map.sex.ratio.immature}
\end{figure}



% 34
\begin{landscape}
\begin{figure}[ht]
\begin{center}
\scalebox{0.09}{\includegraphics*{ts.kriged.results/totno\D female\D imm.png}}
\scalebox{0.09}{\includegraphics*{ts.kriged.results/totno\D female\D mat.png}}
\scalebox{0.09}{\includegraphics*{ts.kriged.results/totno\D female\D berried.png}}
\scalebox{0.09}{\includegraphics*{ts.kriged.results/totno\D female\D primiparous.png}}
\scalebox{0.09}{\includegraphics*{ts.kriged.results/totno\D female\D multiparous.png}}
\end{center}
\caption{Temporal variations in female snow crab abundance obtained from Kriged estimates. Error bars are 95\% confidence intervals about the estimated total number. The verticle line near 2002 indicates the period in which trawl surveys changed from a spring to an autumn sampling period.}
\label{ts.female.no}
\end{figure}
\end{landscape}



%35
\begin{figure}[ht]
\begin{center}

\begin{tabular}{cc}
\begin{minipage}{3in}
\scalebox{1}{\includegraphics*{R/predictions/kriged.estimates/totno\D female\D imm\D 1999.png}}
\end{minipage}
\begin{minipage}{3in}
\scalebox{1}{\includegraphics*{R/predictions/kriged.estimates/totno\D female\D imm\D 2000.png}}
\end{minipage}
\end{tabular}

\begin{tabular}{cc}
\begin{minipage}{3in}
\scalebox{1}{\includegraphics*{R/predictions/kriged.estimates/totno\D female\D imm\D 2001.png}}
\end{minipage}
\begin{minipage}{3in}
\scalebox{1}{\includegraphics*{R/predictions/kriged.estimates/totno\D female\D imm\D 2002.png}}
\end{minipage}
\end{tabular}

\begin{tabular}{cc}
\begin{minipage}{3in}
\scalebox{1}{\includegraphics*{R/predictions/kriged.estimates/totno\D female\D imm\D 2003.png}}
\end{minipage}
\begin{minipage}{3in}
\scalebox{1}{\includegraphics*{R/predictions/kriged.estimates/totno\D female\D imm\D 2004.png}}
\end{minipage}
\end{tabular}

\begin{tabular}{cc}
\begin{minipage}{3in}
\scalebox{1}{\includegraphics*{R/predictions/kriged.estimates/totno\D female\D imm\D 2005.png}}
\end{minipage}
\begin{minipage}{3in}
\scalebox{1}{\includegraphics*{R/predictions/kriged.estimates/totno\D female\D imm\D 2006.png}}
\end{minipage}
\end{tabular}


\end{center}
\caption{Numerical densities of the immature female snow crabs on the Scotian Shelf; $\log_{10}$~(number/\kmsq).}
\label{maps.fi.no}
\end{figure}



\begin{figure}[ht]
\begin{center}
\begin{tabular}{cc}
\begin{minipage}{3in}
\scalebox{1}{\includegraphics*{R/predictions/kriged.estimates/totno\D female\D mat\D 1999.png}}
\end{minipage}
\begin{minipage}{3in}
\scalebox{1}{\includegraphics*{R/predictions/kriged.estimates/totno\D female\D mat\D 2000.png}}
\end{minipage}
\end{tabular}

\begin{tabular}{cc}
\begin{minipage}{3in}
\scalebox{1}{\includegraphics*{R/predictions/kriged.estimates/totno\D female\D mat\D 2001.png}}
\end{minipage}
\begin{minipage}{3in}
\scalebox{1}{\includegraphics*{R/predictions/kriged.estimates/totno\D female\D mat\D 2002.png}}
\end{minipage}
\end{tabular}

\begin{tabular}{cc}
\begin{minipage}{3in}
\scalebox{1}{\includegraphics*{R/predictions/kriged.estimates/totno\D female\D mat\D 2003.png}}
\end{minipage}
\begin{minipage}{3in}
\scalebox{1}{\includegraphics*{R/predictions/kriged.estimates/totno\D female\D mat\D 2004.png}}
\end{minipage}
\end{tabular}

\begin{tabular}{cc}
\begin{minipage}{3in}
\scalebox{1}{\includegraphics*{R/predictions/kriged.estimates/totno\D female\D mat\D 2005.png}}
\end{minipage}
\begin{minipage}{3in}
\scalebox{1}{\includegraphics*{R/predictions/kriged.estimates/totno\D female\D mat\D 2006.png}}
\end{minipage}
\end{tabular}

\end{center}
\caption{Numerical densities of the mature female snow crabs on the Scotian Shelf; $\log_{10}$~(number/\kmsq).}
\label{maps.fm.no}
\end{figure}




\begin{figure}[ht]
\begin{center}

\begin{tabular}{cc}
\begin{minipage}{3in}
\scalebox{1}{\includegraphics*{R/predictions/kriged.estimates/totno\D female\D berried\D 1999.png}}
\end{minipage}
\begin{minipage}{3in}
\scalebox{1}{\includegraphics*{R/predictions/kriged.estimates/totno\D female\D berried\D 2000.png}}
\end{minipage}
\end{tabular}


\begin{tabular}{cc}
\begin{minipage}{3in}
\scalebox{1}{\includegraphics*{R/predictions/kriged.estimates/totno\D female\D berried\D 2001.png}}
\end{minipage}
\begin{minipage}{3in}
\scalebox{1}{\includegraphics*{R/predictions/kriged.estimates/totno\D female\D berried\D 2002.png}}
\end{minipage}
\end{tabular}

\begin{tabular}{cc}
\begin{minipage}{3in}
\scalebox{1}{\includegraphics*{R/predictions/kriged.estimates/totno\D female\D berried\D 2003.png}}
\end{minipage}
\begin{minipage}{3in}
\scalebox{1}{\includegraphics*{R/predictions/kriged.estimates/totno\D female\D berried\D 2004.png}}
\end{minipage}
\end{tabular}


\begin{tabular}{cc}
\begin{minipage}{3in}
\scalebox{1}{\includegraphics*{R/predictions/kriged.estimates/totno\D female\D berried\D 2005.png}}
\end{minipage}
\begin{minipage}{3in}
\scalebox{1}{\includegraphics*{R/predictions/kriged.estimates/totno\D female\D berried\D 2006.png}}
\end{minipage}
\end{tabular}


\end{center}
\caption{Numerical densities of the berried female snow crabs on the Scotian Shelf; $\log_{10}$~(number/\kmsq).}
\label{maps.fm.berried.no}
\end{figure}




\begin{landscape}
\begin{figure}[ht]
\begin{center}
\scalebox{0.12}{\includegraphics*{ts.kriged.results/imm\D cfanorth.png}}
\scalebox{0.12}{\includegraphics*{ts.kriged.results/imm\D cfasouth.png}}
\scalebox{0.12}{\includegraphics*{ts.kriged.results/imm\D cfa4x.png}}
\end{center}
\caption{Temporal variations in immature instars (9 to 12) of male snow crab abundance obtained from Kriged estimates. Error bars are 95\% confidence intervals about the estimated total number. The verticle line near 2002 indicates the period in which trawl surveys changed from a spring to an autumn sampling period.}
\label{ts.mi.no}
\end{figure}
\end{landscape}



\begin{landscape}
\begin{figure}[ht]
\begin{center}
\scalebox{0.12}{\includegraphics*{ts.kriged.results/imm\D skip\D cfanorth.png}}
\scalebox{0.12}{\includegraphics*{ts.kriged.results/imm\D skip\D cfasouth.png}}
\scalebox{0.12}{\includegraphics*{ts.kriged.results/imm\D skip\D cfa4x.png}}
\end{center}
\caption{Temporal variations in immature skip moulting instars (9 to 12) of male snow crab abundance in N-ENS obtained from Kriged estimates. Error bars are 95\% confidence intervals about the estimated total number. The verticle line near 2002 indicates the period in which trawl surveys changed from a spring to an autumn sampling period.}
\label{ts.mi.sm.no}
\end{figure}
\end{landscape}


\begin{landscape}
\begin{figure}[ht]
\begin{center}
\scalebox{0.12}{\includegraphics*{ts.kriged.results/ma\D CC1to2\D cfanorth.png}}
\scalebox{0.12}{\includegraphics*{ts.kriged.results/ma\D CC1to2\D cfasouth.png}}
\scalebox{0.12}{\includegraphics*{ts.kriged.results/ma\D CC1to2\D cfa4x.png}}
\end{center}
\caption{Temporal variations in mature CC1 and CC2 instars (9 to 13) of male snow crab abundance obtained from Kriged estimates. Error bars are 95\% confidence intervals about the estimated total number. The verticle line near 2002 indicates the period in which trawl surveys changed from a spring to an autumn sampling period.}
\label{ts.maCC12.no}
\end{figure}
\end{landscape}


\begin{landscape}
\begin{figure}[ht]
\begin{center}
\scalebox{0.12}{\includegraphics*{ts.kriged.results/ma\D CC3to4\D cfanorth.png}}
\scalebox{0.12}{\includegraphics*{ts.kriged.results/ma\D CC3to4\D cfasouth.png}}
\scalebox{0.12}{\includegraphics*{ts.kriged.results/ma\D CC3to4\D cfa4x.png}}
\end{center}
\caption{Temporal variations in mature CC3 and CC4 instars (9 to 13) of male snow crab abundance obtained from Kriged estimates. Error bars are 95\% confidence intervals about the estimated total number. The verticle line near 2002 indicates the period in which trawl surveys changed from a spring to an autumn sampling period.}
\label{ts.maCC34.no}
\end{figure}
\end{landscape}




\begin{landscape}
\begin{figure}[ht]
\begin{center}
\scalebox{0.12}{\includegraphics*{ts.kriged.results/ma\D CC5\D cfanorth.png}}
\scalebox{0.12}{\includegraphics*{ts.kriged.results/ma\D CC5\D cfasouth.png}}
\end{center}
\caption{Temporal variations in mature CC5 instars (9 to 13) of male snow crab abundance obtained from Kriged estimates. Error bars are 95\% confidence intervals about the estimated total number. The verticle line near 2002 indicates the period in which trawl surveys changed from a spring to an autumn sampling period.}
\label{ts.maCC5.no}
\end{figure}
\end{landscape}




\clearpage \newpage
\begin{figure}[ht]
\begin{center}
\scalebox{0.2}{\includegraphics*{ts.kriged.results/R0\D mass.png}}
\end{center}
\caption{Temporal variations in fishable biomass estimates. Error bars are 95\% confidence intervals about the estimated total biomass. The verticle line near 2002 indicates the period in which trawl surveys changed from a spring to an autumn sampling period.}
\label{ts.R0.mass}
\end{figure}



\clearpage \newpage
\begin{figure}[ht]
\begin{center}
\begin{tabular}{cc}
\begin{minipage}{3in}
\scalebox{1}{\includegraphics*{R/predictions/kriged.estimates/R0\D mass\D 1999.png}}
\end{minipage}
\begin{minipage}{3in}
\scalebox{1}{\includegraphics*{R/predictions/kriged.estimates/R0\D mass\D 2000.png}}
\end{minipage}
\end{tabular}

\begin{tabular}{cc}
\begin{minipage}{3in}
\scalebox{1}{\includegraphics*{R/predictions/kriged.estimates/R0\D mass\D 2001.png}}
\end{minipage}
\begin{minipage}{3in}
\scalebox{1}{\includegraphics*{R/predictions/kriged.estimates/R0\D mass\D 2002.png}}
\end{minipage}
\end{tabular}

\begin{tabular}{cc}
\begin{minipage}{3in}
\scalebox{1}{\includegraphics*{R/predictions/kriged.estimates/R0\D mass\D 2003.png}}
\end{minipage}
\begin{minipage}{3in}
\scalebox{1}{\includegraphics*{R/predictions/kriged.estimates/R0\D mass\D 2004.png}}
\end{minipage}
\end{tabular}

\begin{tabular}{cc}
\begin{minipage}{3in}
\scalebox{1}{\includegraphics*{R/predictions/kriged.estimates/R0\D mass\D 2005.png}}
\end{minipage}
\begin{minipage}{3in}
\scalebox{1}{\includegraphics*{R/predictions/kriged.estimates/R0\D mass\D 2006.png}}
\end{minipage}
\end{tabular}

\end{center}
\caption{Fishable biomass densities on the Scotian Shelf; $\log_{10}$~(t/\kmsq).}
\label{maps.R0.mass}
\end{figure}



\begin{figure}[ht]
\begin{center}
\scalebox{0.2}{\includegraphics*{ts.kriged.results/R1a\D no.png}}
\end{center}
\caption{Temporal variations in the expected recruitment into the fishable biomass. This includes only legal sized soft crab. As surveys are conducted in the autumn (since 2002/2003), the majority of recruitment into the fishable biomass has already occurred. This figure shows the additional recruitment expected that has not yet become part of the fishable biomass.  Error bars are 95\% confidence intervals about the estimated total biomass. The verticle line near 2002 indicates the period in which trawl surveys changed from a spring to an autumn sampling period.}
\label{ts.R1a.no}
\end{figure}



\clearpage \newpage

\begin{figure}[ht]
\begin{center}
\begin{tabular}{cc}
  \begin{minipage}{3in}
    \scalebox{1}{\includegraphics*{R/predictions/kriged.estimates/R1a\D no\D 1999.png}}
  \end{minipage}
  \begin{minipage}{3in}
    \scalebox{1}{\includegraphics*{R/predictions/kriged.estimates/R1a\D no\D 2000.png}}
  \end{minipage}
\end{tabular}

\begin{tabular}{cc}
  \begin{minipage}{3in}
    \scalebox{1}{\includegraphics*{R/predictions/kriged.estimates/R1a\D no\D 2001.png}}
  \end{minipage}
  \begin{minipage}{3in}
    \scalebox{1}{\includegraphics*{R/predictions/kriged.estimates/R1a\D no\D 2002.png}}
  \end{minipage}
\end{tabular}

\begin{tabular}{cc}
  \begin{minipage}{3in}
    \scalebox{1}{\includegraphics*{R/predictions/kriged.estimates/R1a\D no\D 2003.png}}
  \end{minipage}
  \begin{minipage}{3in}
    \scalebox{1}{\includegraphics*{R/predictions/kriged.estimates/R1a\D no\D 2004.png}}
 \end{minipage}
\end{tabular}

\begin{tabular}{cc}
  \begin{minipage}{3in}
    \scalebox{1}{\includegraphics*{R/predictions/kriged.estimates/R1a\D no\D 2005.png}}
  \end{minipage}
  \begin{minipage}{3in}
    \scalebox{1}{\includegraphics*{R/predictions/kriged.estimates/R1a\D no\D 2006.png}}
  \end{minipage}
\end{tabular}

\end{center}
\caption{Numerical densities of snow crab recruiting into the next year; $\log_{10}$~(number/\kmsq).}
\label{maps.R1a.no}
\end{figure}


\clearpage \newpage


\begin{figure}[ht]
\begin{center}
\rotatebox{0}{\scalebox{0.2}{\includegraphics*{ts.kriged.results/exploitation\D rate.png}}}
\end{center}
\caption{Temporal variations in the amount of biomass landed relative to biomass estimates.}
\label{relativeF}
\end{figure}

\clearpage \newpage


\begin{figure}[ht]
\begin{center}
\rotatebox{0}{\scalebox{0.25}{\includegraphics*{ts/projections\D cfasouth.png}}}
\end{center}
\caption{Temporal variations in fishable biomass relative to that of 2006 (pre-fishery; horizontal line) projected into the future based upon differing exploitation rates for S-ENS. Error bars are 95\% confidence intervals propagated by assuming all errors in parameter estimates to be independent.}
\label{projections.south}
\end{figure}


\clearpage \newpage

\begin{figure}[ht]
\begin{center}
\rotatebox{0}{\scalebox{0.25}{\includegraphics*{ts/projections\D cfanorth.png}}}
\end{center}
\caption{Temporal variations in fishable biomass relative to that of 2006 (pre-fishery; horizontal line) projected into the future based upon differing exploitation rates for N-ENS.  Error bars are 95\% confidence intervals propagated by assuming all errors in parameter estimates to be independent.}
\label{projections.north}
\end{figure}


\clearpage \newpage


\begin{figure}[ht]
\begin{center}
\rotatebox{0}{\scalebox{0.25}{\includegraphics*{ts/projections\D cfa4x.png}}}
\end{center}
\caption{Temporal variations in fishable biomass relative to that of 2006 (pre-fishery; horizontal line) projected into the future based upon differing exploitation rates for CFA-4X.  Error bars are 95\% confidence intervals propagated by assuming all errors in parameter estimates to be independent.}
\label{projections.4x}
\end{figure}







\clearpage \newpage


\appendix

%\renewcommand{\theequation}{A-\arabic{equation}} % redefine the command that creates the equation no.
%\setcounter{equation}{0}  % reset counter
%\section{APPENDIX}  % use *-form to suppress numbering


\clearpage
\newpage
\section{Glossary}


\textbf{Agonistic} -- Behavioural term relating to aggression, appeasement and avoidance behavior that occurs between members of the same species. Agonistic behavior is a much broader term than "aggression," which simply refers to behavior patterns which serve to intimidate or damage another.

\textbf{Anthropogenic} -- Resulting from the influence of human beings on nature.

\textbf{Benthic} -- Occurring on the ocean floor.

\textbf{Biomass} -- The abundance of living organisms measured in terms of it's weight, mass, volume or caloric energy.

\textbf{Brachyura} (Infraorder) -- Known as ``true crabs'' of which the snow crab is a member. Brachyurans are characterised by a body that is short, wide, and flat. The abdomen, is reduced from a strong swimming muscle (e.g., shrimp) to a simple flap covering reproductive appendages and carry eggs. The uropods, which along with the telson form the tailfan in other decapods, are totally absent. All five pairs of walking legs are generally large with the first pair being chelipeds. The antennae and antennules are greatly reduced and originate before the eyestalks.

\textbf{CC, Carapace Condition} -- The condition of the shell of a snow crab. Generally related to the age of the organism and the time since last moult. See Table~\ref{table.carapace.condition.description} for more details.

\begin{itemize}
\item  \textit{CC1} - Newly moulted crab. The top of carapace is light brown and shiny without surface growth of moss or barnacles. Shell is soft and claw is easily broken.
\item  \textit{CC2} - The top of carapace is light brown and less shiny with little to no surface growth of moss or barnacles. Shell is clean but hard.
\item  \textit{CC3} - The top of carapace is light brown and not shiny. Some growth of moss or barnacles. Shell is hard.
\item  \textit{CC4} - The top of carapace is brown and not shiny. Usually some surface growth of moss or barnacles. Shell is hard with small scars. Underneath is yellow brown.
\item  \textit{CC5} - Old crab. Carapace is dark brown with substantially mossy (``dirty'') surface. Decalcification (black spots) noticeable often at joints. Shell may be soft.
\end{itemize}

\textbf{CW, Carapace width} -- the distance across the carapace of a snow crab (millimetres)

\textbf{Chela} -- pincer-like claw of a crustacean or arachnid.

\textbf{CFA, Crab fishing area} -- Refers to an individual management area. On the Scotian Shelf they are from north to south: 20 to 24 and 4X.

\textbf{Commercial biomass} -- see Fishable biomass.

\textbf{CPUE, Catch per unit effort} -- The amount caught by a single fishing event: such as the weight or number of crab captured by a single trap haul.

\textbf{Density} -- The amount (biomass or number) of crab per unit area.

\textbf{Distribution, spatial} -- The geographic area in which an organism exists.

\textbf{Durometer} -- A calibrated instrument used to measure the hardness of an object (such as a crab shell), scaled from 0 (soft) to 100 (hard). A durometer reading of $\geq$~68 has been historically used to determine a hard shelled crab.

\textbf{Dynamic} -- Characterized by continuous change or time. Not fixed.

\textbf{Ecosystem} -- The whole of a system with all the interactions between parts (living and non-living).

\textbf{ENS} -- Eastern Nova Scotia (essentially NAFO statistical divisions 4VW). 

\textbf{ER, Exploitation rate} -- The ratio of biomass fished relative to their abundance. Historically, the GFC calculated ER(t) = Landings(t) / Fishable biomass (t-1), where t is time or year. The Fishable biomass was of the mature segment of the male population $\geq$~95~mm~CW, estimated from kriging. In this document, the exploitation rate is calculated as ER(t) = Landings(t) / ( Landings(t) + Fishable biomass(t) ). This change was made as the the time interval between the end of trawl surveys [Biomass(t-1)] and the beginning of fishing [Landings(t)] was up to 10 months. With the alternate method, this lag is $\sim$~2~months and so likely more accurate. 

\textbf{Extrapolate} -- To infer or estimate by extending or projecting known information.

\textbf{Fishable biomass, FB} -- The biomass of snow crab exploited by the commercial fishery: male, mature, $\geq$~95~mm~CW and hard shell condition (Carapace conditions 2 to 5). Note that Carapace condition 2 snow crab do not have optimal meat yields at the time of the fishery. While immature crab $\geq$~95~mm~CW is part of the biomass that can be legally fished, this component is voluntarily returned to allow greater growth.

\textbf{Fishing mortality, relative} -- see Exploitation rate.

\textbf{IBQ} -- Individual Boat Quota, the amount of snow crab allowed to be legally removed by an individual fisher in a given area over a given period of time.

\textbf{Instar} -- A stage of an organism between moults.

\textbf{Interpolation} -- The method of determining unknown values through the use of surrounding known values.

\textbf{Kriging} -- A method of interpolation for obtaining statistically unbiased estimates of intrinsic variables (i.e., snow crab biomass density) from a set of neighbouring points with known values, constrained by the relative change in variability of the data as a function of distance.

\textbf{Larvae} -- The early, immature form of any animal before the assumption of the mature shape.

\textbf{Metabolic costs} -- The amount of energy dispensed by an organism in the process of living (heat, organic compounds, faeces, urea/uric acid, etc.).

\textbf{Metabolic gains} -- The amount of energy gained through the intake of food or other energy sources.

\textbf{Morphometric maturity} -- Maturity status determined from measurements of body shape and size. Male snow crab claw height increases very rapidly in the adult stage (terminal moult), whereas females' abdominal width increases with maturity. While morphometric maturity generally co-incides with physiological maturity, morphometrically immature males are known to be able to fertilise females.

\textbf{Moult} -- The act of growing through the shedding of an organism's current shell.

\textbf{Multiparous} -- Females bearing eggs resulting from their second or third breeding event (mating).

\textbf{Numerical density} -- The number of snow crab in a given surface area.

\textbf{Pelagic} -- Occurring in the water column (not on bottom).

\textbf{Pencil-clawed crab} -- Immature crab that are legally exploitable ($\geq$~95~mm~CW) but not yet terminally moulted. The final growth increment is estimated to increase the body weight by $\sim$~250\%.

\textbf{Physiological maturity} -- Biologically (functionally) able to reproduce.

\textbf{Primiparous} -- Females bearing eggs resulting from their first breeding event (mating).

\textbf{Recruitment} -- Snow crab that will enter the fishable biomass in the next fishing season, designates as ``R-1''.

\textbf{Sexual dimorphism} -- When shape and/or size differences exists between sexes of a species.

\textbf{Soft shell} -- Carapace condition in which the shell produces a durometer reading of less than 68 durometer units.

\textbf{Spatial} -- Relating to space (such as a given geographic region such as the Scotian Shelf).

\textbf{Substrate} -- bottom type on which an animal exists (rocks, boulders, mud, sand, etc.).

\textbf{TAC} -- Total Allowable Catch, the amount of snow crab allowed to be legally removed in a given area over a given period of time.

\textbf{Temporal} -- Relating to time (such as a given period of time).

\textbf{Terminal moult} -- Snow crab moulted for a final time once mature. The size of these crab will not increase further.

\textbf{Variogram} -- The manner in which the variability of data changes with distance from a given location. Empirical variograms depict the data-derived variation as a function of distance. Theoretical/modelled variograms are fitted curves which are ultimately used by the kriging methodology.



\end{document}

