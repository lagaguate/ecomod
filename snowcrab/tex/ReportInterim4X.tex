
\documentclass[paper=a4, fontsize=11pt]{article} 
\usepackage[T1]{fontenc} 
\usepackage[english]{babel} 
\usepackage{amsmath,amsfonts,amsthm} 
\usepackage{sectsty} 
\usepackage{graphicx}
\usepackage[hidelinks]{hyperref}
\usepackage{fancyhdr}
\usepackage{pdfpages}
\usepackage{subfig}
\setcounter{secnumdepth}{0}
\allsectionsfont{\centering \normalfont\scshape} 
\pagestyle{fancyplain} 
\fancyhead{} 
\fancyfoot[L]{} 
\fancyfoot[C]{} 
\fancyfoot[R]{\thepage} 
\renewcommand{\headrulewidth}{0pt} 
\renewcommand{\footrulewidth}{0pt} 
\setlength{\headheight}{13.6pt} % Customize the height of the header
\newcommand{\D}{.}

\numberwithin{equation}{section} 
\numberwithin{figure}{section} 
\numberwithin{table}{section} 

\setlength\parindent{0pt} 

%----------------------------------------------------------------------------------------
%	TITLE SECTION
%----------------------------------------------------------------------------------------

\newcommand{\horrule}[1]{\rule{\linewidth}{#1}} 

\title{
\normalfont \normalsize 
\textsc{DFO - Population Ecology} \\ [25pt] 
\horrule{0.5pt} \\[0.4cm] 
\huge CFA 4X Early Season Interim Report\\ 
\horrule{2pt} \\[0.5cm] 
}
\author{PED Snow Crab Team} 

\date{\normalsize\today} 

\begin{document}

\maketitle 

%----------------------------------------------------------------------------------------
\section{Methods}

The 2015 snow crab trawl survey was conducted in 4X during 23-26 September 2015. The survey results presented here are to be considered preliminary but can be used in the context of relative trends and spatial patterns. Results shown from the trawl survey reflect the swept area adjusted number per tow of commercial snow crab (\textgreater 95mm male). Logbook information were used to assess the number of active vessels per month, the commercial catch rate index (total weight of commercial animals per trap haul), the spatial variability in catch rate and the total monthly landings. Observer information was obtained to examine the size of the animals retained in the commercial fishery.


\section{Results}
\subsection{Trawl Survey}
The snow crab trawl survey found commerical snow crab only in the eastern most stations (Figure 0.1). Even in years with extremely low trawl survey index for this crab fishing area (2012 and 2013), the commerical animals were found through out the surveyed area (Figure 0.1).  Note that in 2014, not all survey stations were completed and the spatial extent of the commerical snow crab was not estimated. The mean number of commercial snow crab per tow in 2015 was among the lowest since the survey reached its current spatial extent in CFA4X. The number of commercial animals per tow in 2014 was likely an overestimate as the full survey was not completed in CFA4X. 

\subsection{Logbooks}
The logbook data suggests that commercial catch rates for the first 12 days of the fishing season were the highest they have been since 2011, although they were below those of 2006, 2010 and 2011 (Figure 0.3). Four vessels were active during the month of Novemeber, one more than in 2014, however still below the high of 7 or 8 vessels in 2012 or 2011 resptively (Figure 0.4). The total landings for November 2015 reported in logbooks is less than that landed in 2009-2011, but higher than that landed in November for each of the past three fishing seasons (Figure 0.5). Smiilar to previous seasons the spatial extent of the fishing for 2015 has been limited to the eastern portion of CFA4X, around Sambro (Figure 0.6). In other years, fishing progressed further west throughout the season toward a secondary, area known as the Eastern Holes. It is unknown if the biomass of snow crab extends beyond this small pocket beyond Sambro (Figure 0.7).\\ 

\section{Recommendation}
 Currently the only information on the spatial extent of commercial snow crab comes from the snow crab survey. This data source suggests that there is no commercial crab west of Sambro. In 2010 and 2011 when landings remained high for most of the season (Figure 0.4), snow crab were captured during survey throughout the CFA (Figure 0.1) and fishing extended to western portion of the CFA (Figure 0.6). There is no scientific basis to suggest catch rates will remain at their current level throughout the entire fishing season.



\begin{figure}
   \centering
  	\subfloat{\includegraphics[width=0.55\textwidth]{R0\D no\D 2010.pdf}}
  	\subfloat{\includegraphics[width=0.55\textwidth]{R0\D no\D 2011.pdf}}\\
  	\subfloat{\includegraphics[width=0.55\textwidth]{R0\D no\D 2012.pdf}}
  	\subfloat{\includegraphics[width=0.55\textwidth]{R0\D no\D 2013.pdf}}\\
  	\subfloat{\includegraphics[width=0.55\textwidth]{R0\D no\D 2014.pdf}}
  	\subfloat{\includegraphics[width=0.55\textwidth]{R0\D no\D 2015.pdf}}\\
	\caption{Survey densities of commercial male biomasses from trawl survey.}

\end{figure}
\clearpage

\begin{figure}
\centering
	\includegraphics[width=\textwidth]{R0\D mass\D combined.png}
	\caption{Commercial snow crab geometric mean number per tow with standard deviations from the annual snow crab survey.}

\end{figure}


\begin{figure}
	\centering
	\includegraphics[width=\textwidth]{EARLY\D SEASON\D CATCHES.png}
	\caption{Commercial catch rates (kg per trap haul) from the first 12 fishing days of the snow crab season in 4X. The lines around each point represent the standard deviation about the mean.}
\end{figure}

\begin{figure}
	\centering
	\includegraphics[width=\textwidth]{active\D vessels\D by\D month.png}
	\caption{The number of active fishing vessels by month of season in CFA4X for each of 2011-2015. The purple dot represents 2015 as there are no other data for 2015.}
\end{figure}

\begin{figure}
	\centering
	\includegraphics[width=\textwidth]{monthly\D landings.png}
	\caption{The total landings by month in CFA4X obtained from logbook information across multiple years. The purple dot represents 2015 as no other data is available for this fishing season.}
\end{figure}

\begin{figure}
   \centering
  	\subfloat{\includegraphics[width=0.55\textwidth]{logbook\D locations\D cpue\D early\D 2010\D .png}}
  	\subfloat{\includegraphics[width=0.55\textwidth]{logbook\D locations\D cpue\D early\D 2011\D .png}}\\
  	\subfloat{\includegraphics[width=0.55\textwidth]{logbook\D locations\D cpue\D early\D 2012\D .png}}
  	\subfloat{\includegraphics[width=0.55\textwidth]{logbook\D locations\D cpue\D early\D 2013\D .png}}\\
  	\subfloat{\includegraphics[width=0.55\textwidth]{logbook\D locations\D cpue\D early\D 2014\D .png}}
  	\subfloat{\includegraphics[width=0.55\textwidth]{logbook\D locations\D cpue\D early\D 2015\D .png}}

	\caption{Early season (first 12 days) gridded fishing locations and catch rates (geometric mean catchs with shading from blue-green for low catches to orange-red for high catch rates) from logbook information}

\end{figure}
\clearpage


\begin{figure}
   \centering
  	\subfloat{\includegraphics[width=0.55\textwidth]{logbook\D locations\D cpue\D all\D 2010\D .png}}
  	\subfloat{\includegraphics[width=0.55\textwidth]{logbook\D locations\D cpue\D all\D 2011\D .png}}\\
  	\subfloat{\includegraphics[width=0.55\textwidth]{logbook\D locations\D cpue\D all\D 2012\D .png}}
  	\subfloat{\includegraphics[width=0.55\textwidth]{logbook\D locations\D cpue\D all\D 2013\D .png}}\\
  	\subfloat{\includegraphics[width=0.55\textwidth]{logbook\D locations\D cpue\D all\D 2014\D .png}}
  	\subfloat{\includegraphics[width=0.55\textwidth]{logbook\D locations\D cpue\D all\D 2015\D .png}}

	\caption{Full fishing season gridded fishing locations and catch rates (geometric mean catchs with shading from blue-green for low catches to orange-red for high catch rates) from logbook information.}

\end{figure}
\clearpage


\end{document}