\documentclass{beamer}

\usepackage{default}

\mode<presentation> {
  % \usetheme{Hannover}
  % \usetheme{Boadilla}
  % \usecolortheme{dolphin}
  % \usecolortheme{seagull}
  \beamertemplatenavigationsymbolsempty % turn off navigation
  \hypersetup{pdfstartview={Fit}} % fits the presentation to the window when first displayed
}


\usepackage{fourier}
\usepackage[english]{babel}															% English language/hyphenation
\usepackage[protrusion=true,expansion=true]{microtype}				% Better typography

\usepackage[toc,page]{appendix}
\usepackage[utf8]{inputenc}
\usepackage{csquotes}
\usepackage{hyperref}

% linking options
%%% Equation and float numbering
\usepackage{amsmath,amsfonts,amsthm}										% Math packages
\numberwithin{equation}{section}		% Equationnumbering: section.eq#
\numberwithin{figure}{section}		% Figurenumbering: section.fig#
\numberwithin{table}{section}				% Tablenumbering: section.tab#

\usepackage{graphicx} % Allows including images
\usepackage{graphics}
\usepackage{booktabs} % Allows the use of \toprule, \midrule and \bottomrule in tables

\newcommand{\ecomod}{\string~/ecomod_data/}   %  \string~ is a representation of the home directory 
\newcommand{\mpa}{\ecomod/mpa/}       %  \string~ is a representation of the home directory 
\newcommand{\sab}{\ecomod/mpa/sab/}   %  \string~ is a representation of the home directory 
\newcommand{\analysis}{\ecomod/mpa/analysis/}   %  \string~ is a representation of the 


%%% Title page



\title[SAB/MPA Framework]{Assessment Framework for the St Anns Bank Marine Protected Area} 

\author[Choi, et al.]{
  Jae S. Choi, Angelia S.M. Vanderlaan, Gordana Lazin, \\
  Mike McMahon,  Ben Zisserson, Brent Cameron, \\ 
  Jenna Munden 
}

\institute[DFO] {
  Population Ecology Division \\ 
  Fisheries and Oceans Canada \\ 
  Bedford Institute of Oceanography \\
}

\date{\textsc{\today}} % Date, can be changed to a custom date

\begin{document}


% ----------------------------------------------------------------

\begin{frame}
  \titlepage % Print the title page as the first slide
\end{frame}

% ----------------------------------------------------------------

\begin{frame}
  \frametitle{Overview} % Table of contents slide, comment this block out to remove it
  \tableofcontents % Throughout your presentation, if you choose to use \section{} and \subsection{} commands, these will automatically be printed on this slide as an overview of your presentation
\end{frame}

% ----------------------------------------------------------------
\section{Purpose}

\begin{frame}
  \frametitle{Terms of reference}
  \vspace*{-1.0cm}
    
    \begin{itemize}
      \item Health of the Oceans (HOTO, 2007-2012)
      \item National Conservation Plan (NCP, 2014-2019)
      \item Develop a monitoring approach for Marine Protected Areas (MPAs) 
      \item Assess their effectiveness in meeting their objectives. 
    \end{itemize}
    
  
\end{frame}

% ----------------------------------------------------------------

\begin{frame}
  \frametitle{MPA: To conserve and protect}

  \begin{itemize}
    \item Commercial and non-commercial fishery resources, including marine mammals, and their \textbf{habitats}; 
    \item Endangered or threatened marine species, and their \textbf{habitats}; 
    \item Unique \textbf{habitats}; 
    \item Marine areas of high \textbf{biodiversity} or biological \textbf{productivity};  
    \item Any other marine resource or habitat as is necessary to fulfill the mandate of the Minister.
\end{itemize}

\medskip

\textbf{Bottom line}: productivity, biodiversity, habitat \& species of interest

\end{frame}
  

% ----------------------------------------------------------------

\begin{frame}
  \frametitle{St Anns Bank (SAB)}
    \begin{columns}[c]
      \begin{column}{.6\textwidth}
        \begin{figure}[h]
          \centering
          %\includegraphics[width=0.6\textwidth]{\analysis/maps/mpa_closeup.pdf}
          \includegraphics[width=1.0\textwidth]{\sab SAB_MPA.pdf}
          %\caption{Bathymetic (100 m resolution) chart of the  St. Anns Bank area with the proposed St Anns Bank Marine Protected Area (thick maroon polygon) and limited fishing zones (maroon lined polygons). See Figure\ref{fig:SAB} for geographic location in a larger map.}
          \label{fig:SABCloseup}
        \end{figure} 
      \end{column}
      \begin{column}{.4\textwidth}
        Previous discussions of SAB:
        \begin{itemize}
          \item DFO 2012
          \item Kenchington 2013
          \item Ford \& Serdynska 2013
        \end{itemize}
      \end{column}
    \end{columns}
\end{frame}



% ----------------------------------------------------------------

\begin{frame}
  \frametitle{Objectives}
  \begin{block}{}
    \begin{itemize}
    \item	monitor and assess the status/effectiveness of an MPA;
    \item identify data gaps and sources of uncertainty. 
    \end{itemize}
  \end{block}
  
  \begin{block}{}
\begin{flushright}
\emph{... But how?}
\end{flushright}
  \end{block}
  
\end{frame}



% ----------------------------------------------------------------

\begin{frame}
  \frametitle{A few "minor" issues...}
  \begin{block}{SAB is:}
    \begin{itemize}
      \item a large ecosystem and as such complex, operating at various space, time and organizational scales;
      \item connected in various ways to the surrounding environment and so cannot be treated as an isolated system;
      \item measures of system components of interest, namely, productivity, biodiversity, habitat and species of interest, are ambiguous and imperfect at best, and usually non-existent or poor in information quality/quantity.
    \end{itemize}
  \end{block}
\end{frame}


% ----------------------------------------------------------------

\begin{frame}[shrink]
  \frametitle{Design principles}
  \begin{block}{}
  \begin{itemize}
    \item Expansive area of interest (whole shelf sea)
    \item Leverage data from \textbf{on-going sampling} programs to smoothly transition into a routine monitoring approach into the future.
    \begin{itemize}
      \item Data that are clearly and directly related to MPA objectives/priorities
      \item	Sufficient and regular \textbf{spatial} coverage ($>$ 100 sampling locations) inside MPA and throughout the study area.
      \item Sufficient and regular \textbf{temporal} coverage ($\sim$ annual, $>$~10 years) inside MPA and throughout the study area
      \item \textbf{Informative} -- high data quality that is in some manner related to productivity, biodiversity, habitat and species of interest.
    \end{itemize}
    \item Methods that are transparent, collaboratively developed and easily transferable to other regions

  \end{itemize}
  \end{block}
\end{frame}

% ----------------------------------------------------------------

\begin{frame}[shrink]
  \frametitle{Data sources being considered}
\begin{itemize}
	\item AZMP/chlorophyll-a and nutrients: BioChem bottle data 
  \item AZMP/Zooplankton: BioChem database 
  \item Remote Sensing Data: ocean colour and SST (Remote Sensing Group)
  \item Temperature records: from various sources, especially, Groundfish, Snow crab and AZMP surveys
  \item Salinity (Groundfish surveys/AZMP, BioChem)
  \item Oxygen and pH (once the data have been reloaded; Groundfish surveys/AZMP, BioChem)
  \item Bathymetry (CHS, Groundfish survey, Snow crab survey )
  \item Groundfish: Research Vessel Surveys focus upon demersal fish species, since $\sim$ 2000, upon invertebrates as well
  \item Snow crab survey, focus upon benthic invertebrates
  \item Clam survey data in Banquereau and Western Banks (though it does not pass the temporal coverage conditions, it offers very high resolution multispecies data on the banks)
  \item Logbook records of catch and effort (MARFIS/ZIFF)
  \item AIS tracks -- Radio-based Automatic Identification System
  \item VMS potentially -- Satellite-based Vessel Monitoring System
\end{itemize}
    
\end{frame}


% ----------------------------------------------------------------

\begin{frame}
  \frametitle{BioChem Discrete Bottle Data: Chlorophyll-a and Nutrients}
\begin{itemize}
  \item Relevance: productivity, biodiversity, habitat and species of interest (in relative order)
  \item Sampling: AZMP surveys, Groundfish surveys, pelagic net tows and water profiles
  \item Spatial coverage: variable no. stations, 143,499 records, 829 missions
  \item Temporal coverage: 1955 to present, annual surveys
  \item Source code: \url{https://github.com/jae0/ecomod/biochem/src/biochem.r}
\end{itemize}
\end{frame}



% ----------------------------------------------------------------
\section{Data}
\begin{frame}
  \frametitle{}
      \begin{figure}
        \centering
        \includegraphics[width=0.7\textwidth]{\sab 2.png}
        \caption{Number of chlorophyll and nutrient profiles extracted from the BioChem database for each year since 1955.}
        \label{fig:ChloroMap}
      \end{figure}
      \begin{figure}
        \centering
        \includegraphics[width=0.5\textwidth]{\sab 3.png}
        \caption{Number of chlorophyll and nutrient profiles extracted from the BioChem database for the time period 1955-2014, grouped monthly.}
        \label{fig:ChloroFreq}
      \end{figure}
\end{frame}


% ----------------------------------------------------------------

\begin{frame}[shrink]
  \frametitle{}
  \begin{figure}
    \centering
    \includegraphics[width=0.8\textwidth]{\sab 4.png}
    \caption{Monthly spatial distribution of discrete bottle data for the time period 1955-2014.}
    \label{fig:BottleMap}
  \end{figure}
\end{frame}


% ----------------------------------------------------------------

\begin{frame}
  \frametitle{}
  \begin{figure}
    \centering
    \includegraphics[width=1.0\textwidth]{\sab 5.png}
    \caption{Depth profiles of chlorophyll-a and nutrients; all data for the time period 1955-2014.}
    \label{fig:ChloroProfiles}
  \end{figure}
\end{frame}



% ----------------------------------------------------------------

\begin{frame}
  \frametitle{BioChem: Zooplankton}
  \begin{itemize}  
    \item Relevance:  productivity, biodiversity, species of interest, habitat (in relative order)
    \item Sampling:  AZMP surveys, Groundfish surveys, pelagic net tows, 400 taxonomic species
    \item Spatial coverage: 2367 net deployments, 126 missions 
    \item Temporal coverage:,  1999 to 2014, annual surveys
    \item Source code: https://github.com/jae0/ecomod/biochem/src/biochem.r
  \end{itemize}
\end{frame}


% ----------------------------------------------------------------

\begin{frame}
  \frametitle{}
  \begin{figure}[h]
    \centering
    \includegraphics[width=0.7\textwidth]{\sab 6.png}
    \caption{Total number of net deployments for each month during the time period 1999-2014.}
    \label{fig:AZMPdeploymentsMonthly}
  \end{figure}
\end{frame}


% ----------------------------------------------------------------


\begin{frame}[shrink]
\begin{figure}
  \centering
  \begin{tabular}{cc}
    \includegraphics[width=0.3\textwidth]{\sab 8.png}
    \includegraphics[width=0.3\textwidth]{\sab 9.png}
  \end{tabular}
  \begin{tabular}{cc}
    \includegraphics[width=0.3\textwidth]{\sab 10.png}
    \includegraphics[width=0.3\textwidth]{\sab 11.png}
  \end{tabular}
  \caption{Monthly averages of all data from 1999 to 2014: total abundance (top left), total biomass computed from wet weight (top right), ratio of total biomass computed from wet weight  to total abundance (bottom left) as a potential measure of the average weight of the individual organism, and abundance of \textit{Calanus finmarchicus}, \textit{Calanus hyperboreus}, and \textit{Calanus glacialis} (botton right) }
  \label{fig:AZMPBiomassMonthly}
\end{figure}
\end{frame}


% ----------------------------------------------------------------

\begin{frame}[shrink]
  % \frametitle{}
  \begin{figure}
    \centering
    \includegraphics[width=1.0\textwidth]{\sab 7.png}
    \caption{Spatial distribution of net deployments included in the zooplankton dataset. }
    \label{fig:AZMPdeploymentsMonthlyMap}
  \end{figure}
\end{frame}


% ----------------------------------------------------------------

\begin{frame}
  \frametitle{Remote Sensing Data: Chlorophyll-a}
    \begin{itemize}
      \item Relevance:  productivity, habitat, biodiversity and species of interest (in relative order) 
      \item Sampling:  Moderate Resolution Imaging Spectroradiometer (MODIS-Aqua; NASA, RSU)
      \item Spatial coverage: 39 N to 62.5 N and 42 W to 71 W, resolution of 1.5 km
      \item Temporal coverage: August 2002 to March 2015, 610 quarter-monthly (8-day) composite  images
      \item Source code:  \url{https://github.com/jae0/ecomod/remote.sensing/src/remoting.sensing.r}
    \end{itemize}
\end{frame}


% ----------------------------------------------------------------

\begin{frame}[shrink]
  \begin{figure}[ht]
    \centering
    \begin{tabular}{cc}
      \includegraphics[width=0.45\textwidth]{\sab 12.jpg}
      \includegraphics[width=0.45\textwidth]{\sab 13.jpg}
    \end{tabular}
    \begin{tabular}{cc}
      \includegraphics[width=0.45\textwidth]{\sab 14.jpg}
      \includegraphics[width=0.45\textwidth]{\sab 15.jpg}
    \end{tabular}
    \caption{MODIS semi-monthly Chl-a concentration showing spring bloom progression in the NW Atlantic in 2012. Note the intense bloom at St. Anns Bank during the last two weeks in March.}
    \label{fig:MapChlaBloomSpring}
  \end{figure}
\end{frame}


% ----------------------------------------------------------------

\begin{frame}[shrink]
  \begin{figure}[h]
    \centering
    \includegraphics[width=1.0\textwidth]{\sab 16.png}
    \caption{Chlorophyll-a concentration extracted from MODIS 8-day composite images for St.Anns Bank polygon for the time period 2002-2015.}
    \label{fig:modisChlaTS}
  \end{figure}
\end{frame}


% ----------------------------------------------------------------

\begin{frame}[shrink]
  \begin{figure}[h]
    \centering
    \includegraphics[width=0.8\textwidth]{\sab 17.png}
    \caption {Average Chorophyll-a concentration computed from 8-day composite images for St.Anns bank polygon for the time period 2002-2015. }
    \label{fig:ChlaSeasonal}
  \end{figure}
\end{frame}


% ----------------------------------------------------------------

\begin{frame}
  \frametitle{Remote Sensing Data: Primary production}
  \begin{itemize}
    \item Relevance:  productivity, habitat, biodiversity and species of interest (in relative order)
    \item Sampling:  MODIS, ...
    \item Spatial coverage: 39 N to 62.5 N and 42 W to 71 W, resolution of 1.5 km
    \item Temporal coverage: July 2002 to December 2014, 150 monthly  images
    \item Source code \url{https://github.com/jae0/ecomod/remote.sensing/src/remoting.sensing.r}
  \end{itemize}
\end{frame}


% ----------------------------------------------------------------

\begin{frame}[shrink]
  \begin{figure}[h]
    \centering
    \includegraphics[width=1.0\textwidth]{\sab 18.png}
    \caption{Annual monthly Primary Production (PP) computed from PP composite images for St.Anns bank polygon for the time period 2002-2014.}
    \label{fig:ppTSmonthly}
  \end{figure}
\end{frame}


% ----------------------------------------------------------------

\begin{frame}[shrink]
  \begin{figure}[h]
    \centering
    \includegraphics[width=1.0\textwidth]{\sab 19.png}
    \caption{Average Primary Production (PP) computed from monthly composite images for St.Anns bank polygon for the time period 2002-2014.}
    \label{fig:ppTSannual}
  \end{figure}
\end{frame}


\begin{frame}
  \frametitle{Remote Sensing Data: Sea Surface Temperature (SST)}
  \begin{itemize}
  \item Relevance:  productivity, habitat, biodiversity and species of interest (in relative order)
  \item Sampling:  Advanced Very High Resolution Radiometer (AVHRR; NOAA, RSU)
  \item Spatial coverage: 39 N to 62.5 N and 42 W to 71 W, resolution of 1.5 km
  \item Temporal coverage: December 1997 to March 2015, 845 8-day composite images
  \item Source code: \url{https://github.com/jae0/ecomod/remote.sensing/src/remoting.sensing.r}
  \end{itemize}
\end{frame}


% ----------------------------------------------------------------

\begin{frame}[shrink]
  
  \begin{figure}[h]
    \centering
    \begin{tabular}{cc}
      \includegraphics[width=0.5\textwidth]{\sab 20.jpg}
      \includegraphics[width=0.5\textwidth]{\sab 21.jpg} 
    \end{tabular}
    \begin{tabular}{cc}
      \includegraphics[width=0.5\textwidth]{\sab 22.jpg}
      \includegraphics[width=0.5\textwidth]{\sab 23.jpg}
    \end{tabular}
    \caption{Bi-weekly composites from AVHRR showing SST in the North West Atlantic in the spring of 2012, corresponding to the intense spring bloom at St.Anns bank shown in Figure~\ref{fig:ChlaSeasonal}.}
    \label{fig:SSTfromAVHRRmap}
  \end{figure}

\end{frame}



% ----------------------------------------------------------------

\begin{frame}[shrink]
  \begin{figure}[h]
    \centering
    \includegraphics[width=1.0\textwidth]{\sab 24.png}
    \caption{Sea Surface Temperature (SST) extracted from 8-day AVHRR composite images for St.Anns bank polygon for the time period 1997-2015.}
    \label{fig:SSTfromAVHRRts}
  \end{figure}
\end{frame}

% ----------------------------------------------------------------

\begin{frame}
  \begin{figure}[h]
    \centering
    \includegraphics[width=0.8\textwidth]{\sab 25.png}
    \caption{Average Sea Surface Temperature (SST) computed from 8-day AVHRR composite images for St.Anns bank polygon for the time period 1997-2015.}
    \label{fig:SstSeasonal}
  \end{figure}
\end{frame}

% ----------------------------------------------------------------

\begin{frame}
\frametitle{Bottom temperatures}
  \begin{itemize}
    \item Relevance:  productivity, habitat, biodiversity and species of interest
    \item Sampling:  Groundfish survey, snow crab survey, AZMP profiles 
    \item Spatial coverage: full extent, varied sampling
    \item Temporal coverage: 1950 - present (more historical data present but coverage is variable)
    \item Source code: https://github.com/jae0/ecomod/temperature/src/temperature.r
  \end{itemize}
\end{frame}

% ----------------------------------------------------------------

\begin{frame}
  \begin{figure}
    \centering
    \includegraphics[width=0.8\textwidth]{\ecomod temperature/maps/SSE/bottom.predictions/global/{temperatures.bottom}.png}
    \caption{Average bottom temperatures computed from all available data 1950-2016.}
    \label{fig:TemperatureBottomMap}
  \end{figure}
\end{frame}


% ----------------------------------------------------------------

\begin{frame}
\frametitle{Demersal fish and macro-invertebrates}
\begin{itemize}
  \item Relevance:  productivity, habitat, biodiversity and species of interest
  \item Sampling:  Groundfish survey, Snow crab survey 
  \item Spatial coverage
  \begin{itemize}
    \item Groundfish: full extent, random stratified, variable number of stations
    \item Snow crab: Colder water environment, geostatistical grids of $\sim$~10 minutes, $\sim$~400 stations 
  \end{itemize}
  \item Temporal coverage
  \begin{itemize}
    \item Groundfish: 2000 - present (started in 1970, but consistent sampling since 2000)
    \item Snow crab: 2005 - present (started in 1996, but consistent sampling since 2005)
  \end{itemize}
  
  \item Source code
  \begin{itemize}
    \item url{https://github.com/jae0/ecomod/groundfish/src/groundfish.r}
    \item url{https://github.com/jae0/ecomod/snowcrab/src/1.snowcrab.r}
  \end{itemize}
\end{itemize}
\end{frame}

% ----------------------------------------------------------------

%\begin{frame}[shrink]
%\end{frame}

% ----------------------------------------------------------------

%\begin{frame}[shrink]
%\end{frame}

% ----------------------------------------------------------------

%\begin{frame}[shrink]
%\end{frame}

% ----------------------------------------------------------------

%\begin{frame}[shrink]
%\end{frame}

% ----------------------------------------------------------------

%\begin{frame}[shrink]
%\end{frame}

% ----------------------------------------------------------------

%\begin{frame}[shrink]
%\end{frame}

% ----------------------------------------------------------------
% ----------------------------------------------------------------
% ----------------------------------------------------------------

\end{document}
